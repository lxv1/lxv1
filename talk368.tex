Mail/Talk3.64.8

\font\ch=cmbx10 at 18truept
\font\sec=cmbx10 at 12truept

TALK March 6, 2008
\bigskip

\centerline {\ch  Linear Approximation, Matrix Games, }
\centerline {\ch  and Linear Programming}
\bigskip

Number = real (or rational) number

Definition. Linear form in variables  $x_1,\ldots, x_n  :$
$$ a_1x_1 + \cdots +a_nx_n$$
where $a_i $ are given numbers.

Example: $2x-3y+z $ is a linear form in $ x,y,z.$

Definition. Affine function:  linear form + given number.

Definition. A linear equation: affine function = affine function.
In standard form,  linear equation is

linear form = given number.

Example: $2x = 3.$

System of linear equations in standard matrix form: $Ax = b$
where $A$ is a given $m$ by $n$ matrix (the coefficient matrix) , $x$ is a column of $n$ distinct variables (unknowns), and  $b$ is a given column with $m$ entries.
\medskip

{\it Notations. } For any number $c,$ let  $c_{m\times n}$ denote the constant $m$ by $n$ matrix (every entry is $c).$  Following the tradition, the size is often dropped for the zero matrices.
The identity matrix is denoted $1_n.$
\medskip


{\sec 1. Linear Approximation}
\smallskip

Discrete Linear Approximation  (or linear regression) Problem is: 

find the best approximate solution to a system $Ax=b$  of linear equations. 

How do we measure the quality of approximation?
For any number    ${\rm p}>0$  and a    column  $y= (y_i), $ set
$$
||y|| _{\rm p}= \sum_i |y_i|^{\rm p}.
$$
If we want the triangular inequaity  $||y+z||_{\rm p}^{1/{\rm p}}  \le ||y||_{\rm p}^{1/{\rm p}}   + ||z||_{\rm p}^{1/{\rm p}}  ,$ then  
${\rm p} \ge 1.$ If we want to stay with rational numbers,  then ${\rm p}$ is an integer.

We also can define   $||y||_{\rm p}$ for ${\rm p} = \infty$ and ${\rm p} = 0$ as follows
$$
||y||_{\infty} =  \max |y| = \max_i  |y_i| = \lim _{{\rm p} \to \infty}  ||y||_{\rm p}^{1/{\rm p}} 
$$
(uniform of Chebyshev norm) and 

$$
||y||_0 = \sum {\rm sign}|y_i| = \lim _{{\rm p} \to 0}  ||y||_{\rm p}
$$
(Hamming or mode norm).


Thus, we are trying to find the best  (in the sense $||Ax-b||_{\rm p} \to \min)$ fit  $Ax$ to a given vector $b$  in the column space  of a given matrix $A.$ In other words, we want a point $Ax-b$ in a given affine subspace closest to the origin..


In statistics, usually, ${\rm p}=2$ (the least-squares criterion) and  the first column of the coefficient matrix
$A$ consists of ones.  
A reason that the choice ${\rm p}=2$ is so popular (in particular after Legandre 1805, Gauss, 1809)  is that the sum of squares has a big
group of linear automorphisms. Another reason is that  finding the best $l_2$ fit can be reduced to solving  the system $A^TAx=A^Tb$ of linear equations.


However, the choice $p=1$ (least-absolute-deviations) was popular  in the 18th century (e.g., 
Boscovic 1755, Laplace 1789) and more recently.  Fourier (about 1822)  considered both $p = 1$ and ${\rm p}= \infty.$ 
He also suggested the simplex method for finding the corresponding best fits. Dantzig  rediscover
simplex method which is now  the most popular way to solve linear programs.

Example: $m =1$ and  the column $A$ consists of ones. So our system is
$$x=b_i   \ {\rm for} \   i = 1,\ldots, m$$
($m$ equations for one variable) where $b = (b_i).$     

The least-squares fit (${\rm p}=2)$ is the mean  $(b_1+\cdots+b_m)/m.$ The least uniform fit (${\rm p}=\infty)$ is
the midrange $(\min(b) + \max(b))/2.$ The least-absolute-deviations fits (${\rm p}=1)$ are the medians.
In the case ${\rm p}=0,$ the best fits are the modes.

% \filbreak 
\bigskip 

{\sec 2. Matrix Games}.
\medskip

A matrix game is given by a (payoff) matrix  $P.$  
He, the row player chooses a row. She, the column player chooses a column. The corresponding matrix entry is what she pays to him.

Let $P$ be of size $m$ by $n$.
A mixed strategy for him is a row $p$ (do not mix with the number $p$ above)  with $m$ non-negative entries whose sum is 1. The corresponding worst case payoff is $\min(p^TP).$

A mixed strategy for her is a row $q$ with $n$ non-negative entries whose sum is 1. 
In the worst case, she pays him $\max(Aq^T).$

His optimal mixed strategy is a maximizer of  $\min(p^TP)$ over his mixed strategies.
Her optimal mixed strategy
is  a minimizer for    $\max(Pq^T)$ over her mixed strategies. 
 The minimax theorem of J. von Neumann asserts that
$\max\  \min(p^TP) =  \min  \max (Pq^T).  $   In other words, there is an equilibrium  $(p, q)$ in mixed strategies.
The payoff    $p^TPq^T$ at  an equilibrium is known as the value of the game. At an equilibrium, no player can do better by a unilateral change.


Example: The rock-scissors-paper.
 The payoff function is defined by the rules {\it Rock  breaks Scissors, Scissors cuts Paper, Paper covers Rock}, and every strategy ties against itself.   Valuing a win at $1,$ a tie at 0, and a loss at $-1$, we can represent the game with the following matrix, where, for both players, strategy $1$ is {\it Rock}, strategy $2$ is {\it Scissors}, and strategy $3$ is {\it Paper}:

$$P=\left[ \matrix{ 0 &  1&  -1 \cr  -1 &  0 &  1 \cr  1 &  -1 &  0 \cr} \right]. $$

His optimal strategy is $[1/3, 1/3, 1/3],^T$ Her optimal strategy is  $[1/3, 1/3, 1/3],$  The value of game is 0.

More generally, if $P = -P^T,$ then the game is called symmetric,  the value is 0, and his optimal strategies are the transposes of her optimal strategies.



If we add the same number $c$ to all entries of $P$ we do not change the optimal strategies, but we add $c$ to the values of game.  Every matrix game, with the payoff matrix $P$  of size $m$ by $n,$ is equivalent to the symmetric game
with the payoff matrix 
$$
M=\left[ \matrix{ 0_{n\times n} &  -P^T- c_{n\times m} &  1_{n\times 1} \cr  P +c_{m\times n}  &  0_{m\times m} &  -1_{m\times 1} \cr  -1_{1\times n} & 1_{1\times m} &  0 \cr} \right] 
$$
of size $m+n+1$ by   $m+n+1$ 
where   the number $c$ is chosen so that  all entries of $ P +c_{m\times n}$ are positive 
 (ie.g., $c = -\min(P) +1)$   so the value of  the corresponding matrix game  is positive.  

 

Namely, $(p,q), v$ are an equilibrium and the value for $P$  if and only if
$(q,p^T,v+c)$ is an optimal strategy for $M.$

As observed by von Neumann, there is another way to reduce any matrix game 
to a symmetric game:  Let them play, together  with the game $P,$ the game $-P^T,$
i.e., the game with switched sides.
For $P$ of size  $m$ by $n,$ this leads to a skew-symmetric matrix of size $mn$ by $mn.$  
Its entries are the products of the entries of $P.$

The   more compact symmetric game,  with  the payoff matrix $M$  of size $m+n+1$ by   $m+n+1$ 
was observed by Dantsig a few years later.
\filbreak
\bigskip

{\sec 3. Linear Programming}
\medskip
First, we give a few definitions

An optimization problem is  maximization or minimization of a real-valued function (called the objective function) on a set (called the feasible region or the set of feasible solutions).

A feasible value is a value of the objective function on the feasible set. An optimal value  (or optimum)  is
the maximal (resp., minimal) feasible value. An optimal solution (or optimizer ) is a feasible solution where the objective function take its optimal value.

A mathematical program is an optimization problem where the objective function is a function of finitely many real variables.

 A linear constraint is a linear equation or is obtained from a linear equation by replacing  the equl sign with the sign $\le$ or $\ge.$

  A linear program is an optimization problem with objective function being affine and the feasible region given by a finite system of linear constraints.

Example: 

$ 2x+3y \to \max$ subject to $|x|+|y| \le 1$

\noindent
is a linear program with 2 variables and 4 linear constraints.

Answer: $\max= 3$ at $x=0, y=1.$

In general, there are 3 possible answers to a linear program:

an optimal solution and the optimal value,

the program is infeasible, i.e., the feasible region is empty, i.e., there are no feasible solutions,

the program is unbounded, i.e., there are arbitrary large (resp., small) feasible values.

The duality theorem allows to reduce solving any linear program to finding  feasible solutions 
for system of linear constraints

Namely, given a linear program
$$cx+d \to \min, Ax +b \ge 0, x \ge 0 \eqno{(1)}$$
in canonical form we can consider the dual linear program    
$$-yb+d \to \max, -yA+c \ge 0, y \ge 0.$$
For any feasible $x, y,$ we have
$$-yb+d \le yAx +d \le cx+b,$$
i.e., every feasible value for the primal problem is less or equal to any feasible value for the dual problem.
The duality theorem asserts that if both problems are feasible, then the optimal values are the same.
Therefore the pairs $(x,y)$  of optimal solutions are in one to one correspondence with the feasible solutions for the system
$$Ax +b \ge 0, x \ge 0,  -yA+c \ge 0, y \ge 0,   cx+yb \le 0. \eqno{(2)}$$

\smallskip

Recent interior-point methods in linear programming work with the Karmarkar form:
$$cx \to \min, Ax = 0, x \ge 0,  1_{1\times m}x=1 \eqno{(3)}$$
where the $m$ by $n$ matrix $A$ has the property that  $A1_{n\times 1}  =1_{n\times 1} .$ This linear program is bounded and feasible ($x=1_{n\times 1} /n$ is a feasible solution).
Since there are unbounded and infeasible linear programs, reduction of any linear program to the 
Karmarkar form is an issue. The known conversions require  an upper bound on  the size
of a feasible solution provided that the program is feasible. Such a bound is available 
in terms of  numerators and denominators of given numbers provided that the given numbers are rational. Here we show how to do reduction without  referring to digits of data or to ``sufficiently large" numbers, so our reduction is independent on the size of data and works when given numbers are real numbers. 

Remark. An arbitrary  linear program with $m$ linear  constraints for $n$ variables can be easily written in canonical form (1) with a matrix $A$ of size at most $2m$ by $2n.$  A
 smaller canonical form can be computed  (using polynomial in  $m+n$ number of arithmetic operations)
 with a matrix $A$ of size at most $m$ by $n.$

 
\bigskip
\filbreak

{\sec 4. Reduction of $l_1$ and $l_{\infty}$ approximations to linear programming}
\medskip
Given a linear $l_{\infty}$ approximation problem   $||Ax-b||_{\infty} \to \min$ (where $A$ is an $m$ by $n$ matrix) we can reduce it to a linear program
$u \to \min, u_{m\times 1}  \le Ax-b \le u_{m\times 1}$ with $n+1$ variables and $2m$ linear constraints.
Here $u$ is a new variable.

This linear program can be converted to a canonical form(1) without any computations:
$$
\left[ \matrix{ A &  -A&  1_{m\times 1}    \cr -A  &A  &  1_{m\times 1}    \cr} \right] 
\left[ \matrix{  x'  \cr   x'' \cr  u    \cr} \right]  +\left[ \matrix{  -b \cr  b   \cr} \right]   \ge 0, \ 
x' \ge 0, x'' \ge 0, u \ge 0, u \to \min
$$
with $x=x'-x''.$  The coefficient matrix is of size $2m$ by $2n+1.$
By additional computations, a smaller canonical form can be obtained. Namely the size of the
coefficient matrix in canonical form can be reduced to   $2m$ by $d+1$ where $d$ is the rank of $A$ so
$d \le \min(m,n).$

Given a linear $l_1$ approximation problem   $||Ax-b||_1\to \min$ (where $A$ is an $m$ by $n$ matrix) we can reduce it to a linear program
$1_{1\times m} u = \sum_{i=1}^mu_i \to \min, u \le Ax-b \le u$ with $n+m$ variables and $2m$ linear constraints.
Here $u =  (u_i)$ is a column with $m$ new variables.

This linear program can be converted to a canonical form(1) without any computations:
$$
\left[ \matrix{ A &  -A&  1_m  \cr -A  &A  &  1_m \cr} \right] 
\left[ \matrix{  x'  \cr   x'' \cr  u    \cr} \right]  +\left[ \matrix{  -b \cr  b   \cr} \right]   \ge 0, \ 
x' \ge 0, x'' \ge 0, u \ge 0, 1_{1\times m} u \to \min
$$
with $x=x'-x''.$ The coefficient matrix is of size $2m$ by $2n+m.$
By additional computations, a smaller canonical form can be obtained. Namely the size of the
coefficient matrix in canonical form can be reduced to   $2m$ by $d+m$ where $d$ is the rank of $A$ so
$d \le \min(m,n).$

Remark. It is well-known, that a linear $l_2$ approximation problem 
  $||Ax-b||_2\to \min$ (where $A$ is an $m$ by $n$ matrix) we can reduce it to a linear system
  $A^TAx=A^Tb$ of $n$ equations for $n$ unknowns.
  
\bigskip
{\sec 5. Reduction of Matrix Games to Linear Programming}
\medskip
Given a payoff matrix $P$  with $m$ rows and $n$ columns, the optimal strategies $p$ for the row player are the optimal solutions for the 
linear program
$$ u \to \max, p^TP \ge u,   1_{1\times m}p=\sum_{i=1}^m p_i = 1, p \ge 0$$
 with  $m+1$  variables and $2m+1$ linear constraints.
Here $u$ is a new variable.

The optimal strategies $q$ for the column player are the optimal solutions for the linear program
$$ v \to \min,  Pq^T \le v,    q1_{n\times 1}=  \sum_{j=1}^m q_j = 1, q \ge 0$$
 with $n+1$  variables and $2n+1$ linear constraints.
Here $v$ is a new variable.

These two linear programs are dual to each other, so the minimax theorem follows from the duality theorem.

Now we show how to reduce  any matrix game to  the Karmarkar form (3). Without loss of generality, we can assume that the matrix game is symmetric.
Let $M = -M^T$ be the payoff matrix. We have to find an optimal strategy $q$ for the column player:
$$Mq \le 0, q \ge 0,  q1_{k\times 1}=1.$$
We introduce the slack variables  $v = -Mq \ge 0.$ The problem takes the form
$$[M, 1_k][{q \atop v}] = 0, q \ge 0,  q1_{k\times 1}=1, v \ge0.$$
where $1_k$ is the identity matrix.

Next we introduce the sum  $C=M1_{k\times 1}+ 1_{k\times 1}$   of column of the matrix   $[M, 1_k]$ an a new variable $u.$   Set $A = [   M, 1_k, 1_{k\times 1}-C] $    
Notice that  $A1_{2k+1\times 1}= 1_{2k+1\times 1}.$
Then our problem becomes
$$A\left[ \matrix{ q \cr   v  \cr  u \cr} \right]  =0,  q \ge 0, q1_{k\times 1}=1, v \ge 0, u \ge 0, u \to \min.$$
(the optimal value is $u=0).$  Now we set
$$x = \left[ \matrix{ q \cr   v  \cr  u \cr} \right]   /(1+1_{1\times k}v+u)$$
  to obtain the Karmarkar standard form. 
  Note that $q$ can be recovered for an optimal $x = [x',x'',u]^T$  as  $q = x'/1_{1\times k}x'$ because
the last component  $u = 0$ and $x'=0$ would imply $ x''=0.$
% \filbreak
\bigskip

 
{\sec 6. Reduction of Linear Programming to Matrix Games}
\medskip

We saw above that the optimal  solutions of a linear program and its dual linear program are the feasible solutions $(x,y)$ for the system (3)  of linear constraints
$$Ax +b \ge 0, x \ge 0,  -yA+c \ge 0, y \ge 0,   cx+yb \le 0.$$

This can be written as  
$$[y,x^T,1) M \ge 0 \ {\rm with} \  M=\left[ \matrix{ 0 &  -A & -b \cr  A^T&  0 &  -c^T \cr  b^T & c &  0 \cr} \right]. $$
When $A$ is of size $m$ by $n,$  the matrix $M$ is of size  $m+n+1$ by  $m+n+1.$

 When $(x,y)$ is a feasible solution,  $[y,x^T,1)  /(1_{1\times n}x+ y1_{m\times 1} +1)$ is an optimal strategy for the row player in a symmetric game  given by $M.$  The last entry in the strategy, 
 $ 1  /(1_{1\times n}x+ y1_{m\times 1} +1)$  is nonzero.
 Conversely, given an optimal strategy    $u = (u_i),$  i.e., $u \ge 0,  u1_{m+n+1\times 1}= 1, uM \ge 0$
 with $u_{m+n+1} \ne 0,$
 we  obtain $(x,y)$ by dropping the last entry 1 in $u/u_{m+n+1} .$

Thus, we obtain an 1-1 correspondence between 
the pairs $(x,y)$ of optimal solutions for a linear program and its dual program on one side
and the optimal strategies for the row player with the nonzero last entry.

A known way to remove  the the last condition depends on separation from 0 for the nonzero minors of $M$  which can be easily done when the entries of $M$ are rational numbers. Here we give another way which works for real numbers.

We set  $k=m+n+1$ and choose a number $c$ such that     all entries of the matrix $M+c_{k\times k}$ are positive, e.g., $ c = \max(M)+1 \ge 1.$

Consider the linear program  $u \ge 0,  u1_{k\times 1} = 1, u(M  +  c_{k\times k)} \ge  1_{1\times k},      u_k \to \max$
with $k$ variables $u = (u_i).$  Its optimal solutions are the optimal strategies $u$ for the symmetric game with the payoff matrix$M$ (the number $c$ does not matter) with the maximal values for the last entry of $u.$ If  its optimal values is 0, i.e., there is no optimal strategies with nonzero last entry, then
the original linear program has no optimal solutions. Otherwise, the optimal solution gives
an optimal solution for that program.

 It is convinient to replace the constraint $ u1_{k\times 1}= 1$ by  $ u1_{k\times 1}  \le 1$  which does not change the optimal value. This is because in the view of   the other constraints  
 there are no feasible solutions with  $ u1_{k\times 1}  <  1.$ 

The corresponding symmetric game has the following skew-symmetric payoff matrix of size $2k+2$ by $2k+2:$
$$
M' = \left[ \matrix{ 
0_{k\times k} & 0_{k\times k}  &   M+c_{k\times k}     & -c_{k\times 1} \cr  
0_{k\times k} & 0_{k\times k}  &         1_{1\times k}                 &  -1\cr 
M-c_{k\times k}       & -1_{k\times 1}  &        0_{k\times k}                &  {0_{k-1\times 1}   \atop 1}   \cr 
c_{1\times k} & 1 &         0_{1\times k-1},-1                  &0} \right]. 
$$

Now we claim that  the last component  $v_{2k+2} \ne 0$ of every optimal strategy  $v = (v_i)
=[v',v_{k+1}, v'',   v_{2k+2} ] $  where  $v' = (v_i)_{1\le i \le k}$ and    $v'' = (v_i)_{k+2\le i \le 2k+1}.$ 
Otherwise, the condition $vM' \ge    c_{2k+2\times 1}$ i would include   the condition
$-v''1_{l\times 1}\ge 0,$ hence  $v'' = 0.$ Then  looking at the last entry of  $vM' $ we conclude that 
$v', = 0$ and $v_{k+1},= 0.$  But $v=0$ contradicts the condition  $v1_{1\times 2k+2} =1.$

Thus, every optimal strategy $v$ for $M'$ solves the  problem 
of finding a feasible solution    for
$$Ax +b \ge 0, x \ge 0,  -yA+c \ge 0, y \ge 0,   cx+yb \le 0$$
(and hence finding an optimal solution for the linear program $cx+d \to \min, Ax +b \ge 0, x \ge 0.$)
Namely, if   the entry $v_{k+1}$ is zero, there are no feasible solutions.
Otherwise, we get an optimal strategy for  $M$ with a nonzero last component   and hence a feasible solution for the system .



%  \filbreak
\bigskip

{\sec 7. Reduction of Matrix Games to Linear Approximation}
\medskip
As shown above it suffices to show that solving any symmetric game
can be reduced to finding the best $l_1$ fit as well as the best $l_{\infty}$ fit. Let $P = - P^T$ be a skew symmetric $k$ by $k$ matrix.




We want to find an optimal strategy $p$ for the row player, i.e., a column $p$ such that
$$Pp \le 0,  p \ge 0, 1_{1\times k}p = \sum _{i=1}^k  p_i= 1.$$
Let $c=\max(P)$ be the maximal entry of the matrix $P.$
Consider the following $l_1$ approximation problem:
$$
|| \left[ \matrix{Pp \cr  c_{k\times 1}-Pp \cr  p \cr 1_{k\times 1}-p \cr 1_{1\times k}p -1} \right]||_1 =
||Pp||_1 + ||c_{k\times 1}-Pp||_1  +||p||_1+||1_{k\times 1}-p||_1+| 1_{1\times k}p  -1|   \to \min.
$$
Here the first column has $4k+1$ entries.

When $p$ is optimal, the objective function is 
$k+k+0 =2k.$ When $q$ is a column with $k$ entries which is not a mixed strategy or not optimal, then
the objective function  is $>2k.$ So the  optimal solutions $p$ for the approximation problem are the same as optimal strategies $p.$ 

Thus,  $m$ by $n$ matrix game can be reduced to an $l_1$ approximation problem of size 
$4k+1$ by $k,$ where $k = m+n+1.$

\smallskip

Now we consider the following $l_{\infty}$ approximation problem:
$$
|| \left[ \matrix{Pp+c_{k\times 1} \cr  c_{k\times 1}-p  \cr    c-1+ 1_{1\times k}p   \cr c+1- 1_{1\times k}p } \right]||_{\infty} =
\max(||Pp+ c_{k\times 1}||_{\infty} ,  ||c_{k\times 1}-p||_{\infty},      c-1+   1_{1\times k}p   ,   c+1-   1_{1\times k}p  )   \to \min.
$$
The first column has $2k+2$ entries.

When $p$ is optimal, the objective function is  $\max(c,c,c,c)=c.$
 When $q$ is a column with $k$ entries which is not a mixed strategy or not optimal, then
the objective function  is $>c.$ So the  optimal solutions $p$ for the approximation problem are the same as optimal strategies $p.$ 


Thus,  $m$ by $n$ matrix game can be reduced to an $l_1$ approximation problem of size 
$2k+2$ by $k,$ where $k = m+n+1.$


% \bigskip
 \filbreak

{\sec 8. Reduction of  $l_1$  Approximation to  $l_{\infty}  $ Approximation }
\medskip

We start with   
$$
||Ax-b||_1       \to \min
$$
where the matrix $A$ is size $m$ by $n.$

By Section  4, without any computations, we can write a linear program in canonical form with the coefficient matrix 
of size   $2m$ by $2n+m.$  

By Section 6,     this linear program can be transformed to a symmetric matrix game of size $2l+2$ by $2l+2,$ where  $l=2n+3m+1.$

By Section 7,    this symmetric game can be reduced to a linear    $l_{\infty}  $ approximation
of size $2(2l+2) +2$ by $2l+2,$

Thus,   $l_1  $ approximation problem of size $m$ by $n$
can be converted to   $l_{\infty}  $ approximation problem of size 
$8n+ 12m + 10$    by $4n+6m+3.$



\bigskip
% \filbreak


{\sec 9. Reduction of  $l_{\infty}  $  Approximation to   $l_1$    Approximation }
\medskip

We start with   
$$
||Ax-b||_{\infty}        \to \min
$$
where the matrix $A$ is size $m$ by $n.$

By Section  4, without any computations, we can write a linear program in canonical form with the coefficient matrix 
of size   $2m$ by $2n+1.$  

By Section 6,     this linear program can be transformed to a symmetric matrix game of size $2l+21$ by $2l+2,$ where  $l=2m+2n+2.$

By Section 7,    this symmetric game can be reduced to a linear    $l_1  $ approximation
of size $4l(2l+2)+1l$ by $2l+2$

Thus,  $l_{\infty}  $   approximation problem of size $m$ by $n$
can be converted to   $l_1  $    approximation problem of size 
$16m+ 16n + 17$    by $4m+4n +6.$


\bigskip
% \filbreak


{\sec 10. Reduction of  Linear Programming to Linear   Approximation }
\medskip

By Section 6,  any linear program (1)  in canonical form with an $m$ by $n$ coefficient matrix $A$ can be reduced to
a symmetric  $2k+2$  by $2k+2$ matrix game where
$k=m+n+1.$ By Section 7, this game can be reduced to 
an     $l_1$ approximation problem of size $4(2k+2)1$   by  $2k+2$ as well  as to
an $l_{\infty}$ approximation problem of size $2(2k+2)+2$  by $2k+2.$

Thus, the linear program (1) of size  $m$ by $n$ can be reduced to  
an     $l_1$ approximation problem of size $8(m+n+2)$   by  $2(m+n+2)$ as well  as to
an $l_{\infty}$ approximation problem of size $4m+4n+10$  by $2(m+n+2.)$


This shows the computational complexity of linear programming is the same as that of 
$l_1$ approximation  as well as that of    $l_{\infty}$ approximation. The complexity here is measured in terms of elementary operations with given numbers (without referring to their digits).
Presently, it is unknown whether  linear programming problems have less that exponential complexity in terms of $m+n$ where $n$ is the number of variables and $m$ is the number of linear constraints.
See Problem 9 in [6] about existence of  a strongly polynomial time algorithm for linear programming.
The known  versions of the simplex method  have  exponential running time in the worst case.

It is known the the running time is polynomial in terms of  $m,n,$ and the number of bits  in data.
(i.e., there    exist    weakly polynomial time algorithms for
 linear programming).   

It is shown in [1] that a bounded feasible linear program  (with a given bound) can be reduced to an  
$l_1$ approximation .
In spite  of this, on page 169,  
discussing the existence of exponential examples for known methods for 
$l_1$ approximation , the authors
 claimed that ``the truth of the matter is still an open question." 
 
 The present paper  confirms that this question is equivalent to the similar question in linear programming.

 \bigskip

\filbreak

{\sec References.}
\medskip


1.  Bloomfield, Peter and   Steiger, William L. {\it 
Least absolute deviations.
Theory, applications, and algorithms.}    Progress in Probability and Statistics, 6.
Birkh�user Boston, Inc., Boston, MA, 1983. xiv+349 pp. ISBN 0-8176-3157-7.

2. Dantzig, George B. {\it Linear programming and extensions. } Princeton University Press, Princeton, N.J. 1963 xvi+625 pp. 

3.  
Ha\v cijan, L. G.,  A polynomial algorithm in linear programming. (Russian)  {\it Dokl. Akad. Nauk SSSR}
244 (1979), no. 5, 1093--1096. 

4.  
Karloff, Howard {\it Linear programming. } Progress in Theoretical Computer Science. Birkh�user Boston, Inc., Boston, MA, 1991. viii+142 pp. ISBN: 0-8176-3561-0  

5. 
 Karmarkar, N.,  A new polynomial-time algorithm for linear programming. {\it Combinatorica}  4
(1984), no. 4, 373--395.



6. Smale, Steve, {\it Mathematical problems for the next century}. Mathematics: frontiers and perspectives, pp. 271�294, American Mathematics Society, Providence, RI (2000).

7. L.N. Vaserstein,  
Matrix Games, Linear Programming, and Linear Approximation, 5 pp. 

\noindent
arXiv:cs/0609056 



\end