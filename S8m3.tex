\documentstyle{article}
\begin{document}
Find:

normal form,

the equilibria in pure strategies,

the Pareto optimal pure payoffs,

the Nash bargaining solution,

the Shapley values,

an imputation in the core.


\begin{picture}(100,200)(10,20)

\put(95,195){A}
\put(95,100){B}
\put(140,120){C}


\put(65,155){1} % A1
\put(90,150){2} %A2
\put(125,155){3} % A3


 \put(100,190){\vector(-2,-3){52}} % A1
\put(100,190){\vector(0,-1){80}} % A2
 \put(100,190){\vector(2,-3){40}}  % A3
 
  \put(45,110){\vector(-2,-3){50}} % prob 1/3
\put(45,110){\vector(2,-3){49.6}} % prob 2/3
 \put(150,120){\vector(2,-3){50}} % C2
  \put(150,120){\vector(-3,-1){45}} % C1
  
    \put(100,95){\vector(0,-1){60}} % B1
        \put(100,95){\line(1,-1){62}}       % B2 1
   \put(80,20){\oval(165,40)[b]}  % B2 2
          \put(-3,20){\vector(0,1){5}}  % B2 4
      \put(162.5,20){\line(0,1){13}} % B2 3
        
 
\put(0,65){1/3} %  prob
\put(50,65){2/3} % prob
\put(87,65){1} % B1
\put(115,65){2} % B2
\put(115,115){1} % C1
\put(180,85){2} % C2


\put(-20,27){0, 0, 3} %terminal
\put(85,27){0, -3, 0} %terminal
\put(190,35){1, 0, 2} %terminal

\end{picture}

\bigskip
\bigskip
\bigskip

See the next page for solutions.
\eject

Solution. 
There are 12 strategy profiles:

strategy       \ \ \ \ \   payoff

A B C    \ \ \ \ \     \  A   \ B \ C

1 1 1 \ \ \  \ \  \ \ \ \ 0\  - 2\  1  

1 1 2\ \ \  \ \  \ \ \ \ \ 0\  - 2\  1  

1 2 1 \ \ \  \ \  \ \ \ \ 0\  - 2\  1  

1 2 2 \ \ \  \ \  \ \ \ \ 0\  - 2\  1  

2 1 1 \ \ \  \ \  \ \ \ \ 0\  - 3\  0

2 1 2\ \ \  \ \  \ \ \ \ \ 0\  - 3\  0

2 2 1 \ \ \  \ \  \ \ \ \ 0\  \ 0\  \ 3

2 2 2 \ \ \  \ \  \ \ \ \ 0\  \ 0\  \ 3  

3 1 1 \ \ \  \ \  \ \ \ \ 0\  - 3\  0

3 1 2\ \ \  \ \  \ \ \  \ \ 1\  \ 0\  \  2

3 2 1  \ \  \ \ \  \ \ \ \ 0\ \   0\  \  3

3 2 2 \ \ \  \ \  \ \ \ \ 1\  \ 0\  \  2

There are 5 equilibria in pure strategies: (1,1,1), (1,2,1), (2,2,1), (3,1,2), and (3,2,1).

There are 2 Pareto optimal  pure  payoffs: (0, 0, 3) and (1, 0, 2).

The characteristic function v is :

v(empty) = 0, v(A,B,C) = 3,

 v(A) = 0, v(B) = -2, v(C) = 0, 
 
 v(A,B) = 0, v(A,C) = 3, v(B,C) = -1.
 \bigskip
 
 The initial point for the Nash bargaining:
 (v(A),v(B), v(C)) = (0, -2, 0).
$$ a(b+2)c \to \max, a \ge 0,  b \ge -2, c \ge 0, $$
where $(a,b,c) $ is a mixture of  (0, 0, 3) and (1, 0, 2) so $b = 0.$
Since $a + c = 3 $ and we want to maximize $ac,$  the optimal solution on
the line ist  $a = c = 3/2.$ It is outside the interval of Pareto optimal solutions.
The closest point in the interval is 
the arbitration triple (1, 0, 2).

 \bigskip
 
\ \       \ \ \ \ \ \ \ \ A B C
        
 ABC     \ \ \  0 \ 0 3
 
 ACB \ \ \ 0\ \ 0 3
 
 BAC \ \ \ 2 -2 3
 
 BCA \ \ \   4 -2 1
 
 CAB \ \ \ 3 \ 0 0
 
 CBA \ \ \  4 -1 0
   
\ \ \ \ \ \ \ \ \ \   ${13 \over 6}   {-5 \over 6}  {10 \over 6}     $ \ \ \ \  the  Shapley values
 

 \bigskip
 
 Core:
 $$ a \ge 0,  b \ge -2, c \ge 0, a+b+c = 3, a+b \ge 0, a+c \ge 3, b + c \ge -1.$$
 
 
So  (a,b,c) = (4, -2,1) belongs to the core.

\end{document}





\end