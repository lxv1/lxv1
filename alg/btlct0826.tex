%
\magnification=\magstep1
\input amstex
\documentstyle{amsppt}
\NoBlackBoxes
\pageheight{7.5in}
\pagewidth{5.5in}

\leftheadtext{George E. Andrews}
\TagsOnRight
\def\c{\cite}
\def\fr{\frac}
\def\pf{\hfill $\square$}
\def\wa{\widetilde{\alpha}}
\def\wb{\widetilde{\beta}}

\topmatter
\title
	Bailey's Transform, Lemma, Chains and Tree
\endtitle
\author
	George E. Andrews\footnote{Partially 
	supported by National Science Foundation Grant
	DMS-9206993. \hfill\hfil }
\endauthor
\address
	The Pennsylvania State University  
	University Park, PA  16802
\endaddress
\abstract
	In this paper we shall provide a brief survey of the work
	begun by L. J. Rogers and W. N. Bailey which has led to an
	iteratave method for producing infinite chains of $q$-series
	identities.  Apart from providing the reader with leads to the
	study of previous accomplishments, we shall emphasize the
	importance of examination of the seminal works in order to
	discern topics open to further development.  This will lead us
	directly to a new construct: the Bailey tree.
\endabstract
\endtopmatter


\document

\subhead
1. \ Introduction
\endsubhead

Bailey chains have their genesis in the work of L. J. Rogers.  He
first proved the following identities in 1894 \c{30}, and they have
become known as the Rogers-Ramanujan identities.
$$
	\sum_{n=0}^{\infty} \fr{q^{n^2}}{(q;q)_n} =
	\fr1{(q,q^4;q^5)_{\infty}}
\tag{1.1}
$$
and
$$
	\sum_{n=0}^{\infty} \fr{q^{n^2+n}}{(q;q)_n} = 
	\fr1{(q^2,q^3;q^5)_{\infty}}
\tag{1.2}
$$
where
$$
	(A)_n = (A;q)_n = \prod_{j=0}^{n-1} (1 - Aq^j),
\tag{1.3}
$$
and
$$
	(A_1,A_2,\dots,A_r;q)_n = \prod_{n=1}^r (A_j;q)_n\,.
\tag{1.4}
$$

We shall provide a proof of (1.1) and (1.2) in Section 2.  It is
essentially Bressoud's simplification of Rogers' second proof \c{Br}.
This will provide us with the simplest possible example of a Bailey
chain (cf. \c5, \c{7; Ch.3})

\subhead
{Definition}
\endsubhead

A General Bailey Chain is an infinite sequence $(i \geqq 0)$ of pairs
of sequences of rational functions $\{\alpha_n^{(i)}\}_{n\geqq 0}$ and
$\{\beta_n^{(i)}\}_{n\geqq 0}$ of two variables $a$ and $q$ and
possibly others.
$$
	(\alpha_n,\beta_n) \to (\alpha_n',\beta_n') \to 
	(\alpha_n'',\beta_n'') \to \cdots
\tag{1.5}
$$

There is an identity independent of $i$ connecting
$\alpha_n^{(i)}$ and $\beta_n^{(i)}$, say
$$
	\beta_n^{(i)} = F_n (\alpha_0^{(i)},\alpha_1^{(i)},\alpha_2^{(i)},
	\dots,\alpha_n^{(i)})\,; \qquad (i \geqq 0)
\tag{1.6}
$$
furthermore there are given rules of construction
$$
	\beta_n^{(i)} = G_n (\beta_0^{(i-1)},\dots,\beta_n^{(i-1)})
	\qquad (i \geqq 1)
\tag{1.7}
$$
and
$$
	\alpha_n^{(i)} = H_n (\alpha_0^{(i-1)},\dots,\alpha_r^{(i-1)})
	\qquad (i \geqq 1)\,,
\tag{1.8}
$$
again $G_n$ and $H_n$ are independent of $i$.
\vskip .1in
Put in this abstract setting, it is hard to discern the relationship
of this iterative process with the Rogres-Ramanujan identities (1.1)
and (1.2).  This will become clear in Section 2.  Section 3 will look
at the overarching method dubbed by L. J. Slater, the Bailey transform
\c{38; p. 58}.  Section 4 will provide a summary of how this idea has
been broadly applied.  In Section 5, we consider the Milne-Lilly
extensions \c{25}-\c{27}, and other extensions \c{12}.  We also return
to a look at Rogers' second paper \c{31} with an eye to extensions as
described in \c8.  In Section 6, we reconsider Bailey's first papers
on this topic; this leads us to new Bailey chains in Section 7 and
eventually to the Bailey tree in Section 8.

\subhead
2. \ The Bressoud-Rogers Proof
\endsubhead

In this section we will provide the simplest known example of a
non-trivial Bailey chain.

\proclaim
{Thereom 1}  Let
$$
	\alpha_n^{(0)} = \fr{(1 - aq^{2n})(a;q)_n (-1)^n q^{\binom{n}{2}}}
	{(1 - a)(q;q)_n} \;
\tag{2.1}
$$
and
$$
	\beta_n^{(0)} = \delta_{n,0}\,.
\tag{2.2}
$$
Furthermore for $i > 0$ 
$$
	\alpha_n^{(i)} = a^n q^{n^2} \alpha_n^{(i-1)}
\tag{2.3}
$$
and 
$$
	\beta_n^{(i)} = \sum_{j=0}^n \fr{q^{j^2} a^j \beta_j^{(i-1)}}
	{(q;q)_{n-i}}\;.
\tag{2.4}
$$
Then $\{\alpha_n^{(i)}\}_{n \geqq 0}$, $\{\beta_n^{(i)}\}_{n\geqq 0}$
$(i \geqq 0)$ form a Bailey Chain.
\endproclaim

\demo
{Proof}  The $G_n$ and $H_n$ of (1.7) and (1.8) are clearly defined
by (2.3) and (2.4).  We shall show that 
$$
	F_n(x_0,x_1,\dots,x_n) = \sum_{j=0}^n \fr{x_j}{(q;q)_{n-j}
	(aq;q)_{n+j}}\;.
\tag{2.5}
$$
The treatment of the assertion that
$$
	\beta_n^{(0)} = \sum_{j=0}^n \fr{\alpha_j^{(0)}}{(q;q)_{n_j}
	(aq;q)_{n+j}}
\tag{2.6}
$$
follows immediately from the fact (proved easily by induction on
$M$) that
$$
	\sum_{j=0}^M \fr{(a;q)_j(1 - aq^{2j})(q^{-n};q);q^{nj}}
	{(q;q)_j(aq^{n+1};q)_j(1 - a)} = \fr{(aq;q)_M q^{nM} 
	(q^{1-n};q)_M}{(q;q)_M (aq^{n+1};q)_M}
\tag{2.7}
$$

Now we assume that (1.6) holds for $i - 1$ with $F_n$ given by (2.5).
Hence
$$
\align
	\beta_n^{(i)} & = \sum_{j=0}^n \fr{q^{j^2} a^j \beta_j^{(i-1)}}
	{(q;q)_{n-j}}   \\
	& = \sum_{j=0}^n \fr{q^{j^2}a^j}{(q;q)_{n-j}} \;\sum_{r = 0}^j
	\fr{\alpha_r^{(i-1)}}{(q;q)_{j-r}(aq;q)_{j+r}}  \\
	& = \sum_{r=0}^n \sum_{j=r}^n \fr{q^{j^2}a^j \alpha_r^{(i-1)}}
	{(q;q)_{n-j}(q;q)_{j-r}(aq;q)_{j+r}}   \\
	& = \sum_{r=0}^n \sum_{j=0}^{n-r} \fr{q^{j^2 + 2jr}a^j \alpha_r^{(i)}}
	{(q;q)_{n-j-r} (q;q)_j (aq;q)_{j+2r}}   \\
	& = \sum_{r=0}^n \fr{\alpha_r^{(i)}}{(q;q)_{n-r}(aq;q)_{2r}}  \\
	& \qquad\qquad \sum_{j\geqq 0}  \fr{(q^{-n+r};q)_j (-1)^j 
	q^{(n-r)j+\binom{j+1}{2} + 2jr}}{(q;q)_j(aq^{2r+1};q)_j}   \\
	& = \sum_{r=0}^r \fr{\alpha_r^{(i)}}{(q;q)_{n-r} (aq;q)_{2r}}
	\;\fr1{(aq^{2r+1};q)_{n-r}}   \\
	& \qquad\qquad \text{(by \c{23; p. 11, eq (1.5.2) with $b \to \infty$
		})}   \\
	& = \sum_{r=0}^n \fr{\alpha_r^{(i)}}{(q;q)_{n-r} (aq;q)_{n+r}}\,,
	\tag2.8
\endalign
$$
which proves that (1.6) hodls for all $i$.  \pf
\enddemo

The first couple of instances of (1.6) reveal the power of Bailey
Chains.  When $i = 1$, 
$$
	\beta_n^{(1)} = \fr1{(q;q)_n}
\tag2.9
$$
by (2.4), and
$$
	\alpha_n^{(1)} = \fr{(1 - aq^{2n})(a;q)_n (-a)^n
	q^{n(3n-1)/2}}{(1 - a) (q;q)_n}\;.
\tag2.10
$$
Next,
$$
	\beta_n^{(2)} = \sum_{j=0}^n \fr{q^{j^2} q^j}{(q)_{n-j}(q)_j}\;,
\tag2.11
$$
and
$$
	\alpha_n^{(2)} = \fr{(1 - aq^{2n})(a;q)_n(-1)^n a^{2n}
	q^{n(5n-1)/2}}{(1 - a)(q;q)_n}\;.
\tag2.12
$$

Now by Theorem 1, we know that
$$
	\beta_n^{(2)} = \sum_{r=0}^n \fr{\alpha_r^{(2)}}{(q;q)_{n-r}
	(aq;q)_{n+r}}\;,
\tag2.13
$$
and therefore
$$
	\beta_{\infty}^{(2)} = \fr1{(q;q)_{\infty}(aq;q)_{\infty}}
	\sum_{r=0}^{\infty} \alpha_r^{(2)}\,,
\tag2.14
$$
i.e. after multiplication by $(q;q)_{\infty}$, we obtain
$$
	\sum_{n=0}^{\infty} \fr{a^n q^{n^2}}{(q;q)_n} =
	\fr1{(aq;q)_{\infty}} \sum_{n=0}^{\infty}
	\fr{(a;q)_n(1 - aq^{2n})(-1)^n a^{2n} q^{n(5n-1)/2}}
	{(q;q)_n (1 - a)}
\tag2.15
$$

From this familiar formula and Jacobi's Triple Product Indentity
\c{23; p. 12, eq. (1.6.1)} we obtain (1.1) by setting $a = 1$, and
(1.2) by setting $a = q$.

This process was essentially carried out by D. Bressoud in \c{19}
(cf. \c{17; Sec. 3.4}).  It is a limiting case of the proof given
in Chapter 3 of \c7.  We go through it here to illustrate as 
cleanly as possible precisely how a Bailey Chain works.

\subhead
3. \ The Bailey Transform
\endsubhead

The actual transformation that we performed in (2.8) was treated in
a fully abstract manner by Bailey in \c{15}.  L. J. Slater \c{38; p. 58}
has christened this the Bailey Transform:

If
$$
	\beta_n = \sum_{r=0}^n \alpha_r u_{n-r} v_{n+r}\,,
\tag3.1
$$
and 
$$
	\gamma_n = \sum_{r=n}^{\infty} \delta_r u_{r-n} v_{r+n}\,,
\tag3.2
$$
then
$$
	\sum_{n=0}^{\infty} \alpha_n \gamma_n = \sum_{n=0}^\infty
	\beta_n \delta_n\,.
\tag3.3
$$

As Bailey notes \c{15; p. 1} the proof is a mere series 
rearrangement.
$$
\align
	\sum_{n=0}^{\infty} \alpha_n \gamma_n & = \sum_{n=0}^{\infty}
	\sum_{r=n}^{\infty} \alpha_n \delta_r u_{r-n} v_{r+n}  \tag3.4
	\\
	& = \sum_{r=0}^{\infty}
	\sum_{n=0}^{r} \delta_r \alpha_n  u_{r-n} v_{r+n}  \tag3.5
	\\
	& = \sum_{r=0}^{\infty} \delta_r \beta_r\,.  \tag3.6
\endalign
$$

In (2.8), the Bailey Transform underlies the process in the 
instance $u_n = 1/(q;q)_n$, $v_n = 1/(aq;q)_n$, $\delta_r =
a^r q^{r^2}$, $\gamma_n = a^n q^{n^2}/(aq;q)_{\infty}$.

In Section 4, we shall examine other instances of the Bailey Transform
that lead to other Bailey Chains.  Indeed, one of the reasons for a
separate section on the Bailey Transform is to distinguish it from
Bailey's Lemma (an instance of the Bailey Transform [cf. (4.1)-(4.4)
in Section 4]) and Bailey Chains (as defined in Section 1).

Section 4 will only consider the case $u_n = 1/(q;q)_n$, $v_n =
1/(aq;q)_n$, and in Section 5 we shall restrict ourselves to natural
multidimensional extensions of this instance. However, there are
several quite distinct instances of Bailey's Transform in the 
literature, and we shall touch briefly on a few.

D. Bressoud \c{18} was perhaps the first to obtain really striking
variations.  He chose
$$
	u_n = \fr{(\beta;q)_n}{(q;q)_n} \quad,\quad v_n = 
	\fr{(\alpha\beta;q)_n}{(q;q)_n} \quad,\quad \text{ and }
	\delta_n = r^n\,.
\tag3.7
$$
Perhaps his most appealing application is the discovery of two new
polynomial identities which imply the Rogers-Ramanujan identities:
$$
	\sum_{m=0}^N q^{m^2} \bmatrix N \\ m \endbmatrix =
	\sum_{m=-\infty}^{\infty} (-1)^m q^{m(5m+1)/2}
	\bmatrix 2N \\ N + 2m \endbmatrix
\tag3.8
$$
and
$$
	(1 - q^N) \sum_{n=0}^{\infty} q^{m^2 + m} \bmatrix N -1 \\
	m\endbmatrix = \sum_{m=-\infty}^{\infty} (-1)^m q^{m(5m+3)/2}
	\bmatrix 2N \\ N + 2m + 1 \endbmatrix
\tag3.9
$$
where
$$
	\bmatrix A \\ B \endbmatrix = \bmatrix A \\ B \endbmatrix_q
	= \left\{
	\aligned
	\fr{(q;q)_A}{(q;q)_B (q;q)_{A-B}} \qquad & \text{ if } 0 \leqq
		A \leqq B  \\
	0 \qquad & \text{ otherwise }\,.
	\endaligned  \right.
\tag3.10
$$

In \c{10}, it was pointed out that one of D. B. Sears' most useful
transformations \c{35; p. 167, Th. 3} was the following instance 
of the Bailey Transform:
$$
\aligned
	u_n & = \fr{(ab/e;q)_n}{(q;q)_n}\quad,\quad v_n = 1\,,\quad
	\gamma_s = \sum_{t=0}^{\infty} \fr{(ab/e;q)_t \delta_{s+t}}
	{(q;q)_t}
	\\
	\alpha_n & = \fr{(e/a;q)_n (e/b;q)_n\left(\fr{ab}{e}\right)^n}
	{(q;q)_n (e;q)_n}\quad,\quad \beta_n = \fr{(a;q)_n (b;q)_n}
	{(q;q)_n (e;q)_n}
\endaligned
\tag3.11
$$

Following up on this \c{10}, one can obtain an $(r-1)$-fold expansion of
$$
\align
	_{r+1}\phi_r & \pmatrix c_0,c_1,\dots,c_r;q, \fr{f_1 f_2 \cdots f_r}
	{c_0 c_1\cdots c_r} \\ f_1,\dots,f_r \endpmatrix   \\
	: = & \sum_{n\geqq 0} \fr{(c_0,c_1,\dots,c_r;q)_n}{(q,f_1,
	\dots, f_r;q)_n} \;\left( \fr{f_1 f_2 \dots f_r}{c_0 c_1 \dots c_r}
	\right)^n\;.   \tag3.12
\endalign
$$

When $r = 2$, the result reduces to
$$
\align
	_3\phi_2 & \pmatrix c_0,c_1,c_2;q, \fr{f_1 f_2}{c_0 c_1 c_2}
	\\ f_1,f_2 \endpmatrix   \\
	= & \fr{\left(\fr{f_2}{c_2}\;,\; \fr{f_1 f_2}{c_0 c_2} ; 
		q\right)_{\infty}}{\left(f_2, \fr{f_1 f_2}{c_0 c_1 c_2}
		\;;\; q\right)_{\infty}} \; _3\phi_2 \;
		\pmatrix f_1/c_0,f_1/c_1,c_2;q,f_2/c_2 \\
		f_1,\fr{f_1 f_2}{c_0 c_1}\endpmatrix  \tag3.13
\endalign
$$
a familiar and useful formula \c{23, p. 62, eq. (3.2.7)}.

Very much related to this is a series of papers by Schilling and
Warnaar \c{32}, \c{33}, \c{34} in which they also concentrate on the
$\gamma_n$ and $\delta_n$ and consider multi-dimensional extensions
thereof.  Also a variation on $\gamma_n$ and $\delta_n$ appears 
in Bailey's paper \c{15; $\S$5}.

In this same general area, but not explicitly following from the
Bailey Transform is the work in \c9 on a $q$-trinomial coefficient
analog of Bailey's Lemma.  In \c{39}, O. Warnaar shows how this work is
closely related to the classical Bailey Lemma.

\subhead
4. \ One-Dimensional Bailey Chains
\endsubhead

The proof of the Rogers-Ramanujan identities given in Section 2 is in
fact a very special case of the following result known as Bailey's
Lemma \c5, \c{7; Ch. 3}.

If for $n\geq 0$
$$
	\beta_n = \sum_{r=0}^n \fr{\alpha_r}{(q;q)_{n-r}(aq;q)_{n+r}}\;,
\tag4.1
$$
then
$$
	\beta_n' = \sum_{r=0}^n \fr{\alpha_r'}{(q;q)_{n-r}(aq;q)_{n+r}}\;,
\tag4.2
$$
where
$$
	\alpha_r' = \fr{(\rho_1;q)_r(\rho_2;q)_r(aq/\rho_1\rho_2)^r \alpha_r}
	{(aq/\rho_1;q)_r(aq/\rho_2;q)_r} 
\tag4.3
$$
and
$$
	\beta_n' = \sum_{j\geq 0} \fr{(\rho_1;q)_j(\rho_2;q)_j
		(aq/\rho_1\rho_2;q)_{n-j}(aq/\rho_1\rho_2)^j \beta_j}
		{(q;q)_{n-j}(aq/\rho_1;q)_n(aq/\rho_2;q)_n}\;.
\tag4.4
$$

We shall suppress the proof of this; it is very similar to the proof
of (2.8) (cf. \c{7; Ch. 3}).  This, of course, can be iterated
immediately to produce a Bailey Chain as described in Section 1.

This result as outlined (but not stated) by Bailey himself \c{15; $\S$4}
uses the Bailey Transform in the case where
$$
\aligned
 	u_n & = \fr1{(q;q)_n} \quad v_n = \fr1{(aq;q)_n}   \\
	\delta_n & = \fr{(\rho_1, \rho_2, q^{-N};q)_n q^n}{(\rho_1
	\rho_2 q^{-N}/a;q)_n}\;,  \\
	\gamma_n & = \fr{(aq/\rho_1,aq/\rho_2;q)_N}{(aq,aq/(\rho_1\rho_2);q)_N}
	\;\; \fr{(\rho_1,\rho_2, q^{-N};q)_n \left(\fr{-aq}{\rho_1\rho_2}
	\right)^n q^{nN-\binom{n}{2}}}{(aq/\rho_1, aq/\rho_2,aq^{N+1};q)_n}
\endaligned
\tag4.5
$$

Subsequently this Bailey Chain has been applied to prove many 
disparate results.  Its original application \c3 (cf. \c{7; Ch 3})
was to provide an analog of Watson's $q$-analog of Whipple's 
Theorem \c{40} for 
$$
	_{2k+r}\phi_{2k+3} \pmatrix a,q\sqrt{a},-q\sqrt{a},b_1,c_1,
	b_2,c_2,\dots,b_k, c_k; q^{-N};q, \fr{a^k q^{k+N}}{b_1 c_1
	\cdots b_k c_k}  \\
	\sqrt{a},-\sqrt{a},\fr{aq}{b_1}, \fr{aq}{c_1},\fr{aq}{b_2},
	\fr{aq}{c_2}\,,\quad \,, \fr{aq}{b_k},\fr{aq}{c_k}\,,
	aq^{N+1} \endpmatrix
\tag4.6
$$
This in turn can be specialized to the natural extension of the
first Rogers-Ramanujan identity \c2, \c{4; Ch. 7}
$$
\align
	\sum_{n_1 \geqq n_2 \geqq \cdots n_{k-1}\geqq 0}\quad
	& \fr{q^{n_1^2 + n_2^2 + \cdots + n_{k-1}^2}}
	{(q;q)_{n_1-n_2}(q;q)_{n_2-n_3} \cdots (q;q)_{n_{k-2}-n_{k-1}}
	(q;q)_{n_{k-1}}}    \\
	& = \prod_{\Sb{n=1} \\ {n\not\equiv 0,\pm k\pmod{2k+1}}\endSb}^{\infty}
	\quad\fr1{1 - q^n}  \tag4.7
\endalign
$$

But other results also follow such as \c{11; p. 392}
$$
	\sum_{n=0}^{\infty} \fr{q^{\binom{n+1}{2}}}{(-q;q)_n} =
	\sum_{\Sb n \geqq 0 \\ |j| \leqq n \endSb} (-1)^{n+j} 
	q^{n(3n+1)/2 - j^2} (1 - q^{2n+1})
\tag4.8
$$
or \c{11; p. 404}
$$
	\sum_{n\geqq 1} \fr{(-1)^n q^{n^2}}{(q;q^2)_n} = 
	\sum_{n\geqq 1} \sum_{j=0}^{2n-1} (-1)^n q^{n(3n-1) - j
	(j+1)/2} (1 + q^{2n})
\tag4.9
$$

Further applications are given in \c5, \c6, \c{36} and \c{37}.

\subhead
5. \ Multidimensional Bailey Chains
\endsubhead

Lilly and Milne were the first researchers \c{25}, \c{26} and \c{27}
to recognize that the concepts described in Sections
2--5 could be extended to higher dimensional series.  In a substantial
series of papers extending classical ordinary and $q$-hypergeometric
functions to multiple series based on the unitary or symplectic
groups, S. Milne and his collaborators laid the ground work for an
extension of Bailey's Lemma and Bailey chains to these groups.

As an example of their achievement we state their generalization for
the symplectic groups $C_{\ell}$ \c{26; pp. 494--495}:

\proclaim
{Theorem}  Let $A = \{A_y\}$ and $B = \{B_y\}$ be sequences that
satisfy
$$
	B_N = \sum_{\Sb 0 \leq y_k \leq N_k \\ k = 1,2,\dots,\ell\endSb}
	\left\{ \prod_{r,s=1}^{\ell} \left[\left( q\;\fr{x_r}{x_s}\;
	q^{y_r - y_s}\right)_{N_r - y_r}^{-1} 
	(qx_r x_s q^{y_r + y_s})_{N_r - y_r}^{-1} \right] A_y \right\}
\tag5.1
$$
for every $N_i\geq 0$, $i = 1,2,\dots,\ell$.  ($A$ and $B$ form a $C_{\ell}$
Bailey pair).  If we define
$$
	A_N' : = \left\{ \prod_{k=1}^{\ell} \left[ \fr{(\alpha x_k)_{N_k}
	(qx_k \beta^{-1})_{N_k}}{(\beta x_k)_{N_k}(qx_k \alpha^{-1})_{N_k}}
	\right] \left(\fr{\beta}{\alpha}\right)^{(N_1 + \cdots + N_{\ell})}
	A_N \right\}
\tag5.2
$$
and if we also define
$$
\aligned
	B_N' : = \sum_{\Sb 0 \leq m_k \leq N_k \\ k = 1,2,\dots,\ell\endSb}
	& \left\{\left( \fr{\beta}{\alpha}\right)_{(N_1 + \cdots + N_{\ell})
	- (m_1 + \cdots + m_{\ell})}  \right.  \\
	& \times  \prod_{k=1}^{\ell} \left[ 
	\fr{(\alpha x_k)_{m_k}(qx_k\beta^{-1})_{m_k}}{(\beta x_k)_{N_k}
	(qx_k \alpha^{-1})_{N_k}} \right]
	\left(\fr{\beta}{\alpha}\right)^{(m_1 + \cdots + m_{\ell})}  \\
	& \times \prod_{1 \leq r < s \leq \ell}  [(qx_r x_s q^{m_r + m_s})
	^{-1} (qx_r x_3 q^{N_s - m_s})_{N_r - m_r}^{-1}] \\
	& \times \left. \prod_{r,s = 1}^{\ell} \left(q \fr{x_r}{x_s} 
	q^{m_r - m_s}\right)_{N_r - m_r}^{-1} B_m\right\} \,,
\endaligned
$$
then $A' = \{A_y'\}$ and $B' = \{B_y'\}$ also satisfy (4.13); i.e. 
they form a $C_{\ell}$ Bailey pair.
\endproclaim

In \c{27}, they consider the applications of their extensions and
provide relevant generalizations of the $q$-analog of Whipple's
transformation, the $q$-Dougall summation and other results.

In \c{12}, Schilling, Warnaar and I provide an alternative approach to
multiple series.  It is not nearly as broad in scope as the
Lilly-Milne work, but it does yield a variety of new series-product
identities.  The principal ideas of \c{12} culminate in the following
formulation.

\definition
{Definition 5.1}   ($A_2$ Bailey pair of type I).  Denote $\alpha_k =
\alpha_{k_1,k_2,k_3}$ and $\beta_L = \beta_{L_1,L_2}$, and let
$\alpha = \{\alpha_k\} \Sb k_1 \geq k_2 \geq k_3 \\ k_1 + k_2 + k_3 = 0
	\endSb$ and $\beta = \{\beta_l\}_{L_1,L_2 \geq 0}$  be a pair 
of sequences that satisfies
\enddefinition
$$
	\beta_L = \sum_{\Sb k_1 \geq k_2 \geq k_3 \\ k_1 + k_2 + k_3
	= 0 \endSb} \alpha_k \sum_r \fr{q^{r_1 r_{23}}}
	{(q)_{r_1}(q)_{r_2}(aq)_{r_3}(q)_{r_{12}}(q)_{r_{13}}(q)_{r_{23}}}\;,
\tag5.4
$$
where $\sum_r$ denotes a sum over $r_1,\dots,r_{23}$ such that
$$
	r_1 + r_{12} + r_{13} = L_2 - k_1, \quad r_2 + r_{12} +
	r_{23} = L_2 - k_2, \quad r_3 + r_{13} + r_{23} = L_2
	- k_3
$$
and
$$
	r_1 + r_2 + r_3 = 2L_1 - L_2, \quad r_{12} + r_{13} + r_{23}
	= 2L_2 - L_1.
$$
Then $(\alpha,\beta)$ forms an $A_2$ Bailey pair of type I relative
to $\alpha$.

\definition
{Definition 5.2} ($A_2$ Bailey pair of type II).  Denote $\alpha_k =
\alpha_{k_1,k_2,k_3}$ and $\beta_L = \beta_{L_1,L_2}$, and let
$\alpha = \{\alpha_k\} \Sb k_1 \geq k_2 \geq k_3 \\ k_1 + k_2 + k_3 = 0
\endSb$ and $\beta = \{\beta_L\}_{L_1,L_2 \geq 0}$ be a pair of
sequences that satisfies
$$
	\beta_L = \sum_{\Sb k_1 \geq k_2 \geq k_3 \\ k_1 + k_2 + k_3 = 0
	\endSb} \alpha_k \;\fr{(aq)_{L_1 + L_2}}{(aq)_{L_1 + k_1}
	(aq)_{L_1 + k_2}(q)_{L_1 + k_3}(q)_{L_2 - k_1}(q)_{L_2-k_2}
	(aq)_{L_2 - k_3}}\;.
\tag5.5
$$
Then $(\alpha,\beta)$ forms an $A_2$ Bailey pair of type II relative
to $\alpha$.
\enddefinition

With these definitions, the $A_2$ version of Bailey's lemma reads

\proclaim
{Theorem \c{12}} Le t$(\alpha,\beta)$ form an $A_2$ bailey pair of
type $T = I,II$ relative to $a$.  Then the pair $(\alpha',\beta')$
given by
$$
\aligned
	\alpha_k' & = a^{k_1 + k_2} q^{\fr12(k_1^2 + k_2^2 + K_3^2)}
	\alpha_k\,,   \\
	\beta_L' & = f_L^{(T)} \sum_{r_1 = 0}^{L_1} \sum_{r_2 = 0}^{L_2}
	\fr{a^{r_1} q^{r_1^2 - r_1 r_2 + r_2^2}}{(q)_{L_1 - r_1}
	(q)_{L_2 - r_2}}\;\beta_r
\endaligned
\tag5.6
$$
forms an $A_2$ Bailey pair of type II relative to $a$.  Here
$f_L^{(I)} = (aq)_{L_1 + L_2}^{-1}$ and $f_L^{(II)} = 1$.
\endproclaim

As examples of the series-product results that follow from this
theorem, we include \c{12; Th. 5.2}

\proclaim
{Theorem}  For $|q| < 1$,
$$
	\sum_{r_1 r_2 \geq 0} \fr{q^{r_1^2 - r_1 r_2 + r_2^2}}
	{(q)_{r_1}} \bmatrix 2r_1 \\ r_2 \endbmatrix =
	\prod_{n=1}^{\infty} \fr1{(1 - q^{7n-1})^2(1 - q^{7n-3})
	(1 - q^{7n-4})(1 - q^{7n-6})^2}\;,
\tag5.7
$$
$$
\aligned
	\sum_{r_1 r_2 \geq 0} \fr{q^{r_1^2 - r_1 r_2 + r_2^2+r_1+r_2}}
	{(q)_{r_1}} & \bmatrix 2r_1 \\ r_2 \endbmatrix \\
	& = \prod_{n=1}^{\infty} \fr1{(1 - q^{7n-2})(1 - q^{7n-3})^2
	(1 - q^{7n-4})^2(1 - q^{7n-5})}\;,
\endaligned
\tag5.8
$$
and
$$
\aligned
	& \sum_{r_1 r_2 \geq 0} \fr{q^{r_1^2 - r_1 r_2 + r_2^2+r_1}}
	{(q)_{r_1}} \bmatrix 2r_1 + 1 \\ r_2 \endbmatrix 
	= \sum_{r_1,r_2 \geq 0} \fr{q^{r_1^2-r_1 r_2 + r_2^2 + r_2}}
	{(q)_{r_1}}\;\bmatrix 2r_1 \\ r_2 \endbmatrix \\
	& = \prod_{n=1}^{\infty} \fr1{(1 - q^{7n-1})(1 - q^{7n-2})(1 - 
	q^{7n-3})(1 - q^{7n-4})(1 - q^{7n-5})(1 - q^{7n-6})}\;.
\endaligned
\tag5.9
$$
(the symbol $\bmatrix A \\ B\endbmatrix$ is defined in (3.10)).
\endproclaim

In \c8, I undertook an examination of the work of L. J. Rogers which
inspired all of Bailey's insights.  The main result there might be 
described as a ``direct product'' generalization of Bailey's Lemma
and Bailey Chains \c{7; Ch. 3}.  Its bilateral nature is dealt 
with in the $s = 1$ case in \c{16} and \c{29}.

\proclaim
{Theorem}  If for $n_1,n_2,\dots,n_2 \geqq 0$,
$$
	\beta_{n_1,n_2,\dots,n_r} = \sum_{r_1 = - \infty}^{n_1}
	\sum_{r_2 = - \infty}^{n_2} \cdots \sum_{r_x = - \infty}^{n_r}
	\fr{\alpha_{r_1,r_2,\dots,r_x}}{\prod_{j=1}^{s} (q;q)_{n_j-r_j}
	(a_j q;q)_{n_j + r_j}}
\tag5.10
$$
then
$$
	\beta_{n_1,n_2,\dots,n_r}' = \sum_{r_1 = - \infty}^{n_1}
	\sum_{r_2 = - \infty}^{n_2} \cdots \sum_{r_x = - \infty}^{n_r}
	\fr{\alpha_{r_1,r_2,\dots,r_x}'}{\prod_{j=1}^{s} (q;q)_{n_j-r_j}
	(a_j q;q)_{n_j + r_j}}\;,
\tag5.11
$$
where
$$
	\alpha_{r_1,r_2,\dots,r_x}' = \left( \prod_{j=1}^s
	\fr{(\rho_j)_{r_j} (\sigma_j)_{r_j}(a_j q/(\rho_j \sigma_j))^{r_j}}
	{(a_jq/\rho_j)_{r_j}(a_jq/\sigma_j)_{r_j}}\right) \;
	\alpha_{r_1,r_2,\dots,r_x}
\tag5.12
$$
and
$$
	\beta_{n_1,n_2,\dots,n_x}' = \prod_{j=1}^s 
		\sum_{m_j = -\infty}^{n_j}
	\fr{(\rho_j)_{m_j} (\sigma_j)_{m_j} (a_j q/(\rho_j\sigma_j))_{n_j - 
		m_j}
	\left(\fr{a_j q}{\rho_j\sigma_j}\right)^{m_j} \beta_{m_j,\dots,m_x}}
	{(q)_{n_j-m_j}(a_jq/\rho_j)_{n_j}(a_j q/\sigma_j)_{n_j}}\;.
\tag5.13
$$
\endproclaim

The instance of this theorem that is utilized in \c8 is the
\vskip .1in
{\bf First Corollary.}  If for $n_1,n_2,\dots,n_x \geqq 0$
$$
	\beta_{n_1,n_2,\dots,n_s} = \sum_{r_1 = - n_1}^{n_1}
	\sum_{r_2 = - n_2}^{n_2} \cdots \sum_{r_s = - n_s}^{n_s}
	\fr{\alpha_{r_1,r_2,\dots,r_s}}{\prod_{j=1}^s (q;q)_{n_j - r_j}
	(q;q)_{n_j + r_j}}\,,
\tag5.14
$$
then
$$
\aligned
	\sum_{n_1,\dots,n_s \geqq 0} & q^{n^2_1 + n_2^2 + \cdots 
	+ n_s^2}\beta_{n_1,n_2,\dots,n_s}  \\
	& = \fr1{(q;q)_{\infty}^s}  \sum_{m_1,m_2,\dots,m_s = -\infty}
	q^{m^2_1 + m_2^2 + \cdots + m_s^2} \alpha_{m_1,m_2,\dots,m_s}\,.
\endaligned
\tag5.15
$$

To illustrate the implications of this result for $q$-series 
identities, new pentagonal numbers theorems are proved:
$$
	\sum_{n=1}^{\infty} \fr{q^{2n^2}}{(q;q)_{2n}} =
	\fr{\sum_{n,m  =-\infty}^{\infty} (-1)^{n+m} q^{n(3n-1)/2 + 
		m(3m-1)/2+nm}}{\prod_{n-1}^{\infty} (1 - q^n)^2}\,.
\tag5.16
$$
and
$$
\aligned
	\sum_{i,j,k \geqq 0} & \fr{q^{i^2 + j^2 + k^2}}{(q;q)_{i+j-k}
	(q;q)_{i+k-j}(q;q)_{j+k-i}}   \\
	& = \fr{\sum_{n,m,p=-\infty}^{\infty} (-1)^{n+m+p}
	q^{n(3n-1)/2 + m(3m-1)/2 + p(3p - 1)/2+nm+np+mp}}{\prod_{n=1}^{\infty}
	(1 - q^n)^3}
\endaligned
\tag5.17
$$

To further illustrate the application of these results we shall 
prove:

\proclaim
{Theorem 2}
$$
\aligned
	\sum_{m,n \geqq 0} & \fr{(-1)^m q^{2m^2 + 2n^2}(-q;q^2)_{n-m}}
	{(q;q^2)_{n-m}(q^4;q^4)_m (q^4;q^4)_m}   \\
	& = \fr{\sum_{i,j=-\infty}^{\infty} (-1)^i q^{3i^2 + 3j^2 + 2ij}}
	{(q^2;q^2)_{\infty}^2}
\endaligned
\tag5.18
$$
\endproclaim

\demo
{Proof}  This result is a direct application of the First Corollary
wherein we replace $q$ by $q^2$, set $s = 2$, and
$$
	\alpha_{i,j} = (-1)^i q^{(i + j)^2}
\tag5.19
$$
$$
	\beta_{m,n} = \fr{(-1)^m (-q;q^2)_{n-m}}{(q;q^2)_{n-m}
	(q^4;q^4)_m (q^4;q^4)_n}\;.
\tag5.20
$$
This means that we must prove that
$$
\align
	\sum_{i,j=-\infty}^{\infty}  (-1)^i & q^{(i + j)^2}
	\bmatrix 2m \\ m + i \endbmatrix_{q^2} \bmatrix 2n \\
	n + j \endbmatrix_{q^2}  \\
	& = \fr{(-1)^m (-q;q^2)_{n-m} (q^2;q^4)_m(q^2;q^4)_n}
	{(q;q^2)_{n-m}}\;.   \tag5.21
\endalign
$$
To see this, we note
$$
\align
	& \sum_{i,j} (-1)^i q^{(i+j)^2} \bmatrix 2m \\ m+i \endbmatrix_{q^2}
	\bmatrix 2n \\ n+j \endbmatrix_{q^2}   \\
	& = \sum_j \bmatrix 2n \\ n + j\endbmatrix_{q^2} q^{j^2} \sum_i
	(-1)^k q^{i^2 + 2ij} \bmatrix 2m \\ m + i \endbmatrix_{q^2}  \\
	& = \sum_j \bmatrix 2n \\ n + j\endbmatrix_{q^2} q^{j^2} 
	(q^{2j+1},q^2)_m (q^{1-2j}; q^2)_m  \\
	& \hskip 2in \text{(by \c{4; p. 49, Ex. 1})}   \\
	& = \fr{(-1)^m (q;q^2)_{m-n} q^{(n+m)^2}}{(q;q^2)_{-m-n}}
	\quad  _2\phi_1 \; \pmatrix q^{-4n}, q^{2m-2n+1};q^2, - q^{2n-2m+1} \\
	q^{-2m - 2n+1} \endpmatrix  \\
	& = \fr{(-1)^m (-q;q^2)_{n-m}(q^2;q^4)_m(q^2;q^4)_r}{(q;q^2)_{n-m}}
	\\
	& \qquad\qquad \text{(by \c{1; eq. (1.7)}, \c{23; $\S$1.8})}
	\tag5.22
\endalign
$$
which is the desired result.  
\pf
\enddemo

\subhead
6. \ A New Bailey Chain, the $WP$-Bailey Chain
\endsubhead

The work described in Section 4 yields (among many other results)
transformations for the series given in (4.6).  The most famous of
these are the case $k = 1$, a summation due to F. H. Jackson \c{24},
and $k = 2$, Watson's $q$-analog of Whipple's theorem.

However, there is a classical transformation due to Bailey \c{23;
p. 39, eq. (2.9.1)}) which is not a direct corollary of the Bailey
Chains considered in Section 4.
$$
\aligned
	& _{10}\phi_9 \left[\gathered a,qa^{\fr12},-qa^{\fr12},b,c,d,e,f,
	\lambda aq^{n+1}/ef,q^{-n}  \\ a^{\fr12},-a^{\fr12},aq/b,
	aq/c,aq/d,aq/e,aq/f,efq^{-n}/\lambda,aq^{n+1}\endgathered
	\,;q,q \right]  \\
	& = \fr{(aq,aq/ef,\lambda q/e,\lambda q/f;q)_n}
	{(aq/e,aq/f,\lambda q/ef,\lambda q;q)_n} \;_{10}\phi_9\;
	\left[ \gathered \lambda,q\lambda^{\fr12}, q\lambda^{\fr12},
	\lambda b/a,\lambda c/a,\lambda d/a, \\ \lambda^{\fr12} -
	\lambda^{\fr12}, aq/b,aq/c,aq/d,\endgathered   \right.   \\
	& \quad \left. \gathered e,f,\lambda aq^{n+1}/ef,q^{-n}, \\
	\lambda q/e,\lambda q/f,efq^{-n}/a,\lambda q^{n+1}
	\endgathered\, ; q,q \right]\;,
\endaligned
\tag6.1
$$
where $\lambda = qa^2/bcd$.

If we examine the Bailey method for proving (6.1) (beautifully,
presented in \c{23; $\S$2.2 and $\S$2.9}), we find that we can easily
produce a new type of Bailey Chain.  We shall call it a $WP$-Bailey
Chain because both the series in (6.1) are examples of very
well-poised basic hypergeometric series \c{23; p. 32}.

\demo
{\bf Definition}  We say that two sequences $\alpha_n(a,k)$ and $\beta_n(a,k)$
form a $WP$-Bailey pair provided
$$
	\beta_n(a,k) = \sum_{i=0}^n \fr{(k/a;q)_{n-i}(k;q)_{n+i}}
	{(q;q)_{n-i}(aq;q)_{n+i}}\;\alpha_i (a,k)\,.
\tag6.2
$$

Note that if $k = 0$ then we have a Bailey pair as considered in
Section 4.
\enddemo

\proclaim
{Theorem 3}  If $\alpha_n(a,k)$ and $\beta_n(a,k)$ form a $WP$-Bailey
pair, then so do $\alpha_n'(a,k)$ and $\beta_n'(a,k)$, where
$$
	\alpha_n'(a,k) = \fr{(\rho_1;q)_n(\rho_2;q)_n(aq/\rho_1\rho_2)^n}
	{(aq/\rho_1;q)_n (aq/\rho_2;q)_n}\;\alpha_n \left( a,
	\fr{k\rho_1\rho_2}{aq}\right)  
\tag6.3
$$
and
$$
\align
	\beta_n'(a,k) & = \fr{\left(\fr{k\rho_1}{a};q\right)_n \left(
	\fr{k\rho_2}{a}q\right)_n}{\left(\fr{aq}{\rho_1},q\right)_n
	\left(\fr{aq}{\rho_2},q\right)_n}\quad \sum_{j=0}^n\quad
	\fr{(\rho_1;q)_j (\rho_2;q)_j}{\left(\fr{k\rho_2}{a};q\right)_j
	\left(\fr{k\rho_1}{a};q\right)_j}   \\
	& \times\; \fr{\left(1 - \fr{k\rho_1\rho_2 q^{2j-1}}{a}\right)
	\left(\fr{aq}{\rho_1\rho_2};q\right)_{n-j}(k;q)_{n+j}}
	{\left(1 - \fr{k\rho_1\rho_2}{aq}\right) (q;q)_{n-j} \left(
	\fr{k\rho_1\rho_2}{a} ;q\right)_{n+j}} \left(\fr{aq}{\rho_1\rho_2}
	\right)^j  \beta_j \left(a, \fr{k\rho_1\rho_2}{aq}\right)\,.
	\tag6.4
\endalign
$$
\endproclaim

\remark
{Remark}  Note that if we at $k = 0$, this result reduces to the
assertions in (4.1)--(4.4).
\endremark

\demo
{Proof}  We must show that $\alpha_n'(a,k)$ and $\beta_n'(a,k)$
as given in (6.3) and (6.4) actually satisfy (6.2).  By (6.4) 
and (6.2)
$$
\aligned
	\beta_n'(a,k) & = \fr{\left(\fr{k\rho_1}{a}\,,\, \fr{k\rho_2}{a};
	q\right)_n}{\left(\fr{aq}{\rho_1}\,,\,\fr{aq}{\rho_2};q\right)_n}
	\;\sum_{j=0}^n\; \fr{(\rho_1,\rho_2;q)_j}{\left( \fr{k\rho_2}{a}\,,\,
	\fr{k\rho_1}{a};q\right)_j}   \\
	& \times \fr{\left(1 - \fr{k\rho_1\rho_2 q^{2j-1}}{a}\right)
	\left(\fr{aq}{\rho_1\rho_2};q\right)_{n-j}(k;q)_{n+j}}{\left(
	1 - \fr{k\rho_1\rho_2}{aq}\right) (q;q)_{n-j}\left(\fr{k\rho_1\rho_2}
	{a};q\right)_{n+j}} \; \left(\fr{aq}{\rho_1\rho_2}\right)^j  \\
	& \times \sum_{i=0}^j \fr{\left( \fr{k\rho_1\rho_2}{a^2 q};
	q\right)_{j-i}\left( \fr{k\rho_1\rho_2}{aq};q\right)_{j+1}\;
	\alpha_i\left( a,\fr{k\rho_1\rho_2}{aq}\right)}{(q;q)_{j-i}
	(aq;q)_{j+i}}  \\
	& = \fr{\left( \fr{k\rho_1}{a}\,,\,\fr{k\rho_2}{a}; q\right)_n}
	{\left(\fr{aq}{\rho_1}\,,\,\fr{aq}{\rho_2};q\right)_n}
	\sum_{i=0}^n \sum_{j=0}^{n-i} \fr{(\rho_1,\rho_2;q)_{j+i}}
	{\left( \fr{k\rho_2}{a}\,,\, \fr{k\rho_1}{a};q\right)_{j+i}}  \\
	& \times \fr{\left(1 - \fr{k\rho_1\rho_2 q^{2i+2j-1}}{a}\right)
	\left(\fr{aq}{\rho_1\rho_2};q\right)_{n-i-j}(k;q)_{n+i+j}
	\left(\fr{aq}{\rho_1\rho_2}\right)^{i+j}}{\left(1 - 
	\fr{k\rho_1\rho_2}{aq}\right) (q;q)_{n-i-j} \left(\fr{k\rho_1\rho_2}
	{a};q\right)_{n+i+j}}   \\
	& \times \fr{\left(\fr{k\rho_1\rho_2}{a^2 q};q\right)_j
	\left(\fr{k\rho_1\rho_2}{aq};q\right)_{j+2i}\alpha_i
	\left(a,\fr{k\rho_1\rho_2}{aq}\right)}{(q;q)_j (aq;q)_{j+2i}}  \\
	& = \fr{\left(\fr{k\rho_1}{a}\,,\,\fr{k\rho_2}{a};q\right)_n}
	{\left(\fr{aq}{\rho_1}\,,\,\fr{aq}{\rho_2};q\right)_n}\sum_{i=0}^n
	\fr{(\rho_1,\rho_2;q)_i}{\left(\fr{k\rho_2}{a}\,,\,\fr{k\rho_1}{a};
	q\right)_i} \alpha_i\left(a,\fr{k\rho_1\rho_2}{aq}\right)  \\
	& \fr{\left(\fr{aq}{\rho_1\rho_2};q\right)_{n-i}(k;q)_{n+i}\left(
	\fr{aq}{\rho_1\rho_2}\right)^i \left(1 - \fr{k\rho_1\rho_2 q^{2i-1}}
	{a}\right)\left(\fr{k\rho_1\rho_2}{aq};q\right)_{2i}}{(q;q)_{n-i}
	\left(\fr{k\rho_1\rho_2}{a};q\right)_{n+i} \left(1 - 
	\fr{k\rho_1\rho_2}{aq}	\right)(aq;q)_{2i}}  \\
	&  _8\phi_7 \left(\gathered \fr{k\rho_1\rho_2 q^{2i-1}}{a}\,,\,
	q^{i+1} \sqrt{\fr{k\rho_1\rho_2}{aq}}\,, \,-q^{i+1}
	\sqrt{\fr{k\rho_1\rho_2}{aq}}\,,\,\rho_1 q^i,\rho_2 q^i\,,\,
	\fr{k\rho_1\rho_2}{a^2 q}\,,\,kq^{n+i}\,,\, q^{-n+i};q,q \\
	q^i \sqrt{\fr{k\rho_1\rho_2}{aq}}\,,\, -q^i 
	\sqrt{\fr{k\rho_1\rho_2}{aq}}\,,\,\fr{k \rho_2 q^i}{a}\,,\,
	\fr{k\rho_1 q^i}{a}\,,\, aq^{2i+1}\,,\, 
	\fr{\rho_1\rho_2 q^{-n+i}}{a}\,,\, \fr{k\rho_1\rho_2 q^{n+i}}{a}
	\endgathered\right) \\
	& = \fr{\left( \fr{k\rho_1}{a}\,,\,\fr{k\rho_2}{a};q\right)_n}
	{\left(\fr{aq}{\rho_1}\,,\,\fr{aq}{\rho_2};q\right)_n}
	\sum_{i=0}^n \fr{(\rho_1,\rho_2;q)_i}{\left(\fr{k\rho_2}{a}\,,\,
	\fr{k\rho_1}{a};q\right)_i} \;\alpha_i \left(a\,,\,\fr{k\rho_1\rho_2}
	{aq}\right)  \\
\endaligned
$$
$$
\align
	& \times \fr{\left(\fr{aq}{\rho_1\rho_2};q\right)_{n-i}(k;q)_{n+i}
	\left(\fr{aq}{\rho_1\rho_2}\right)^i \left(\fr{k\rho_1\rho_2}{a};
	q\right)_{2i}}{(q;q)_{n-i} \left(\fr{k\rho_1\rho_2}{a};q\right)_{n+i}
	(aq;q)_{2i}}   \\
	& \times \fr{\left(\fr{k\rho_1\rho_2 q^{2i}}{a}\,,\, \fr{k}{a}\,,\,
	\fr{\rho_1 q^{-n}}{a}\,,\, \fr{\rho_2 q^{-n}}{a}; q\right)_{n-i}}
	{\left(\fr{k\rho_2 q^i}{a}\,,\, \fr{k\rho_1 q^i}{a}\,,\,
	\fr{\rho_1\rho_2 q^{-n+i}}{a}\,,\,\fr{q^{-n-i}}{a};q\right)_{n-i}}  \\
	& \hskip 1in \text{(by \c{23; p. 35, eq. (2.6.2)})}   \\
	& = \sum_{i=0}^n \fr{\left(\fr{k}{a};q\right)_{n-i}(k;q)_{n+i}}
	{(q;q)_{n-i} (aq;q)_{n+i}}\;\fr{(\rho_1,\rho_2;q)_i}{\left(
	\fr{aq}{\rho_1}\,,\,\fr{aq}{\rho_2}; q\right)_i} \left(
	\fr{aq}{\rho_1\rho_2}\right)^i \alpha_i \left(a,\fr{k\rho_1\rho_2}
	{aq}\right)   \\
	& = \sum_{i=0}^n \fr{\left(\fr{k}{a};q\right)_{n-i} (k;q)_{n+i}}
	{(q;q)_{n-i}(aq;q)_{n+i}} \;\alpha_i' (a,k)\,,  \tag6.4
\endalign
$$
as desired.  \pf
\enddemo

It is now a simple matter to begin a $WP$-Bailey chain by noting that
$$
	\beta_n (a,k) = \delta_{n,0}
\tag6.5
$$
$$
	\alpha_n(a,k) = \fr{(a;q)_n (1-aq^{2n})\left(\fr{a}{k};q\right)_n}
	{(q;q)_n \,(1 - a) \,(kq;q)_n}\;\left(\fr{k}{a}\right)^n
\tag6.6
$$
forms a $WP$-Bailey pair.  This follows immediately by setting $m =n$
in the following identity
$$
\align
	& \sum_{n=0}^m \fr{\left(\fr{k}{a};q\right)_{n-i} (k;q)_{n+i} 
	\alpha_i(a,k)}{(q;q)_{n-i}(aq;q)_{n+i}}   \\
	& = \fr{\left(\fr{kq^{-m}}{a};q\right)_n (1-k) (kq^{m+1};q)_n
	(q^{1-n};q)_m q^{nm}}{(1 - kq^n)(aq^{m+1};q)_n (q;q)_n (q;q)_m}
	\tag6.7
\endalign
$$
Identity (6.7) follows immediately by mathematical induction on $m$.

\proclaim
{Corollary}  (F. H. Jackson's $q$-analog of Dougall's theorem
\c{24}, \c{23; p. 35, eq. (2.6.2)})
$$
\align
	_8\phi_7  & \left(\gathered a,q\sqrt{a},-q \sqrt{a},kq^n,\rho_1,
	\rho_2,\fr{a^2 q}{k\rho_1\rho_2}\,,\, q^{-n};q,q  \\
	\sqrt{a}, - \sqrt{a}, \fr{aq^{1-n}}{k}\,,\, \fr{aq}{\rho_1}\,,\,
	\fr{aq}{\rho_2}\,,\, \fr{k\rho_1\rho_2}{a}\,,\, aq^{n+1}
	\endgathered\right)  \\
	& = \fr{\left(aq, \fr{aq^{1-n}}{\rho_1 k}\,,\, \fr{aq^{1-n}}{\rho_2 k}
	\,,\, \fr{aq}{\rho_1\rho_2}; q\right)_n}{\left(\fr{aq^{1-n}}{k}\,,\,
	\fr{aq}{\rho_1}\,,\, \fr{aq}{\rho_2}\,,\, \fr{aq^{1-n}}{k\rho_1\rho_2}
	;q\right)_n}\;.  \tag6.8
\endalign
$$
\endproclaim

\demo
{Proof} This is just the assertion that the $WP$-Bailey pair
$\alpha_n' (a,k)$ and $\beta_n'(a,k)$ defined from Theorem 3 satisfy
(6.2) where $\alpha_n(a,k)$ and $\beta_n(a,k)$ are given by (6.5) and
(6.6).  \pf
\enddemo

\proclaim
{Corollary}  (W. N. Bailey's very well-poised $_{10}\phi_9$ identity
\c{23; p. 38, eq. (2.9.1)}).  Equation (6.1) is valid.
\endproclaim

\demo
{Proof} This is (after changes of variable), the assertion that the
$WP$-Bailey pair $\alpha_n''(a,k)$, $\beta_n''(a,k)$ (with new
parameters $\rho_3$ and $\rho_4$) defined from (6.3) and (6.4) satisfy
(6.2) where $\alpha_n'(a,k)$ and $\beta_n'(a,k)$ are given in the
previous corollary.  \pf
\enddemo

We close this section by remarking that the $N$-th entry in this
$WP$-Bailey chain produces a representation of 
$$
	_{2N+4}\phi_{2N+3} \pmatrix a,q\sqrt{a}, - a\sqrt{a},\rho_1,\rho_2,
	\rho_3 \rho_4,\dots,\rho_{2N-3},\rho_{2N-2},\lambda q^n, q^{-n};
	q,q  \\
	\sqrt{a}, - \sqrt{a}, \fr{aq}{\rho_1},\fr{aq}{\rho_2}\dots, 
	\fr{aq}{\rho_{2N-3}},\fr{aq}{\rho_{2N-2}},\fr{aq^{1-n}}{\lambda},
	aq^{n+1} \endpmatrix
\tag6.9
$$
(where $a^N q^{N-1} = \rho_1\rho_2 \cdots \rho_{2N-2} \lambda$) as an
$(N-2)$ fold summation.  Thus when $N=4$ (the first new case) we find
the very-well poised $_{12}\phi_{11}$ of the above type represented 
by a double sum.

\subhead
7. \ An Alternative WP-Bailey Pair
\endsubhead

In \c{13; $\S$9} and \c{14}, W. N. Bailey used ideas similar to those
recounted up to this point in order to develop identities for nearly
poised series.  These are series in which one column of parameters
does not satisfy the well-poised requirement.

When we translate Bailey's ideas into the world of Bailey chains, a
most extraordinary thing happens.  Namely, in contrast with Theorem 3,
we find that from any given $WP$-Bailey pair we can construct a new
$WP$-Bailey pair $\wa_n(a,k)$, $\wb_n(a,k)$ which is not at all the
pair given by (6.3) and (6.4).

\proclaim
{Theorem 4} If $\alpha_n(a,k)$ and $\beta_n(a,k)$ form a $WP$-Bailey
pair, then so do $\wa_n(a,k)$ and $\wb_n(a,k)$, where
$$
	\wa_n(a,k) = \fr{\left(\fr{qa^2}{k};q\right)_{2n} \left(
	\fr{k^2}{qa^2}\right)^n}{(k;q)_{2n}} \;\alpha_n \left(a,
	\fr{qa^2}{k}\right)
\tag7.1
$$
$$
	\wb_n(a,k) = \sum_{j=0}^n \fr{\left(\fr{k^2}{a^2 q};q\right)_{n-j}}
	{(q;q)_{n-z}} \;\left(\fr{k^2}{a^2 q}\right)^j \beta_j
	\left(a,\fr{aq^2}{k}\right)\,.
\tag7.2
$$
\endproclaim

\remark
{Remark}  There is absolutely no intersection of the $WP$-Bailey
pairs produced by Theorem 3 and those produced by Theorem 4.
\endremark

\demo
{Proof}  We are given that the sequences $\alpha_n(a,k)$ and 
$\beta_n(a,k)$ satisfy (6.2).  Hence
$$
\align
	\wb_n(a,k) & = \sum_{j=0}^n \fr{\left(\fr{k^2}{a^2 q};q\right)_{n-j}
	\left( \fr{k^2}{a^2 q}\right)^j}{(q;q)_{n-j}} \sum_{i=0}^j
	\fr{\left(\fr{qa}{k};q\right)_{j-1} \left(\fr{qa^2}{k};q\right)_{j+i}}
	{(q;q)_{j-i} (aq;q)_{j+i}} \alpha_i\left(a,\fr{qa^2}{k}\right)
	\\
	& = \sum_{i=0}^n \sum_{j=0}^{n-i} \fr{\left(\fr{k^2}{a^2 q};
		q\right)_{n-i-j} \left(\fr{k^2}{a^2 q}\right)^{j+i}
	\left(\fr{qa}{k};q\right)_j \left(\fr{aq^2}{k};q\right)_{j+2i}
	\alpha_i\left(a,\fr{qa^2}{k}\right)}{(q;q)_{n-j}(q;q)_j
	(aq;q)_{j+2i}}
	\\
	& = \sum_{i=0}^n \fr{\left(\fr{k^2}{a^2 q};q\right)_{n-i}
	\left(\fr{k^2}{a^2 q}\right)^i \left(\fr{aq^2}{k};q\right)_{2i}
	\alpha_i \left(a,\fr{aq^2}{k}\right)}{(q;q)_{n-i} (aq;q)_{2i}}
	\\
	& \quad \times \sum_{j=0}^{n-i} \fr{(q^{-n+i};q)_j}{\left(
	\fr{a^2 q^{2-n+i}}{k^2}; q\right)_j} \;q^j\; \fr{\left(
	\fr{aq}{k};q\right)_j\left(\fr{a^2 q^{2i+1}}{k};q\right)_j}
	{(q;q)_j(aq^{2i+1};q)_j}
	\\
	& = \sum_{i=0}^n \fr{\left(\fr{k^2}{a^2 q};q\right)_{n-i}
	\left(\fr{k^2}{a^2 q}\right)^i \left(\fr{qa^2}{k};q\right)_{2i}
	\alpha_i\left(a,\fr{qa^2}{k}\right)}{(q;q)_{n-i}(aq;q)_{2i}}
	\\
	& \hskip .5in \fr{\left(\fr{k}{a};q\right)_{n-i}(kq^{2i};q)_{n-i}}
	{(aq^{2i+1};q)_{n-i}\left(\fr{k^2}{a^2 q};q\right)_{n-i}}
	\\
	& \hskip 1in \text{(by \c{23; p. 13, eq. (1.7.2)})}
	\\
	& = \sum_{i=0}^n \fr{\left(\fr{k}{a};q\right)_{n-i}(k;q)_{n+i}}
	{(q;q)_{n-i}(aq;q)_{n+i}} \;;\fr{\left(\fr{qa^2}{k};q\right)_{2i}
	\left(\fr{k^2}{a^2 q}\right)^i \alpha_i \left(a;\fr{qa^2}{k}\right)}
	{(k;q)_{2i}}
	\\
	& = \sum_{i=0}^n \fr{\left(\fr{k}{a};q\right)_{n-i} (k;q)_{n+i}}
	{(q;q)_{n-i}(aq:q)_{n+i}} \wa_i (a,k),  \tag7.3
\endalign
$$
as desired.  \pf
\enddemo

The most immediate application of Theorem 4 is the application in
which the $WP$-Bailey pair $\alpha_n'(a,k)$ and $\beta_n'(a,k)$ of
the first Corollary are taken as input.  The resulting assertion that
$\wa_n(a,k)$ and $\wb_n(a,k)$ form a $WP$-Bailey pair is equivalent to
the result of Bailey \c{13; p. 431}:
$$
\aligned
	& _5\Phi_4 \bmatrix a,b,c,d, z^{-N};\;x,x  \\ ax/b,\;ax/c,\; ax/d,
	\;a^2 x^{-N}/k^2 \endbmatrix  \\
	& = \fr{(kx/a)_N(k^2 x/a)_N}{(kx)_N(k^2 x/a^2)_N}    \\
	\times _{12}\Phi_{11} & \left[\aligned  k,\,x &\sqrt k, - x\sqrt k,
	kb/a,\,kc/a,\,kd/a,\qquad \sqrt a, - \sqrt a,  \\
	& \sqrt k,\; - \sqrt k,\,ax/b,\;ax/c,\;ax/d,\;kx/\sqrt a,
	\;- kx/\sqrt a, \endaligned \right.  \\
	& \qquad\qquad \left. \aligned  & \sqrt (ax), \quad - \sqrt (ax),
	\;k^2 x^{N+1}/a,\quad x^{-N}; x;x  \\ k & \sqrt (x/a), \; - k \sqrt
	(x/a), \quad ax^{-N}/k,\;kx^{N+1}
	\endaligned \right]\;,
\endaligned
$$

\subhead
8. \ Summary and the Bailey Tree
\endsubhead
 
First it should be noted that this survey paper has shortchanged a
number of papers that have made substantial contributions to Bailey
chains.  Among these are \c{16}, \c{20}, \c{21}, \c{22}, \c{29}.

Finally, I note again the surprise contained in Sections 7 and 8.  The
new, two parameter $WP$-Bailey pairs (which reduce to the original
Bailey pairs when $k = 0$) generate new $WP$-Bailey pairs in two
entirely different ways (Theorems 3 and 4).  Thus at each step in the
generation of a chain there are two offspring pairs.  So instead of
generating a chain, we are generating a
\underbar{binary tree}.  We shall wait to explore Bailey Trees
subsequently.


\Refs

\ref
  \no 1
  \by G. E.Andrews
  \paper On the $q$-analog of Kummer's theorem and applications
  \jour Duke Math. J.
  \vol 40
  \yr 1973
  \pages 525--528
\endref


\ref
  \no 2
  \by G. E. Andrews
  \paper An analytic generalization of the Rogers-Ramanujan identities
	for odd moduli
  \jour Proc. Nat. Acad. Sci. USA
  \vol 71
  \yr 1974
  \pages 4082--4085
\endref


\ref
  \no 3
  \by G. E. Andrews
  \paper Problems and prospects for basic hypergeometric functions
  \paperinfo in Theory and Applications of Special Functions (R. 
	Askey, ed.), Academic Press, New York, 1975, pp. 191--224
\endref


\ref
  \no 4
  \by G. E. Andrews
  \paper The Theory of Partitions
  \paperinfo Encyl. of Math. and Its Appl., Vol. 2, Addison-Wesley,
	Reading, 1976 (Reissued:  Cambridge University Press, 
	Cambridge, 1985 and 1998)
\endref


\ref
  \no 5
  \by G. E. Andrews
  \paper Multiple series Rogers-Ramanujan type identities
  \jour Pacific J. Math.
  \vol 114
  \yr 1984
  \pages 267--283
\endref


\ref
  \no 6
  \by G. E. Andrews
  \paper The fifth and seventh order mock theta functions
  \jour Trans. Amer. Math. Soc.
  \vol 293
  \yr 1986
  \pages 113--134
\endref


\ref
  \no 7
  \by G. E. Andrews
  \paper $q$-Series:  Their Development
  \paperinfo CBMS Regional Conf. Lecture Series 66, Amer. Math. Soc.,
	Providence, 1986
\endref


\ref
  \no 8
  \by G. E. Andrews
  \paper Umbral calculus, Bailey chains and pentagonal number theorems
  \paperinfo J. Comb. Th., Ser. A, (to apear in 91 (2000))
\endref


\ref
  \no 9
  \by G. E. Andrews and A. Berkovich
  \paper A trinomial analogue of Bailey's lemma and $N = 2$ superconformal
	invariance
  \jour Comm. in Math. Phys.
  \vol 192
  \yr 1998
  \pages 245--260
\endref


\ref
  \no 10
  \by G. E. Andrews and D. Bowman
  \paper The Bailey transform and D. B. Sears
  \jour Quaest. Math.
  \vol 22
  \yr 1999
  \pages 19--26
\endref


\ref
  \no 11
  \by G. E. Andrews, F. J. Dyson and D. Hickerson
  \paper Partitions and indefinite quadratic forms
  \jour Invent. Math.
  \vol 91 
  \yr 1988
  \pages 391--407
\endref


\ref
  \no 12
  \by G. E. Andrews, A. Schilling and S. O. Warnaar
  \paper An $A_2$ Bailey Lemma and Rogers-Ramanujan-type identities
  \jour J. Amer. Math. Soc.
  \vol 12
  \yr 1999
  \pages 677--702
\endref


\ref
  \no 13
  \by W. N. Bailey
  \paper Some identities in combinatory analysis
  \jour Proc. London Math. Soc. (2)
  \vol 49
  \yr 1947
  \pages 421--435
\endref


\ref
  \no 14
  \by W. N. Bailey
  \paper A transformation of nearly-poised basic hypergeometric series
  \jour J. London Math. Soc.
  \vol 22
  \yr 1947
  \pages 237--240
\endref


\ref
  \no 15
  \by W. N. Bailey
  \paper Identities of the Rogers-Ramanujan type
  \jour Proc. London Math. Soc. (2)
  \vol 50 
  \yr 1949
  \pages 1--10
\endref


\ref
  \no 16
  \by A. Berkovich, B. M. McCoy and A. Schilling
  \paper $N = 2$ supersymmetry and Bailey pairs
  \jour Phys. A
  \vol 228
  \yr 1996
  \pages 33--62
\endref


\ref
  \no 17
  \by J. M. and P. B. Borwein
  \paper Pi and the AGM
  \paperinfo Wiley, New York, 1987
\endref


\ref
  \no 18
  \by D. M. Bressoud
  \paper Some identities for terminating $q$-series
  \jour Math. Proc. Camb. Phil. Soc.
  \vol 81
  \yr 1981
  \pages 211--223
\endref


\ref
  \no 19
  \by D. M. Bressoud
  \paper An easy proof of the Rogers-Ramanujan identities
  \jour J. Number Th.
  \vol 16
  \yr 1983
  \pages 235--241
\endref


\ref
  \no 20
  \by D. M. Bressoud
  \paper A matrix inverse
  \jour Proc. Amer. Math. Soc.
  \vol 88
  \yr 1983
  \pages 446--448
\endref


\ref
  \no 21
  \by D. M. Bressoud
  \paper The Bailey lattice:  an introduction, from Ramanujan Revisited
  \paperinfo (G. E. Andrews et al, eds.), Academic Press, New York
	1988, pp. 57--67
\endref


\ref
  \no 22
  \by O. Foda and V.-H. Iuano
  \paper Virasoro character identities from the Andrews-Bailey 
	construction
  \jour Int. J. Mod. Phys. A
  \vol 12
  \yr 1997
  \pages 1651--1676
\endref


\ref
  \no 23
  \by G. Gasper and M. Rahman
  \paper Basic Hypergeometric Series
  \jour Encyl. of Math and Its Appl.
  \paperinfo Vol. 35, Cambridge University Press, Cambridge, 1990
\endref


\ref
  \no 24
  \by F. H. Jackson
  \paper Summation of $q$-hypergeometric series
  \jour Messenger of Math.
  \vol 50 
  \yr 1921
  \pages 101--112
\endref


\ref
  \no 25
  \by G. M. Lilly and S. C. Milne
  \paper The $A_{\ell}$ and $C_{\ell}$ Bailey transform and lemma
  \jour Bull. Amer. Math. Soc. (NS)
  \vol 26
  \yr (1992)
  \pages 258--263
\endref


\ref
  \no 26
  \by G. M. Lilly and S. C. Milne
  \paper The $C_{\ell}$ Bailey transform and Bailey lemma
  \jour Constr. Approx.
  \vol 9
  \yr 1993
  \pages 473--500
\endref


\ref
  \no 27
  \by G. M. Lilly and S. C. Milne
  \paper Consequences of the $A_{\ell}$ and $C_{\ell}$ Bailey
	transform and Bailey lemma
  \jour Discr. Math.
  \vol 139
  \yr 1995
  \pages 319--346
\endref


\ref
  \no 28
  \by P. Paule
  \paper On identities of the Rogers-Ramanujan type
  \jour J. Math. Anal. and Appl.
  \vol 107
  \yr 1985
  \pages 225--284
\endref


\ref
  \no 29
  \by P. Paule
  \paper The concept of Bailey chains
  \jour Sem. Lothar. Combin. B
  \vol 18f
  \yr 1987
  \pages 24
\endref


\ref
  \no 30
  \by L. J. Rogers
  \paper Second memoir on the expansion of certain infinite products
  \jour Proc. London Math. Soc.
  \vol 25
  \yr 1894
  \pages 318--343
\endref


\ref
  \no 31
  \by L. J. Rogers
  \paper On two theorems of combinatory analysis and allied identities
  \jour Proc. London Math. Soc. (2)
  \vol 16
  \yr 1917
  \pages 315--336
\endref


\ref
  \no 32
  \by A. Schilling and S. O. Warnaar
  \paper A higher-level Bailey lemma
  \jour Int. J. Mod. Phys.
  \vol B11
  \yr 1997
  \pages 189--195
\endref


\ref
  \no 33
  \by A. Schilling and S. O. Warnaar
  \paper A higher level Bailey lemma, proof and application
  \jour Ramanujan journal
  \vol 2
  \yr 1998
  \pages 327--349
\endref


\ref
  \no 34
  \by A. Schilling and S. O. Warnaar
  \paper Conjugate Bailey pairs
  \paperinfo (to appear)
\endref


\ref
  \no 35
  \by D. B. Sears
  \paper On the transformation theory of basic hypergeometric
	functions
  \jour Proc. London Math. Soc. (2)
  \vol 53
  \yr 1951
  \pages 158--180
\endref


\ref
  \no 36
  \by L. J. Slater
  \paper A new proof of Rogers' transformations of infinite series
  \jour Proc. London Math. Soc. (2)
  \vol 53
  \yr 1951
  \pages 460--475
\endref


\ref
  \no 37
  \by L. J. Slater
  \paper Further identities of the Rogers-Ramanujan type
  \jour Proc. London Math. Soc. (2)
  \vol 54
  \yr (1952)
  \pages 147--167
\endref


\ref
  \no 38
  \by L. J. Slater
  \paper Generalized Hypergeometry Functions
  \paperinfo Cambridge University Press, Cambridge, 1966
\endref

\ref
  \no 39
  \by O. Warnaar
  \paper A note on the trinomial analogue of Bailey's lemma
  \jour J. combin. Th. A
  \vol 81
  \yr 1998
  \pages 114--118
\endref


\ref
  \no 40
  \by G. N. Watson
  \paper A new proof of the Rogers-Ramanujan identities
  \jour J. London Math. Soc.
  \vol 4
  \yr 1929
  \pages 4--9
\endref

\endRefs












\enddocument


