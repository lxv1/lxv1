%
\input amstex
\documentstyle{amsppt}
\NoBlackBoxes

\leftheadtext{George E. Andrews}
\def\c{\cite}
\def\fr{\frac}
\def\ovbeta{\overline{\beta}}
\def\pf{\hfill $\square$}

\topmatter
\title A Bailey Lemma from the Quintuple product  \endtitle

\author George E. Andrews$^{(1)}$ \endauthor

\dedicatory To the memory of Srinivasa Ramanujan \enddedicatory

\abstract 
In a previous paper, the discovery of further Rogers-Ramanujan
type identities from new Bailey Lemmas was discussed.  In that
paper, the starting point was a product of independent Jacobi
triple products.  In this paper, we start from the quintuple
product.
\endabstract
\endtopmatter

\document
\footnote""{$^{(1)}$Partially supported by the National Science Foundation 
under Grant DMS-9206993.}

\subhead
1. \ Introduction
\endsubhead

In his second paper, \c6 on Rogers-Ramanujan type identities, L.
J. Rogers began with an umbral transformation of Jacobi's triple
product.  In \c4, I used Rogers' idea on a product of several
Jacobi triple products.  As a result, multi-dimensional Bailey chains
(see \c3 for the origin of this term) followed naturally, and some
new Pentagonal Number Theorems were found.  For example, for $|q| < 1$.
$$
\align
	& \sum_{m,n,p\geqq 0}  \fr{q^{m^2 + n^2 + p^2}}{(q)_{m+n-p}
	(q)_{m+p-n}(q)_{p+n-m}} \tag1.1   \\
	& = \fr1{(q)_{\infty}^3} \sum_{i,j,k = -\infty}^{\infty}
	(-1)^{i+j+k} q^{\fr12 i(3i-1)+\fr12 j(3j - 1) + \fr12
	k(3k-1)+ij+ik+jk}\,,
\endalign
$$
where
$$
	(a)_n = (a;q)_n = (1 - a)(1 - aq) \cdots (1 - aq^{n-1})\,,
\tag1.2
$$
and
$$
	(a)_{\infty} = \lim_{n \rightarrow \infty} (a)_n\,.
\tag1.3
$$
In the above sum, the indices are restricted so that each of 
$m+n-p$, $m+p-n$ and $p+n-m$ is non-negative, or for simplicity
we may assume $1/(q)_n = 0$ when $n < 0$.

In this paper, our starting point will be the quintuple product
identity \c{5; p. 134, Ex. 5.6}
$$
	\sum_{n=-\infty}^{\infty} (-1)^n q^{n(3n-1)/2} z^{3n}
	(1 + zq^n) = (q)_{\infty} (-z)_{\infty} (- q/z)_{\infty}
	(qz^2;q^2)_{\infty} (qz^{-2};q^2)_{\infty}\,.
\tag1.4
$$
In Section 2, we shall obtain the Bailey lemmas that this identity
implies.  In Section 3, we note that the simplest application of our
Bailey lemmas is Euler's Pentagonal Number Theorem.  In Section 4, we
develop backaground necessary for our deeper applications in Section
5.  The difficulty arising in this latter application makes clear that
more extensive studies of classical very well poised basic
hypergeometric series will be necessary for general application of the
Quintuple Product Bailey Lemma.

The result we shall prove in Section 5 is
$$
\align
	\sum_{r,n,t,j \geqq 0} & \fr{(-1)^t \, q^{n^2+n + 
	j(3j+1)/2 +
	r(3r - 1)/2 + t(11t-1)/2 + 5rt + 5jt + 2jr + jn} (q)_{r+t+j}
	(1 - q^{6r+12t + 6j +6})}{(q)_r (q)_n (q)_t (q)_j}
	\tag1.5
	\\
	& = \fr{(q^7;q^7)_{\infty}(-q^2;q^7)_{\infty}(-q^5;q^7)_{\infty}
	(q^3;q^{14})_{\infty} (q^{11};q^{14})_{\infty}}{(q;q^2)_{\infty}}
	\;.
\endalign
$$
The right-hand side resembles but is, in fact, quite different
from \c{7; p. 160, eq. (80)}.

\subhead
2. \ New Bailey Lemmas
\endsubhead

Using Jacobi's Triple Product \c{2; p. 21, eq. (2.2.10)}
$$
	\sum_{n=-\infty}^{\infty} q^{n^2} y^n = (q^2;q^2)_{\infty}
	(-yq;q^2)_{\infty} (-y^{-1} q;q^2)_{\infty},
\tag2.1
$$
we rewrite (1.4) as
$$
\align
	\sum_{n=-\infty}^{\infty} & (-1)^n q^{n(3n-1)/2}z^{3n}
	(1 + zq^n) \tag2.2  \\
	& = \fr1{(-q;q)_{\infty}} (-z)_{\infty}
	(-q/z)_{\infty} \sum_{r=-\infty}^{\infty} (-1)^r
	z^{2r} q^{r^2}\,.
\endalign
$$

Next we expand the two products in the numerator of (2.2) using
Euler's identity \c{2; p. 19, eq. (2.2.6)}
$$
	\sum_{n=0}^{\infty}  \fr{q^{\binom{n}{2}}y^n}{(q)_n}
	= (-y)_{\infty}.
\tag2.3
$$

Hence
$$
\align
	& \sum_{n=-\infty}^{\infty} (-1)^n q^{n(3n-1)/2} z^{3n}
	(1 + zq^n)  \tag2.4   \\
	& = \fr1{(-q;q)_{\infty}} \sum_{i=0}^{\infty}
	\fr{q^{\binom{i}{2}}z^i}{(q)_i} \sum_{j=0}^{\infty}
	\fr{q^{\binom{j+1}{2}}z^{-j}}{(q)_j} \sum_{r=-\infty}^{\infty}
	(-1)^r z^{2r} q^{r^2}\,.
\endalign
$$

Following the original ideas of Rogers, we replace all odd powers 
of $z$ by $0$ and all even powers $z^{2n}$ by $\alpha_n$; furthermore
in the $i$ and $j$ sum we replace $i$ by $n + h$ and $j$ by $n - h$
(which is admissable because only even power of $z$ are not 
annihilated).

Hence we obtain (subject to obvious convergence conditions):
\proclaim
{Lemma 1}
$$
\align
	\sum_{n=-\infty}^{\infty} & q^{n(6n-1)} \alpha_{3n} -
	\sum_{n=-\infty}^{\infty} q^{(2n-1)(3n-1)} \alpha_{3n-1}
	\tag2.5  \\
	& = \fr1{(-q;q)_{\infty}}\; \sum_{n\geqq 0}\;
	\sum_{r=-\infty}^{\infty} \;(-1)^r q^{r^2 + n^2}
	\beta_n(r)\,,
\endalign
$$
where
$$
	\beta_n(r) = \sum_{h=-n}^n \;\fr{q^{h^2 - h} \alpha_{h+r}}
		{(q)_{n-h}\,(q)_{n+h}}\;.
\tag2.6
$$
\endproclaim

On the other hand, we may replace all even powers of $z$ in (2.4)
by $0$ and all odd powers $z^{2n-1}$ by $\alpha_n$; this time
$i = n+h-1$ and $j = n-h$.

We now obtain (upon replacing $n$ by $n + 1$):

\proclaim
{Lemma 2}
$$
\align
	\sum_{n=-\infty}^{\infty} & q^{n(6n+1)} \alpha_{3n+1} -
	\sum_{n=-\infty}^{\infty} q^{(2n-1)(3n-2)} \alpha_{3n-1}
	\tag2.7  \\
	& = \fr1{(-q;q)_{\infty}}\; \sum_{n\geqq 0}\;
	\sum_{r=-\infty}^{\infty} \; q^{r^2+n^2+n}(-1)^r \ovbeta_n(r)\,,
\endalign
$$
where
$$
	\ovbeta_n(r) = \sum_{h=-n-1}^{n} \;
	\fr{q^{h^2} \alpha_{h+r+1}}{(q)_{n-h}\,(q)_{n+h+1}}\;.
\tag2.8
$$
\endproclaim

\subhead{3. \ Euler's Pentagonal Number Theorem}\endsubhead

Let us examine Lemma 1, when $\alpha_n = (-1)^n$.  In this case,
the left-hand side of (2.5) becomes
$$
\align
	\sum_{n=-\infty}^{\infty} & q^{n(6n-1)} (-1)^n + 
		\sum_{n=-\infty}^{\infty} (-1)^n q^{6n^2 - 5n+1}
		\tag3.1   \\
	& = \sum_{n=-\infty}^{\infty} (-1)^n (-q)^{n(3n-1)/2}\,.
\endalign
$$
Putting $\alpha_{h+r} = (-1)^{h+r}$ in (2.6), we find by 
\c{5; p. 238, (II.21)}
$$
	\beta_n(r) = \fr{q^n (-1)^r}{(q^2;q^2)_n}
\tag3.2
$$
Hence under the substitution $\alpha_n = (-1)^n$, the right-hand
side of (2.5) becomes
$$
\align
	\fr1{(-q;q)_{\infty}} &\; \sum_{n\geqq 0}\; \sum_{r=-\infty}^{\infty}
	\; \fr{q^{r^2+n^2+n}}{(q^2;q^2)_n}  \tag3.3   \\
	& = \fr1{(-q;q)_{\infty}} \;(q^2;q^2)_{\infty} (-q;q^2)_{\infty}^2
	(-q^2;q^2)_{\infty}   \\
	& \hskip 1.5in \text{ (by (2.1) and (2.3))}   \\
	& = (-q;-q)_{\infty}\,.
\endalign
$$

Matching up (3.1) and (3.3) and changing $q$ into $-q$, we find
Euler's Pentagonal Number Theorem \c{2; p. 11, eq. (1.3.1)}
$$
	(q)_{\infty} = \sum_{n=-\infty}^{\infty} (-1)^n 
	q^{n(3n-1)/2}\,.
\tag3.4
$$

Of course, this provides no compelling reason to study Lemmas 1 or 2.
Indeed, Euler's Pentagonal Number Theorem is, in fact, a Corollary
of Jacobi's Triple Product itself.  Consequently, we must await 
Section 5 to see really new identities following from the Lemmas
in Section 2.

\subhead
4. \ Background from Very Well-Poised Series
\endsubhead

We require for Section 5, new representations of some of the series
appearing in \c{1; p. 433}.  In particular,
$$
	C_{1,r}(a;x;q) = \sum_{n\geqq 0} x^n a^{-n} q^{n^2 + n - rn}
	\fr{(1 - x^r q^{2nr})(x)_n(a)_n}{(1 - x)(q)_n \left(
	\fr{xq}{a}\right)_n}\;,
\tag4.1
$$
and
$$
	H_{1,r}(a;x;q) = \fr{\left(\fr{xq}{a}\right)_{\infty}}
	{(xq)_{\infty}}\;C_{1,r} (a;x;q)\,.
\tag4.2
$$


From \c{1; p. 439, eq. (3.7)}, we see that
$$
	H_{1,1} (a; x; q) = 1\,,
\tag4.3
$$
and from \c{1; p. 439, eq. (3.4)}, we deduce that 
$$
	H_{1,2} (a; xq; q) = 1 + \fr{xq}{a}\,.
\tag4.4
$$

From \c{1; p.l 435, eqs. (2.1) with $k = \lambda = 1$, $i = 
2-r$},
$$
	J_{1,r} (a;x;q) = H_{1,r-1}(a;x;q) - x\,H_{1,r-2}
	(a;x;q)\,,
\tag4.5
$$
where we have invoked \c{1; p. 435, eq. (2.3)}.  The $J_{1,r}
(a;x;q)$ is defined in \c{1; p. 433, eq. (1.4)}, but it will be
eliminated shortly.  So its exact definition is not required here.
Also from \c{1; p. 435, eq. (2.2)}
$$
	J_{1,r}(a;x;q) = H_{1,r} (a;xq;q) - \fr{xq}{a}
	H_{1,r-1} (a;xq;q)\,.
\tag4.6
$$

Eliminating $J_{1,r}(a;x;q)$ from (4.5) and (4.6), we find
$$
	H_{1,r}(a;xq;q) = \fr{xq}{a} H_{1,r-1}(a;xq;q) + 
	H_{1,r-1}(a;x;q) - x H_{1,r-2} (a;x;q)\,.
\tag4.7
$$

Equations (4.3), (4.4) and (4.7) reveal tha the $H_{1,r}(a;x;q)$
are polynomials in $x,a^{-1},q$ and $q^{-1}$.

Our final result in this section provides an explicit representation
for these polynomials.

\proclaim
{Lemma 3}  For each integer $r \geqq 1$,
$$
	H_{1,r} (a;xq^{r-1};q) = \sum_{r-1\geqq t,j\geqq 0}
	\bmatrix r - t - 1 \\ t,j,r - 2t - j - 1 \endbmatrix
	(-1)^t x^{t+j} a^{-j} q^{(t+j)^2 + \binom{t}{2}} 
\tag4.8
$$
where $\bmatrix A \\ B,C,A - B -C \endbmatrix = (q;q)_A\big/
\big((q;q)_B(q;q)_C(q;q)_{A-B-C}\big)$
\endproclaim

\demo
{Proof}  Let us designate the right-hand side of (4.8) as 
$h_r(xq^{r-1})$.  By inspection we see that 
$$
	h_1(x) = 1\,,
\tag4.9
$$
and
$$
	h_2(xq) = 1 + \fr{xq}{a}\,,
\tag4.10
$$

Next
$$
\align
 	& \fr{xq^{r-1}}{a} h_{r-1}(xq^{r-1}) + h_{r-1} (xq^{r-2}) -
	xq^{r-2}h_{r-2} (xq^{r-2})  \tag4.11   \\
	& = \sum_{t,j\geqq 0} \bmatrix r - t - 2 \\ t,j,r-2t-j-2
	\endbmatrix (-1)^t x^{t+j+1} a^{-j-1} q^{(t+j)^2 +
	\binom{t}{2} + t+j+r-1}  \\
	& + \sum_{t,j\geqq 0} \bmatrix r - t - 2 \\ t,j,r-2t-j-2
	\endbmatrix (-1)^t x^{t+j} a^{-j} q^{(t+j)^2+\binom{t}{2}}
	\\
	& - \sum_{t,j\geqq 0} \bmatrix r - t - 3 \\ t,j,r-2t-j-3
	\endbmatrix (-1)^t x^{t+j+1} a^{-j} q^{(t+j)^2 + \binom{t}{2}
	+t+j+r-2}
	\\
	& = \sum_{t,j\geqq 0} \bmatrix r - t - 2 \\ t,j-1,r-2t-j-1
	\endbmatrix (-1)^t x^{t+j} a^{-j} q^{(t+j-1)^2 + \binom{t}{2}
	+ t + j + r-2}
	\\
	& + \sum_{t,j\geqq 0} \bmatrix r - t - 2 \\ t,j,r-2t-j-2
	\endbmatrix (-1)^t x^{t+j} a^{-j} q^{(t+j)^2 + \binom{t}{2}}
	\\
	& + \sum_{t,j\geqq 0} \bmatrix r - t - 2 \\ t-1,j,r-2t-j-1
	\endbmatrix (-1)^t x^{t+j} a^{-j} q^{(t+j-1)^2+\binom{t-1}{2}
	+t + j + r-3}
	\\
	& = \sum_{t,j\geqq 0} \fr{(q;q)_{r-t-2}}{(q;q)_t(q;q)_j
	(q;q)_{r-2t-j-1}} (-1)^t x^{t+j} a^{-j} q^{(t+j)^2+\binom{t}{2}}
	\\
	& \qquad\quad \times \left\{ (1-q^j) q^{-2t-2j+1+t+j+r-2}
		+ (1 -q^{r-2t-j-1}) \right.
	\\
	& \qquad\qquad\quad + \left.(1 - q^t) q^{-2t-2j+1+j+r-2}\right\}
	\\
	& = \sum_{t,j\geqq 0} \fr{(q;q)_{r-t-1} (-1)^t x^{t+j} a^{-j}
	q^{(t+j)^2 + \binom{t}{2}}}{(q;q)_t(q;q)_j(q;q)_{r-2t-j-1}}
	\;.
	\\
	& = h_r (xq^{r-1})\,. 	
\endalign
$$

So by (4.9) and (4.10) $h_r(xq^{r-1})$ satisfies the same initial
conditions as $H_r(a;xq^{r-1};q)$  (see (4.3) and (4.4)).

In addition by (4.11) and (4.7), we see that $h_r(xq^{r-1})$ and
$H_{1,r}(a;xq^{r-1};q)$ satisfy the same second order recurrence.
Therefore
$$
	H_{1,r}(a;xq^{r-1};q) = h_r(xq^{r-1}),
\tag4.12
$$
and Lemma 3 is proved.  \pf
\enddemo

\proclaim
{Corollary}  For $r \geqq 1$.
$$
\align
	& C_{1,r}(a;x;q)  \\
	& = \fr{(xq)_{\infty}}{\left(\fr{xq}{a}\right)_{\infty}}
	\sum_{r-1\geqq t,j\geqq 0} \bmatrix r - t - 1 \\
	t,j,r-2t-j-1\endbmatrix (-1)^t x^{t+j} a^{-j} 
	q^{(t+j)^2+\binom{t}{2} + (1 - r)(t+j)}
\endalign
$$
\endproclaim

\demo
{Proof}  This follows imeediately from Lemma 3 and (4.2).  \pf
\enddemo

\subhead
5. \ Proof of Identity (1.5)
\endsubhead

We shall utilize Lemma 2 in the instance
$$
	\alpha_n = (-1)^n q^{\binom{n}{2}}\,.
\tag5.1
$$
Hence in this case,
$$
\align
	& \ovbeta_n(r) = \sum_{h=0}^n \fr{q^{h^2} (-1)^{h+r+1} 
		q^{\binom{h+r+1}{2}}}{(q)_{n-h}(q)_{n+h+1}}  
		\tag5.2     \\
	& \qquad\quad + \sum_{h=0}^n \fr{q^{(-h-1)^2} (-1)^{h+r}
	q^{\binom{-h+r}{2}}}{(q)_{n-h}(q)_{n+h+1}}   \\
	& = (-1)^{r+1}q^{\binom{n+1}{2}} \sum_{h=0}^n
		\fr{q^{h^2+\binom{h+1}{2}+ hr}(-1)^h}{(q)_{n-h}
		(q)_{n+h+1}}\;(1 - q^{2h+1-2hr-r})
	\\
	& = \fr{(-1)^{r+1}q^{\binom{r+1}{2}}}{(q)_n(q)_{n+1}}
		\sum_{h=0}^n \fr{(q^{-n})_h}{(q^{n+2})_h}\;
		q^{h^2 + h + h(r+n)}(1 - q^{(2h+1)(1-r)})
	\\
	& = \fr{(-1)^{r+1}q^{\binom{r+1}{2}}}{(q)_n(q^2;q)_n}\;
		C_{1,1-r} (q^{-n}; q;q)
	\\
	& = \fr{(-1)^{r}q^{\binom{r}{2}+1}}{(q)_n(q^2;q)_n}\;
		C_{1,r-1} (q^{-n}; q;q)\,.
\endalign
$$

Next we consdier the left-hand side of (2.7) under the 
substitution (5.1).  The result is
$$
\align
	& \sum_{n=-\infty}^{\infty} q^{n(6n+1)} (-1)^{3n+1}
	q^{\binom{3n+1}{2}} - \sum_{n=-\infty}^{\infty}
	q^{6n^2 - 7n+2} (-1)^{3n-1} q^{\binom{3n-1}{2}}
	\\
	& = - \sum_{n=-\infty}^{\infty} q^{\fr{21}{2} n^2 + \fr{5n}{2}}
	(-1)^n + \sum_{n=-\infty}^{\infty} q^{\fr{21n^2}{2} - \fr{23}{2}
	n+3}(-1)^n
	\\
	& = - \sum_{n=-\infty}^{\infty} (-1)^n q^{\fr{21}{2} n^2 + 
	\fr{5n}{2}} - \sum_{n=-\infty}^{\infty} (-1)^n q^{\fr{21}{2}
	n^2 + \fr{19n}{2} + 2}
	\\
	& = - (q^7;q^7)_{\infty} (-q^2;q^7)_{\infty} (-q^5;q^7)_{\infty}
	(q^3;q^{14})_{\infty} (q^{11};q^{14})_{\infty}
\endalign
$$
(by (1.4)).

As for the right-hand side of (2.7), we find that it is (noting
that $\ovbeta_n(1) = 0$ by (5.2))
$$
\aligned
	& \fr1{(-q;q)_{\infty}} \left\{ \sum_{n\geqq 0}\;
		\sum_{r=2}^{\infty} q^{r^2 + n^2 + n} (-1)^r
		\ovbeta_n(r) \right.    \\
	& \qquad + \left. \sum_{n\geq 0} \; \sum_{r=0}^{\infty} q^{r^2+n^2+n}
		(-1)^r \ovbeta_n(-r)\right\}  \\
	& = \fr1{(-q;q)_{\infty}} \;\sum_{n\geqq 0}\; q^{n^2 + n}
		\sum_{r=0}^{\infty} (-1)^r (q^{4r+r} \ovbeta_n(r+2)
		+ \ovbeta_n(-r))  \\
	& = \fr1{(-q;q)_{\infty}} \sum_{n\geqq 0} \fr{q^{n^2 + n}}
		{(q)_n(q^2;q)_n} \; \sum_{r=0}^{\infty} q^{r^2} (-1)^r
		((-1)^r q^{\binom{r+2}{2} + 4r+5} C_{1,r+1}(q^{-n};q;q)
	\\
	& \hskip 1in - (-1)^r q^{\binom{r}{2}} C_{1,r+1}(q^{-n};q;q)
	\\
	& = \fr{-1}{(-q;q)_{\infty}} \sum_{n\geqq 0}\;
		\fr{q^{n^2 +n}}{(q)_n(q^2;q)_n} \;\sum_{r=0}^{\infty}
		q^{r(3r-1)/2} C_{1,r+1}(q^{-h}; q;q)   \\
	& \hskip 1.5in \times (1 - q^{6r+6}) 
	\\
	& = \fr{-1}{(-q;q)_{\infty}} \sum_{n\geqq 0} 
		\fr{q^{n^2 +n}}{(q)_n(q^2;q)_n} \;\sum_{r=0}^{\infty}
		q^{r(3r-1)/2}(1 - q^{6r+6})
	\\
	& \hskip .5in (q^2;q)_n \sum_{0\leqq t,j\leqq r} \bmatrix
		r - t \\ t,j,r - 2t - j\endbmatrix (-1)^t q^{nj}
	\\
	& \hskip 1in \times q^{(t+j)^2+\binom{t}{2} + (t+j)(1-r)}
	\\
	& \hskip 1.5in \text{(by Lemma 3)} 
	\\
	& = \fr{-1}{(-q;q)_{\infty}} \sum_{r,n,t,j\geqq 0}
		(-1)^t \fr{q^{n^2+n+j(3j+1)/2 +r(3r-1)/2+t(11t-
			1)/2+5rt+5jt+2jr+jn} 
			(q)_{r+t+j}(1 - q^{4r+12t+6j+6})}
			{(q)_n(q)_t(q)_j(q)_r}
\endaligned
$$
which is identity (1.5), as desired.

\subhead
6. \ Conclusion
\endsubhead

The essential role of Lemma 3 in the proof of (1.5) indicates 
clearly the difficulties entailed in the applications of either 
Lemma 1 or Lemma 2.  Indeed, in order to apply these lemmas, 
it seems likely that one will at a minimum require analogs of
Lemma 3 for the $C_{k,i}((a);x;q)_{\lambda}$ of \c{1; p. 433, 
eq. (1.1)} with $i$ an arbitrary positive integer.

\Refs

\ref
  \no 1
  \by G. E. Andrews
  \paper $q$-Difference equations for certain well-poised basic
	hypergeometric series
  \jour Quart. J. Math., Oxford Ser.
  \vol 19
  \yr 1968
  \pages 433--447
\endref


\ref
  \no 2
  \by G. E. Andrews
  \paper The Theory of Partitions
  \paperinfo Encycl. of Math and Its Appl., Vol. 2, G.-C. Rota ed.,
	Addison-Wesley, REading Mass. 1976.  Reissued:  Cambridge
	University Press, Cambridge, 1988
\endref

\ref
  \no 3
  \by G. E. Andrews
  \paper Multiple series Rogers-Ramanujan identities
  \jour Pacific J. Math.
  \vol 114
  \yr 1984
  \pages 267--283
\endref

\ref
  \no 4
  \by G. E. Andrews
  \paper Umbral calculus, Bailey chains, and pentagonal number
	theorems
  \jour J. Comb. Th., Ser. A
  \vol 91
  \yr 2000
  \pages 464--475
\endref

\ref
  \no 5
  \by G. Gasper and M. Rahman
  \paper Basic Hypergeometric Series
  \paperinfo Encycl. of Math. and Its Appl., Vol. 35, G.-C. Rota ed.,
	Cambridge University Press, Cambridge, 1990
\endref
 
\ref
  \no 6
  \by L. J. Rogers
  \paper On two theorems of combinatory analysis and some allied
	identities
  \jour Proc. London Math. Soc. (2)
  \vol 16
  \yr 1917
  \pages 315--336
\endref

\ref
  \no 7
  \by L. J. Slater
  \paper Further identities of the Rogers-Ramanujan type
  \jour Proc. London Math. Soc. (2)
  \vol 54
  \yr 1952
  \pages 147--167
\endref
\endRefs

\vskip .5in

\noindent The Pennsylvania State University  \newline
University Park, Pennsylvania 16802  \newline
U.S.A. \newline
e-mail: \ andrews\@math.psu.edu
\enddocument
