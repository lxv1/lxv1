
\documentclass{amsart}
\usepackage{amssymb}
%\vfuzz2pt 

\newtheorem{thm}{Theorem}[section]
\newtheorem{cor}[thm]{Corollary}
\newtheorem{con}[thm]{Conjecture}
\newtheorem{lem}[thm]{Lemma}
\newtheorem{prop}[thm]{Proposition}
\newtheorem{defn}[thm]{Definition}
\newtheorem{qst}{Question}{}
\newtheorem{idt}{Identity}
%\renewcommand{\thethm}{}  % not numbered
\theoremstyle{remark}
\newtheorem{rem}{Remark}
\newtheorem{examp}{Example}

%\renewcommand{\therem}{}  % not numbered
%\numberwithin{equation}{section}
\DeclareMathOperator{\RE}{Re}
\DeclareMathOperator{\IM}{Im}
\DeclareMathOperator{\ess}{ess}
\newcommand{\eps}{\varepsilon}
\newcommand{\To}{\longrightarrow}
\newcommand{\h}{\mathcal{H}}
\newcommand{\s}{\mathcal{S}}
\newcommand{\A}{\mathcal{A}}
\newcommand{\J}{\mathcal{J}}
\newcommand{\M}{\mathcal{M}}
\newcommand{\W}{\mathcal{W}}
\newcommand{\X}{\mathcal{X}}
\newcommand{\BOP}{\mathbf{B}}
\newcommand{\BH}{\mathbf{B}(\mathcal{H})}
\newcommand{\KH}{\mathcal{K}(\mathcal{H})}
\newcommand{\Real}{\mathbb{R}}
\newcommand{\R}{\mathbb{R}}
\newcommand{\Complex}{\mathbb{C}}
\newcommand{\C}{\mathbb{C}}
\newcommand{\Field}{\mathbb{F}}
\newcommand{\RPlus}{\Real^{+}}
\newcommand{\Polar}{\mathcal{P}_{\s}}
\newcommand{\Poly}{\mathcal{P}(E)}
\newcommand{\EssD}{\mathcal{D}}
\newcommand{\Lom}{\mathcal{L}}
\newcommand{\States}{\mathcal{T}}
\newcommand{\N}{{\mathbb N}}
\newcommand{\Z}{{\mathbb Z}}
\newcommand{\abs}[1]{\left\vert#1\right\vert}
\newcommand{\set}[1]{\left\{#1\right\}}
\newcommand{\seq}[1]{\left<#1\right>}
\newcommand{\norm}[1]{\left\Vert#1\right\Vert}
\newcommand{\qp}[2]{(#1)_{#2}}
\newcommand{\essnorm}[1]{\norm{#1}_{\ess}}

%\newcommand{\proof}[1]{\noindent{\bf Proof#1:\  }}
\newcommand{\QED}{\hfill {\bf q.e.d.} \medskip}
%\renewcommand{\baselinestretch}{1.3}

\def \parag { \hspace{-0,65cm} }
\def \eqskip { \vspace*{2mm} }
\def \disps {\displaystyle}

\newcommand{\al}{{\alpha}}
\newcommand{\Al}{{\Alpha}}
\newcommand{\be}{{\beta}}
\newcommand{\Be}{{\Beta}}
\newcommand{\Sig}{{\Sigma}}
\newcommand{\Om}{{\Omega}}
\newcommand{\om}{{\omega}}
\newcommand{\de}{{\delta}}
\newcommand{\De}{{\Delta}}
\newcommand{\ga}{{\gamma}}
\newcommand{\Ga}{{\Gamma}}
\newcommand{\ka}{{\kappa}}
\newcommand{\la}{{\lambda}}
\newcommand{\si}{{\sigma}}
\newcommand{\La}{{\Lambda}}
\newcommand{\Emb}{{\it Emb}}
\newcommand{\io}{{\iota}}
\newcommand{\Si}{{\Sigma}}

%%%% Modifiquei fun�ies. \operatorname! (JPM)
\newcommand{\atan}{\operatorname{atan}}
\newcommand{\atanh}{\operatorname{atanh}}
\newcommand{\poly}{\operatorname{Li}}
\newcommand{\conjug}{\overline}
\newcommand{\di}{\operatorname{diag}}
\newcommand{\dist}{\operatorname{dist}}
\newcommand{\diam}{\operatorname{diam}}
\newcommand{\ds}{\displaystyle}
\newcommand{\dsum}{\ds\sum}
\newcommand{\dprod}{\ds\prod}
\newcommand{\ep}{$\Box$}
\newcommand{\formsk}{\vspace*{1mm}}
\newcommand{\im}{{\sqrt{-1}}}
\newcommand{\intom}{\int_{\Omega}}
\newcommand{\intdo}{\int_{\partial\Omega}}
\newcommand{\lla}{L(\lambda)}
\newcommand{\nz}{\normalsize}
\newcommand{\ov}{\overline}
\newcommand{\op}{{\overline{\partial}}}
\newcommand{\p}{{\partial}}
\newcommand{\pp}{\mathcal{P}_{p}}
\newcommand{\vdo}{|\partial\Omega|}
\newcommand{\vo}{|\Omega|}
\newcommand{\va}{\mathcal{B}}
\newcommand{\sa}{\mathcal{S}}

\begin{document}

\title{Extension of Abel's Lemma with $q-$series implications}

\author{George E. Andrews and Pedro Freitas}

\address{Department of Mathematics, Penn State University, 
University Park, Pennsylvania 16802, USA}
\email{andrews@math.psu.edu}
\address{Departamento de Matem\'{a}tica, Instituto Superior 
T\'{e}cnico,
Av.Rovisco Pais, 1049-001 Lisboa, Portugal}
\email{pfreitas@math.ist.utl.pt}
\thanks{The first author was partially supported by NSF grant 
DMS--$0200047$. The second author was partially supported by FCT, Portugal, 
through program POCTI}

\subjclass{}

\keywords{}

\date{\today}

\dedicatory{}

%%% ----------------------------------------------------------------------

\begin{abstract}
In previous work arising from the study of Ramanujan's Lost Notebook, 
a new Abel type lemma was proved. In this paper, we discuss 
extensions of this lemma and use it to prove many $q-$series identities.
\end{abstract}

%%% ----------------------------------------------------------------------
\maketitle
%%% ----------------------------------------------------------------------

\section{Introduction}

In the middle of page 14 of the published version of Ramanujan's Lost 
Notebook~\cite{rama}, there appears the following three assertions 
(rewritten in compact notation):

``If $S=\dprod_{n=1}^{\infty} (1+q^{n}),$ then
\[
\begin{array}{lll}
\dsum_{n=0}^{\infty}\frac{\ds q^{n(n+1)/2}}{\ds (-q;q)_{n}} & = &
1+q\dsum_{n=0}^{\infty}(-1)^{n}q^{n}(q;q)_{n}\eqskip\\
& = & 2\left\{ \frac{\ds 1}{\ds 2} 
S+\dsum_{n=0}^{\infty}\left[S-(-q;q)_{n}\right]\right\}
-2S\dsum_{n=1}^{\infty}\frac{\ds q^{n}}{\ds 1-q^{n}}\eqskip\\
&=& 2\left\{ \frac{\ds 1}{\ds 2} 
S+\dsum_{n=0}^{\infty}\left[S-\frac{\ds 1}{\ds (q;q^{2})_{n+1}}\right]\right\}-
2S\dsum_{n=1}^{\infty}\frac{\ds q^{2n}}{\ds 1-q^{2n}}.''
\end{array}
\]
Here we utilize the notation
\[
\left(A\right)_{n}= \left(A;q\right)_{n} = 
(1-A)(1-Aq)\ldots(1-Aq^{n-1}).
\]

The first published proof of these results occurs in~\cite{andr} and 
is nearly as odd as the original identities themselves. Subsequently, 
in~\cite{ajo}, an underlying lemma of Abel type was discovered which 
yields not only Ramanujan's results but a variety of simila bizarre 
identities with interesting applications to $L-$function evaluations.

The result in question is the following~(\cite{ajo}; p. 403):

\begin{prop} \label{propajo}
    Suppose that
    \[
    f(z) = \sum_{n=0}^{\infty} \alpha_{n}z^{n}
    \]
    is analytic for $|z|<1$. If $\alpha$ is a complex number for which
    \begin{itemize}
	\item[(i)] $ \dsum_{n=0}^{\infty}\left(\alpha-\alpha_{n}\right)
	<+\infty,\eqskip
	$
	\item[(ii)] ${\ds \lim_{n\to+\infty}} n(\alpha-\alpha_{n})=0,
	$
    \end{itemize}
    then
    \[
    \lim_{z\to 1^{-}}\frac{\ds d}{\ds dz}(1-z)f(z) = 
    \dsum_{n=0}^{\infty}\left(\alpha-\alpha_{n}\right).
    \]
\end{prop}

Following up on the utility of this result, Coogan, Lovejoy and 
Ono~\cite{coog,coon,loon} have found further applications and 
extensions.

In this paper, we shall consider a different extension of the above 
result in which the first derivative is replaced by the $p^{\rm th}$ 
derivative. Namely
\begin{prop} \label{pextension}
    Let
    \[
    f(z) = \sum_{n=0}^{\infty} \alpha_{n}z^{n}
    \]
    be analytic for $|z|<1$, and assume that for some positive integer 
    $p$ we have that
    \begin{itemize}
	\item[(i)] \label{condi}\[ \sum_{n=0}^{\infty}\left[\prod_{j=1}^{p}(n+j)
	\right]
	(\alpha_{n+p}-\alpha_{n+p-1})\] converges;
	\item[(ii)] \[ \sum_{n=0}^{\infty}\left[\prod_{j=1}^{p-1}(n+j)\right]
	(\alpha-\alpha_{n+p-1})\] converges, where $\alpha$ is a fixed 
	complex number such that
	\item[(iii)] \[ 
	\lim_{n\to\infty}\left[\prod_{j=1}^{p}(n+j)\right](\alpha_{n+p}-\alpha) = 0.
	\]
    \end{itemize}
    Then
    \[
    \frac{\disps 1}{\disps p}\lim_{z\to 1^{-}}\left\{\frac{\disps 
    d^{p}}{\disps d z^{p}}\left[(1-z)f(z)\right]\right\} = 
    \sum_{n=0}^{\infty}
    \left[\prod_{j=1}^{p-1}(n+j)\right](\alpha-\alpha_{n+p-1}).
    \]
\end{prop}

In Section~\ref{prel} we shall prove this propostion along with some related 
results that will prove useful in our application of 
Proposition~\ref{pextension}. 
In Section~\ref{heine} we look at applications of Proposition~\ref{pextension} 
in extending the results in~\cite{ajo}. Section~\ref{gener} looks more broadly 
at other applications and includes another extension of Theorem 1 
in~\cite{ajo}, and a different version of Theorem 2 in the same paper.
Section~\ref{last} concludes with comments and open 
questions.

\section{\label{prel}Preliminaries: some general results for series}

We begin by presenting the proof of Proposition~\ref{pextension}: 

{\it Proof of Proposition~\ref{pextension}.}
Multiplying the series development for $f$ by $1-z$ and 
differentiating $p$ times yields
\[
\begin{array}{ccc}
\frac{\ds d^{p}}{\ds dz^{p}}\left[ (1-z)f(z) \right]& = &
{\ds \sum_{n=p-1}^{\infty}}(n-p+2)\ldots(n+1) 
(\alpha_{n+1}-\alpha_{n})z^{n-p+1}\eqskip\\
 & = & 
 {\ds \sum_{n=0}^{\infty}\left[\prod_{j=1}^{p}(n+j)\right]}(\alpha_{n+p}-\alpha_{n+p-1})z^{n}.
\end{array}
\]
By Abel's limit theorem (and using~(i)), we have that this 
converges to
\[
\sum_{n=0}^{\infty}\left[\prod_{j=1}^{p}(n+j)\right](\alpha_{n+p}-\alpha_{n+p-1})
\]
as $z$ goes to $1^{-}$. This last sum may be written as
\begin{equation}
    \label{sepsum}
{\ds \sum_{n=0}^{N}\left[\prod_{j=1}^{p}(n+j)\right]}(\alpha_{n+p}-
\alpha_{n+p-1})+
{\ds \sum_{n=N+1}^{\infty}\left[\prod_{j=1}^{p}(n+j)\right]}(\alpha_{n+p}-
\alpha_{n+p-1}).
\end{equation}
Now the first of these sums is equal to
\[
\begin{array}{l}
1\times \ldots \times p\alpha_{p}+2
\times\ldots\times(p+1)\alpha_{p+1}+\ldots+(N+1)\times\ldots
\times(N+p)\alpha_{N+p}\eqskip\\
\indent -1\times\ldots \times p\alpha_{p-1}+2
\times\ldots\times(p+1)\alpha_{p}+\ldots+(N+1)\times\ldots
\times(N+p)\alpha_{N+p-1}\eqskip\\
=-p\left[ 1\times\ldots\times(p-1)\alpha_{p-1} + 
2\times\ldots\times p\alpha_{p} + \ldots + 
(N+1)\times\ldots\times(N+p-1)\alpha_{N+P-1}\right]\eqskip\\
\indent +(N+1)\ldots(N+p)\alpha_{n+p}\eqskip\\
=p{\ds \sum_{n=0}^{N}\left[\prod_{j=1}^{p-1}(n+j)\right]}(\alpha_{N+p}-
\alpha_{j+p-1})\eqskip\\
=p{\ds \sum_{n=0}^{N}\left[\prod_{j=1}^{p-1}(n+j)\right]}(\alpha-\alpha_{j+p-1})+
p{\ds \sum_{n=0}^{N}\left[\prod_{j=1}^{p-1}(n+j)\right]}(\alpha_{N+p}-\alpha)
\eqskip\\
=p{\ds \sum_{n=0}^{N}\left[\prod_{j=1}^{p-1}(n+j)\right]}(\alpha-\alpha_{j+p-1})+
(\alpha_{N+p}-\alpha){\ds \prod_{j=1}^{p}}(N+j).
\end{array}
\]
Substituting this back into~(\ref{sepsum}) and letting $N$ go to infinity
gives the desired reult. \qed

We remark that proceeding as above in the case where $p$ is zero gives the
following, which is one version of Abel's Lemma.

\begin{prop}
    Let $f$ be as above, and assume that
    \[
    \lim_{n\to\infty}\alpha_{n} = \alpha.
    \]
    Then
    \[
    \lim_{z\to 1^{-}} (1-z)f(z) = \alpha.
    \]
    As a consequence, we have that if
    \[
    \lim_{n\to\infty}\alpha_{n}=\alpha\neq0,
    \]
    then $f$ has a simple pole at $z=1$ with residue equal to $-\alpha$.
\end{prop}

The next result is a simple observation which is well known but, since
we will be making extensive use of it in the 
paper, we state it as a lemma for ease of reference.

\begin{lem} \label{exchangelem}
    Let $f$ and $g$ be two functions defined by the series
    \[
    \begin{array}{ccc}
	f(x) = {\ds\sum_{n=0}^{\infty}} f_{n}x^n & \mbox{and} &
	g(x) = {\ds\sum_{n=0}^{\infty}} g_{n}x^n,
    \end{array}
    \]
    and assume that the first of this series converges absolutely, and
    that the iterated series
    \[
    {\ds \sum_{n=0}^{\infty}\sum_{k=0}^{\infty}} \abs{g_{n}f_{k}q^{kn}x^k}
    \]
    converges.
    Then
    \[
    {\ds\sum_{n=0}^{\infty}} f_{n}g(q^{n}) x^{n} =
    {\ds\sum_{n=0}^{\infty}} g_{n}f(q^{n}x).
    \]
\end{lem}
\begin{proof} Just replace the function $g$ by the corresponding series 
and exchange the order of summation.
\end{proof}

\section{\label{heine}Applications to Heine series $_{2}\phi_{1}$}

As noted in the Introduction, this entire study has its genesis in 
formulas taken from Ramanujan's Lost Notebook~\cite{rama}. In this 
section, we shall examine the more general situation in which the 
full $_{2}\phi_{1}$ is considered with the full $p^{\rm th}$ 
derivative form of Proposition~\ref{pextension}.

We begin with some definitions and lemmas. We define, for integers 
$n\geq0$, $k>0$,
\[
S_{n}(k,z) = \dsum_{m=n}^{\infty}\frac{q^{km}}{\ds (1-zq^m)^{k}}.
\]

\begin{prop}
    \[
    S_{n}(k,z) =  z^{-k}\dsum_{j=1}^{\infty}\frac{\ds 
    \left(\begin{array}{c}j-1\\ 
    k-1\end{array}\right)(zq^n)^{j}}{\ds 1-q^{j}}.
    \]
\end{prop}
\begin{rem} {\rm We note that $z^{k}S_{n}(k,z)$ is thus a standard Lambert
    series
    \[
    \dsum_{j\geq1}A_{j}\frac{\ds q^{j}}{\ds (1-q^j)}, \mbox{ where }
    A_{j}= \left(\begin{array}{c}j-1\\ 
    k-1\end{array}\right)z^{j}q^{(n-1)j}.
    \]}
    \end{rem}
\begin{proof}
    \[
    \begin{array}{lll}
	S_{n}(k,z) & = & \dsum_{m=n}^{\infty}q^{km}\dsum_{j=0}^{\infty}
	\left(\begin{array}{c}j+k-1\\ 
    k-1\end{array}\right)z^{j}q^{mj}\eqskip\\
    & = & \dsum_{j=0}^{\infty}\left(\begin{array}{c}j+k-1\\ 
    k-1\end{array}\right)z^{j}\frac{\ds q^{n(k+j)}}{\ds 
    1-q^{k+j}}\eqskip\\
    & = & \dsum_{j=k}^{\infty}\left(\begin{array}{c}j-1\\ 
    k-1\end{array}\right)z^{j-k}\frac{\ds q^{nj}}{\ds 
    1-q^{j}}\eqskip\\
    & = & z^{-k}\dsum_{j=1}^{\infty}\frac{\ds \left(\begin{array}{c}j-1\\ 
    k-1\end{array}\right)z^{j}q^{nj}}{1-q^{j}}.
    \end{array}
    \]
\end{proof}

\begin{prop}
    If $g(z)$ is $p$ times differentiable at $z=1$, then
    \begin{equation}\label{gpderiv}
    \frac{\ds 1}{\ds p}\left[ \frac{\ds 
    d^{p}}{dz^{p}}(1-z)g(z)\right]_{z=1} = -g^{(p-1)}(1).
    \end{equation}
\end{prop}
\begin{proof}
    By Leibniz's Rule
    \[
    \begin{array}{lll}
	\frac{\ds 1}{\ds p}\left[ \frac{\ds 
    d^{p}}{\ds dz^{p}}(1-z)g(z)\right]_{z=1} & = &
    \frac{\ds 1}{\ds p}\left[ \dsum_{j=0}^{p}
     \left(\begin{array}{c}p\\ 
    j\end{array}\right)\frac{\ds 
    d^{j}}{dz^{j}}(1-z)\frac{\ds d^{p-j}}{\ds 
    dz^{p-j}}g(z)\right]_{z=1}\eqskip\\
    & = & 
    \frac{\ds 1}{\ds p}\left[ (1-z)g^{(p)}(z)-pg^{(p-1)}(z)\right]_{z=1}
    \eqskip\\
    &=& -g^{(p-1)}(1).
    \end{array}
    \]
\end{proof}

\begin{prop}\label{psrecurr}
    \begin{equation}
	q^{k}S_{n}(k,zq)=S_{n+1}(k,z)
    \end{equation}
    and
    \begin{equation}\label{dsn}
    \frac{\ds d}{\ds dz} S_{n}(k,z\alpha ) = k\alpha S_{n}(k+1,z 
    \alpha)
    \end{equation}
\end{prop}
\begin{proof}
    \[
    \begin{array}{lll}
    q^{k}S_{n}(k,z q) & = & \dsum_{m=n}^{\infty}\frac{\ds 
    q^{(m+1)k}}{\ds (1-zq^{m+1})^{k}}\eqskip\\
    & = & \dsum_{m=n+1}^{\infty}\frac{\ds 
    q^{mk}}{\ds (1-zq^{m})^{k}}\eqskip\\
    & = & S_{n+1}(k,z).\eqskip\\
    \frac{\ds d}{\ds dz}S_{n}(k,z \alpha) & = &
    \frac{\ds d}{\ds dz}\dsum_{m=n}^{\infty}\frac{\ds 
    q^{mk}}{\ds (1-z\alpha q^{m})^{k}}\eqskip\\
    & = & \dsum_{m=n}^{\infty}q^{mk}(-k)(1-z\alpha q^{m})^{-k-1}
    (-\alpha q^{m})\eqskip\\
    & = & k\alpha\dsum_{m=n}^{\infty}\frac{\ds 
    q^{m(k+1)}}{\ds (1-z\alpha q^{m})^{k+1}}\eqskip\\
    & = & k\alpha S_{n}(k+1,z\alpha).
    \end{array}
    \]
\end{proof}

\begin{prop}\label{pol2var} For each positive integer $p$ there is a polynomial in $2p$ 
variables
\[
\pp(x_{1},\ldots,x_{p},y_{1},\ldots,y_{p})
\]
such that for each integer $n\geq0$
\begin{equation}
    \frac{\ds d^{p}}{\ds dz^{p}}\frac{\ds (Axq^{n})_{\infty}}{\ds (B 
    z q^{n})_{\infty}} = \frac{\ds (Axq^{n})_{\infty}}{\ds (B 
    z q^{n})_{\infty}}\pp(S_{n}(1,zB),\ldots,S_{n}(p,zB),S_{n}(1,zA),
    \ldots,S_{n}(p,zA)).
\end{equation}
In addition, the coefficients are polynomials in $A$ and $B$:
\begin{equation}\label{p0}
    \mathcal{P}_{0}() = 1,
\end{equation}
\begin{equation}\label{p1}
    \mathcal{P}_{1}(x_{1},y_{1}) = Bx_{1}-Ay_{1},
\end{equation}
and
\begin{equation}\label{p2}
    \mathcal{P}_{2}(x_{1},x_{2},y_{1},y_{2}) = (Bx_{1}-Ay_{1})^{2}+
    B^{2}x_{2}^{2}-A^{2}y_{2}.
\end{equation}
\end{prop}
\begin{proof} First we treat $p=1$.
    \[
    \begin{array}{lll}
    \frac{\ds d}{\ds dz}\frac{\ds (Azq^{n})_{\infty}}{\ds 
    (Bzq^{n})_{\infty}} & = & \frac{\ds (Azq^{n})_{\infty}}{\ds 
    (Bzq^{n})_{\infty}}\left\{\dsum_{m=n}^{\infty}\frac{\ds 
    Bq^{m}}{1-Bzq^{m}}-\dsum_{m=n}^{\infty}\frac{\ds 
    Aq^{m}}{1-Azq^{m}}\right\}\eqskip\\
    & = & \frac{\ds (Azq^{n})_{\infty}}{\ds (Bzq^{n})_{\infty}}
    \left\{ BS_{n}(1,zB)-AS_{n}(1,zA)\right\},
    \end{array}
    \]
which is the $p=1$ case with $\mathcal{P}_{1}(x_{1},y_{1}) = 
Bx_{1}-Ay_{1},$ as asserted in~(\ref{p1}).

For $p=2$, and taking into account the $p=1$ case and 
Proposition~\ref{psrecurr},
    \[
    \begin{array}{lll}
    \frac{\ds d^{2}}{\ds dz^{2}}\frac{\ds (Azq^{n})_{\infty}}{\ds 
    (Bzq^{n})_{\infty}} & = & \frac{\ds d}{\ds dz}
    \frac{\ds (Azq^{n})_{\infty}}{\ds (Bzq^{n})_{\infty}}
    \left\{ BS_{n}(1,zB)-AS_{n}(1,zA)\right\}\eqskip\\
    & = & \frac{\ds (Azq^{n})_{\infty}}{\ds (Bzq^{n})_{\infty}}
    \left\{ BS_{n}(1,zB)-AS_{n}(1,zA)\right\}^{2}\eqskip\\
    & & \hspace*{1cm}+\frac{\ds (Azq^{n})_{\infty}}{\ds 
    (Bzq^{n})_{\infty}}\left\{ B^{2}S_{n}(2,zB)-A^{2}S_{n}(2,zA)\right\}
    \eqskip\\
    & = & \frac{\ds (Azq^{n})_{\infty}}{\ds 
    (Bzq^{n})_{\infty}}\mathcal{P}_{2}(S_{n}(1,zB),S_{n}(2,zB),
    S_{n}(1,zA),S_{n}(2,zA)),
    \end{array}
    \]
    where $\mathcal{P}_{2}(x_{1},x_{2},y_{1},y_{2}) = (Bx_{1}-Ay_{1})^{2}+
    B^{2}x_{2}^{2}-A^{2}y_{2}.$
    
    The case for general $p$ now follows easily by mathematical 
    induction, by noting that
    \[
    \frac{\ds d^{p+1}}{\ds z^{p+1}}\frac{\ds (Azq^{n})_{\infty}}{\ds 
    (Bzq^{n})_{\infty}}=\frac{\ds d}{\ds dz} \left[ \frac{\ds (Azq^{n})_{\infty}}{\ds 
    (Bzq^{n})_{\infty}}\pp(S_{n}(1,zB),\ldots,S_{n}(p,zB),S_{n}(1,zA),
    \ldots,S_{n}(p,zA))\right].
    \]
    This last expression then yields $\mathcal{P}_{p+1}$ by repeated use 
    of~(\ref{dsn}).
\end{proof}

\begin{thm} \label{npthm}For each integer $p\geq1$,
    \begin{equation}\label{npid}
    \begin{array}{l}
    \dsum_{n=0}^{\infty}(n+1,p-1)\left[\frac{\ds \qp{a}{\infty}\qp{b}{\infty}}{\ds 
    \qp{c}{\infty}\qp{q}{\infty}} - \frac{\ds \qp{a}{n+p-1}\qp{b}{n+p-1}}{\ds 
    \qp{c}{n+p-1}\qp{q}{n+p-1}}\right] \eqskip\\
    \hspace*{1cm} = 
    \frac{\ds 1}{\ds p}\frac{\ds \qp{a}{\infty}\qp{b}{\infty}}{\ds 
    \qp{c}{\infty}\qp{q}{\infty}}\pp(S_{0}(1,q),\ldots,S_{0}(p,q),
    S_{0}(1,a),\ldots,S_{0}(p,a))\eqskip\\
    - \frac{\ds \qp{a}{\infty}\qp{b}{\infty}}{\ds 
    \qp{c}{\infty}\qp{q}{\infty}}\dsum_{n=1}^{\infty}\frac{\ds 
    \qp{c/b}{n}b^{n}}{\ds 
    (1-q^{n})\qp{a}{n}}\mathcal{P}_{n-1}(S_{n}(1,q), \ldots , 
    S_{n}(p,q),S_{n}(1,a),\ldots,S_{n}(p,a))
    \end{array}
    \end{equation}
\end{thm}
\begin{proof}
    By Proposition~\ref{pextension} with
    \[
    \alpha_{n}= \frac{\ds \qp{a}{n}\qp{b}{n}}{\ds \qp{q}{n}\qp{c}{n}},
    \]
    we see that
    \[
    \begin{array}{l}
	\dsum_{n=0}^{\infty}(n+1,p-1)\left(\frac{\ds \qp{a}{\infty}\qp{b}{\infty}}{\ds 
    \qp{c}{\infty}\qp{q}{\infty}} - \frac{\ds \qp{a}{n+p-1}\qp{b}{n+p-1}}{\ds 
    \qp{c}{n+p-1}\qp{q}{n+p-1}}\right) \eqskip\\
    \hspace*{1cm} = \frac{\ds 1}{\ds p}\frac{\ds d^{p}}{\ds dz^{p}}\left[ 
    (1-z)\dsum_{n=0}^{\infty}\frac{\ds \qp{a}{n}\qp{b}{n}}{\ds \qp{q}{n}\qp{c}{n}}
    z^{n}\right]_{z=1}\eqskip\\
    \hspace*{1cm} = \frac{\ds 1}{\ds p}\frac{\ds d^{p}}{\ds dz^{p}}\left[ 
    \frac{\ds \qp{b}{\infty}\qp{az}{\infty}}{\ds 
    \qp{c}{\infty}\qp{qz}{\infty}}+\frac{\ds \qp{b}{\infty}\qp{az}{\infty}}{\ds 
    \qp{c}{\infty}\qp{qz}{\infty}}\dsum_{n=1}^{\infty}\frac{\ds 
    \qp{c/b}{n}\qp{z}{n}b^{n}}{\ds 
    \qp{q}{n}\qp{az}{n}}\right]_{z=1}\eqskip\\
    \hspace*{1cm}\mbox{(by Heine's transformation~\cite{fine}, page 
    26, eq. (20.41))}\eqskip\\
    \hspace*{1cm} = \frac{\ds 1}{\ds p}\frac{\ds d^{p}}{\ds dz^{p}}\left[ 
    \frac{\ds \qp{b}{\infty}\qp{az}{\infty}}{\ds 
    \qp{c}{\infty}\qp{qz}{\infty}}+\frac{\ds \qp{b}{\infty}}{\ds \qp{c}{\infty}}
    (1-z)\dsum_{n=1}^{\infty}\frac{\ds 
    \qp{c/b}{n}b^{n}\qp{azq^{n}}{\infty}}{\ds 
    \qp{q}{n}\qp{zq^{n}}{\infty}}\right]_{z=1}.
    \end{array}
    \]
    The right--hand side of the assertion~(\ref{npid}) now follows 
    immediately from Propositions~\ref{psrecurr} and~\ref{pol2var}.
\end{proof}

The most elegant instance of Theorem~\ref{npthm} are the case $b=c$ 
and the case $a=q$ which we exhibit in the following corollaries.

\begin{cor} For each integer $p\geq 1$
    \begin{equation}\label{cor1}
	\begin{array}{l}
	\dsum_{n=0}^{\infty}(n+1,p-1)\left(\frac{\ds \qp{a}{\infty}}{\ds 
    \qp{q}{\infty}} - \frac{\ds \qp{a}{n+p-1}}{\ds 
    \qp{q}{n+p-1}}\right) \eqskip\\
    \hspace*{1cm} = \frac{\ds 1}{\ds p}\frac{\ds \qp{a}{\infty}}{\ds 
    \qp{q}{\infty}}\pp(S_{0}(1,q),\ldots,S_{0}(p,q),
    S_{0}(1,a),\ldots,S_{0}(p,a))
    \end{array}
    \end{equation}
\end{cor}
\begin{proof} Set $b=c$ in Theorem~\ref{npthm} and note the second 
term on the right--hand side vanishes.
\end{proof}

\begin{cor} For each integer $p\geq 1$
    \begin{equation}\label{cor2}
	\begin{array}{l}
	\dsum_{n=0}^{\infty}(n+1,p-1)\left(\frac{\ds \qp{b}{\infty}}{\ds 
    \qp{c}{\infty}} - \frac{\ds \qp{b}{n+p-1}}{\ds 
    \qp{c}{n+p-1}}\right) \eqskip\\
    \hspace*{1cm} = -(p-1)!\frac{\ds \qp{b}{\infty}}{\ds 
    \qp{c}{\infty}}\dsum_{n=1}^{\infty}\frac{\ds 
    \qp{c/b}{n}(bq^{p-1})^{n}}{\ds (q)_{n}(1-q^n)^p}.
    \end{array}
    \end{equation}
\end{cor}
\begin{proof}
    Set $a=q$ in the second line of the proof of Theorem~\ref{npthm}, 
    and we see that
    \[
    \begin{array}{l}
    \dsum_{n=0}^{\infty}(n+1,p-1)\left(\frac{\ds \qp{b}{\infty}}{\ds 
    \qp{c}{\infty}} - \frac{\ds \qp{b}{n+p-1}}{\ds 
    \qp{c}{n+p-1}}\right) \eqskip\\
    \hspace*{1cm} = \frac{\ds 1}{\ds p}\frac{\ds d^{p}}{\ds dz^{p}}
    \left[\frac{\ds \qp{b}{\infty}}{\ds 
    \qp{c}{\infty}}(1-z)\dsum_{n=1}^{\infty}\frac{\ds 
    \qp{c/b}{n}b^{n}}{\ds (q)_{n}(1-zq^n)}\right]_{z=1}\eqskip\\
    \hspace*{1cm} = -\frac{\ds \qp{b}{\infty}}{\ds \qp{c}{\infty}}
    \dsum_{n=1}^{\infty}\frac{\ds 
    \qp{c/b}{n}b^{n}}{\ds (q)_{n}}\left[\frac{\ds d^{p-1}}{\ds dz^{p-1}}
    (1-zq^{n})^{-1}\right]_{z=1}\eqskip\\
    \hspace*{1cm} = -(p-1)!\frac{\ds \qp{b}{\infty}}{\ds 
    \qp{c}{\infty}}\dsum_{n=1}^{\infty}\frac{\ds 
    \qp{c/b}{n}(bq^{p-1})^{n}}{\ds (q)_{n}(1-q^n)^p},
    \end{array}
    \]
    as desired.
\end{proof}

If in~(\ref{cor2}) we divide by $\qp{b}{\infty}$ and
let $b$ go to zero, the right--hand side of the 
resulting expression is reminiscent of
the type of series found in~\cite{acs}. We shall now use
Proposition~\ref{pextension} to prove an identity for a $q-$series which was
considered in that paper and which is
related to random graphs.
\begin{thm}\label{liouville}
    \[
    \sum_{n=1}^{\infty} (-1)^{n+1} \frac{\ds q^{n(n+1)/2}}
    {\ds \qp{q}{n} (1-q^{n})^{p}} = \frac{\ds \qp{q}{\infty}}{p!}
    \lim_{x\to1^{-}} \frac{\ds d^{p}}{dx^{p}}\left[ \frac{\ds x^{p-1}}
    {\ds \qp{qx}{\infty}}\right], \;\; (p\geq 1)
    \]
\end{thm}
\begin{proof}
    In Lemma~\ref{exchangelem}, let
    \[
    f(x) = 1-\qp{qx}{\infty} = \dsum_{n=1}^{\infty}(-1)^{n+1}\frac{\ds
    q^{n(n+1)/2}}{\qp{q}{n}}x^{n}
    \]
    and
    \[
    \begin{array}{lll}
    g(x) & = & \frac{\ds 1}{\ds (1-x)^{p}}\eqskip\\
    & = & \frac{\ds 1}{\ds (p-1)!}\frac{\ds d^{p-1}}{\ds 
    dx^{p-1}}\left(\frac{\ds 1}{\ds 1-x}\right)\eqskip\\
    & = & \frac{\ds 1}{\ds (p-1)!}\dsum_{n=0}^{\infty}(n+1,p-1)x^{n}
    \end{array}
    \]
    to obtain
    \begin{equation}\label{rand1}
    \dsum_{n=1}^{\infty}(-1)^{n+1}\frac{\ds
    q^{n(n+1)/2}}{\qp{q}{n}}\frac{\ds 1}{\ds (1-q^{n})^{p}} =
    \frac{\ds 1}{\ds (p-1)!}\dsum_{n=0}^{\infty}(n+1,p-1)\left[ 
    1-\frac{\ds \qp{q}{\infty}}{\ds \qp{q}{n}}\right].
    \end{equation}
    By Proposition~\ref{pextension} with
    \[
    \alpha_{n} = \frac{\qp{q}{\infty}}{\ds \qp{q}{n-p+1}}
    \]
    it follows that the right--hand side of identity~(\ref{rand1}) is 
    equal to
    \[
    \frac{\ds 1}{\ds p!}\lim_{x\to 1^{-}}\left[ \frac{\ds d^{p}}{\ds 
    dx^{p}}\left[(1-x)h_{p}(x)\right]\right],
    \]
    where
    \[
    h_{p}(x) = \qp{q}{\infty}\dsum_{n=0}^{\infty}\frac{\ds 
    x^{n}}{\ds \qp{q}{n-p+1}} = \frac{\ds \qp{q}{\infty}}{\ds 
    \qp{x}{\infty}}x^{p-1}.
    \]
\end{proof}
    This, together with Lemma 2.5 from~\cite{acs}, gives Theorem~2.1 
    in that paper.


\section{\label{gener}Some variations on Theorems 1 and 2 
in~\cite{ajo}}

By putting
\[
\begin{array}{lll}
f(x) = \frac{\ds 1}{\ds 1-x} -\frac{\ds \qp{q}{\infty}}{\ds 
\qp{x}{\infty}} & \mbox{and} & g(x)\equiv 1
\end{array}
\]
in Lemma~\ref{exchangelem}, and then letting $x$ go to $1$ we obtain the
well--known identity (see, for instance,~\cite{fine})
\begin{equation}
    \label{fineid}
    \sum_{n=0}^{\infty} \left[ 1- \frac{\ds \qp{q}{\infty}}{\ds 
    \qp{q}{n}}\right] = \sum_{n=1}^{\infty} \frac{\ds q^{n}}{\ds 
    1-q^{n}}.
\end{equation}
We shall now consider some variations on this, beginning with a 
generalization of Theorem 1 in~\cite{ajo}.

\begin{thm} \label{gthm1ajo} In the notation of 
Lemma~\ref{exchangelem}:
    \[
    \sum_{n=0}^{\infty} g_{n}\left[ \frac{\ds \qp{t}{n}}{\ds 
    \qp{a}{n}} - \frac{\ds \qp{t}{\infty}}{\ds 
    \qp{a}{\infty}}\right] = \frac{\ds \qp{t}{\infty}}{\ds 
    \qp{a}{\infty}}\sum_{n=1}^{\infty}\frac{\ds \qp{a/t}{n}}{\ds 
    \qp{q}{n}} g(q^{n}) t^{n}.
    \]
\end{thm}
\begin{proof}
    Take
    \[
    \begin{array}{lll}
    f(x) & = & \frac{\ds \qp{bqx}{\infty}}{\ds 
    \qp{qx}{\infty}}-1\eqskip\\
    & = & \dsum_{n=1}^{\infty}\left[\frac{\ds \qp{bq}{n}}{\qp{q}{n}}-
    \frac{\ds \qp{bq}{n-1}}{\ds \qp{q}{n-1}}\right] x^{n}\eqskip\\
    & = & \dsum_{n=1}^{\infty}\frac{\ds \qp{b}{n}}{\qp{q}{n}} 
    q^{n}x^{n}
    \end{array}
    \]
    in Lemma~\ref{exchangelem}. Then
    \[
    \dsum_{n=1}^{\infty}\left[\frac{\ds \qp{bq}{n}}{\qp{q}{n}}-
    \frac{\ds \qp{bq}{n-1}}{\ds \qp{q}{n-1}}\right] x^{n}= \dsum_{n=0}^{\infty}g_{n}
    \left[\frac{\ds \qp{bqx}{\infty}\qp{qx}{n}}{\ds 
    \qp{bqx}{n}\qp{qx}{\infty}}-1\right]
    \]
    and
    \[
    \frac{\ds \qp{bqx}{\infty}}{\ds 
    \qp{qx}{\infty}}\dsum_{n=0}^{\infty}g_{n}\left[ \frac{\ds 
    \qp{qx}{n}}{\ds \qp{bqx}{n}}-\frac{\ds 
    \qp{qx}{\infty}}{\ds \qp{bqx}{\infty}}\right] = 
    \dsum_{n=1}^{\infty}\frac{\ds \qp{b}{n}}{\qp{q}{n}} 
    q^{n}g(q^{n})x^{n}.
    \]
    Writing $t=qx$ and $a=bqx$ gives
    \[
    \sum_{n=0}^{\infty} g_{n}\left[ \frac{\ds \qp{t}{n}}{\ds 
    \qp{a}{n}} - \frac{\ds \qp{t}{\infty}}{\ds 
    \qp{a}{\infty}}\right]=\frac{\ds \qp{t}{\infty}}{\ds 
    \qp{a}{\infty}} \dsum_{n=1}^{\infty}\frac{\ds \qp{a/t}{n}}{\qp{q}{n}} 
    g(q^{n})t^{n},
    \]
    as desired.
\end{proof}
    
>From this we obtain several identities, the first of which is a 
slightly more {\it compact} version of the corresponding result in~\cite{ajo}.
\begin{cor}\label{cort1}
    \[
    \sum_{n=0}^{\infty} \left[ \frac{\ds \qp{t}{n}}{\ds 
    \qp{a}{n}}-\frac{\ds \qp{t}{\infty}}{\ds 
    \qp{a}{\infty}}\right] =
    \frac{\ds \qp{t}{\infty}}{\ds 
    \qp{a}{\infty}} \sum_{n=1}^{\infty} \frac{\ds \qp{a/t}{n}}{\ds 
    \qp{q}{n}}\frac{\ds t^{n}}{\ds 1-q^{n}}.
    \]
\end{cor}

We now give some examples of identities that can be obtained by specializing
$g, a$ and $t$ in Theorem~\ref{gthm1ajo}.
\begin{cor}
    \begin{enumerate}
	\item[(i)]
    $
        \dsum_{n=0}^{\infty} \left[ \frac{\ds \qp{t}{n}}{\ds 
    \qp{t}{\infty}}-1\right] = \dsum_{n=1}^{\infty} \frac{\ds t^{n}}{\ds 
    \qp{q}{n}(1-q^{n})}\eqskip
    $
    
    \item[(ii)]
	    $
    \dsum_{n=0}^{\infty} \left[ \frac{\ds \qp{t}{n}}{\ds 
    \qp{q}{n}}-\frac{\ds \qp{t}{\infty}}{\ds 
    \qp{q}{\infty}}\right]^{2} =
    \left[\frac{\ds \qp{t}{\infty}}{\ds 
    \qp{q}{\infty}}\right]^{2} \dsum_{n=1}^{\infty} \frac{\ds \qp{q/t}{n}}{\ds 
    \qp{q}{n}}\left[ \frac{\ds \qp{q}{n}}{\ds \qp{t}{n}}-1\right]
    \frac{\ds t^{n}}{\ds 1-q^{n}}\eqskip
    $
    
    \item[(iii)] $     \dsum_{n=0}^{\infty} \left[ 1- \frac{\ds \qp{q}{\infty}}{\ds 
    \qp{q}{n}}\right]^{2} = \dsum_{n=1}^{\infty} (-1)^{n+1}\frac{\ds 
    q^{n(n+1)/2}}{\ds 
    \qp{q}{n}}\frac{\ds 1-\qp{q}{n}}{\ds 1-q^{n}}$
    
    \item[(iv)] $     \dsum_{n=0}^{\infty} \frac{\ds \qp{q}{\infty}}{\ds \qp{q}{n}}
    \left[ 1- \frac{\ds \qp{q}{\infty}}{\ds 
    \qp{q}{n}}\right] =
    \dsum_{n=1}^{\infty} (-1)^{n+1}\frac{\ds 
    q^{n(n+1)/2}}{\ds 
    1-q^{n}}\eqskip $
    
    \item[(v)] $\dsum_{n=0}^{\infty}\left[\frac{\ds \qp{q}{n}}{\ds 
    \qp{q}{\infty}}-\frac{\ds \qp{q}{\infty}}{\ds 
    \qp{q}{n}}\right] = \dsum_{n=1}^{\infty}\frac{\ds 1+\qp{q}{n}}{\ds 
    1-q^{n}}\frac{\ds q^{n}}{\ds \qp{q}{n}} \eqskip$
    
    \item[(vi)] $  \dsum_{n=0}^{\infty}\left\{ \left[\frac{\ds \qp{t}{n}}
    {\ds \qp{q}{n}}\right]^{2}-\left[\frac{\ds \qp{t}{\infty}}
    {\ds \qp{q}{\infty}}\right]^{2}\right\} = 
    \left[\frac{\ds \qp{t}{\infty}}
    {\ds \qp{q}{\infty}}\right]^{2}\dsum_{n=1}^{\infty}\frac{\ds 
    \qp{q/t}{n}}{\ds \qp{q}{n}}\left[\frac{\ds \qp{q}{n}}{\ds 
    \qp{t}{n}}+1\right]\frac{\ds t^{n}}{\ds 1-q^{n}}\eqskip
    $
    
    \item[(vii)] $\dsum_{n=0}^{\infty}\left[ \frac{\ds 1}
    {\ds \qp{q}{n}^{2}}-\frac{\ds 1}
    {\ds \qp{q}{\infty}^{2}}\right] = \frac{\ds 1}{\ds \qp{q}{\infty}^{2}}
    \dsum_{n=1}^{\infty}(-1)^{n} \frac{\ds q^{n(n+1)/2}}{\ds \qp{q}{n}}
    \frac{\ds 1+\qp{q}{n}}{\ds 1-q^{n}}\eqskip$
    
    \item[(viii)] $ \frac{\ds \qp{a}{\infty}}{\ds 
    \qp{t}{\infty}\qp{q}{\infty}}\dsum_{n=0}^{\infty}(-1)^{n} \frac{\ds
    q^{n(n+1)/2}}{\ds \qp{q}{n}}\left[\frac{\ds \qp{t}{n}}{\ds 
    \qp{a}{n}}-\frac{\ds \qp{t}{\infty}}{\ds 
    \qp{a}{\infty}}\right] = 
    \dsum_{n=1}^{\infty}\frac{\qp{a/t}{n}}{\ds \qp{q}{n}^{2}} 
    t^{n}\eqskip$
    
    \item[(ix)] $\frac{\ds 1}{\ds \qp{q}{\infty}^{2}}
    \dsum_{n=0}^{\infty}(-1)^{n} q^{n(n+1)/2}\left[ 1 - \frac{\ds 
    \qp{q}{\infty}}{\ds \qp{q}{n}}\right] = \dsum_{n=1}^{\infty}\frac{\ds q^{n}}
    {\qp{q}{n}^{2}}\eqskip$
    
    \end{enumerate}
\end{cor}
\begin{proof}
    \begin{enumerate}
	\item[(i)] Let $a=0$ in Corollary~\ref{cort1}.
	\item[(ii)] Let $a=q$ and
	\[
	g_{n} = \frac{\ds \qp{t}{n}}{\ds \qp{q}{n}}-
	\frac{\qp{t}{\infty}}{\ds \qp{q}{\infty}}
	\]
	in Theorem~\ref{gthm1ajo}. Then
	\[
	g(x) = \frac{\ds \qp{tx}{\infty}}{\ds \qp{x}{\infty}}-
	\frac{\ds \qp{t}{\infty}}{\ds \qp{q}{\infty}}\frac{\ds 1}{\ds 1-x}
	\]
	and
	\[
	\begin{array}{lll}
	    \dsum_{n=0}^{\infty} \left[ \frac{\ds \qp{t}{n}}{\ds 
    \qp{q}{n}}-\frac{\ds \qp{t}{\infty}}{\ds 
    \qp{q}{\infty}}\right]^{2} & = &
    \frac{\ds \qp{t}{\infty}}{\ds 
    \qp{a}{\infty}}\dsum_{n=1}^{\infty}\frac{\ds \qp{a/t}{n}}{\ds 
    \qp{q}{n}} \left[
    \frac{\ds \qp{tq^{n}}{\infty}}{\ds \qp{q^{n}}{\infty}}-
	\frac{\ds \qp{t}{\infty}}{\ds \qp{q}{\infty}}\frac{\ds 1}{\ds 1-q^{n}}
	\right]t^{n}\eqskip\\
	& = & \left[\frac{\ds \qp{t}{\infty}}{\ds 
    \qp{q}{\infty}}\right]^{2} \dsum_{n=1}^{\infty} \frac{\ds \qp{q/t}{n}}{\ds 
    \qp{q}{n}}\left[ \frac{\ds \qp{q}{n}}{\ds \qp{t}{n}}-1\right]
    \frac{\ds t^{n}}{\ds 1-q^{n}}
    \end{array}
    \]
    as desired.
    \item[(iii)] Let $t\to0$ in (ii).
    \item[(iv)] Let $a=q$ and $g_{n}=\qp{q}{\infty}/\qp{q}{n}$ in Theorem~\ref{gthm1ajo}.
    Then $g(x) = \qp{q}{\infty}/\qp{x}{\infty}$ which yields, on 
    letting $t$ go to zero on both sides,
    \[
    \begin{array}{lll}
	\dsum_{n=0}^{\infty} \frac{\ds \qp{q}{\infty}}{\ds \qp{q}{n}}
    \left[ \frac{\ds \qp{q}{\infty}}{\ds 
    \qp{q}{n}}-1\right] & = & \dsum_{n=1}^{\infty}\frac{\ds 1}{\ds 
    \qp{q}{n}}\frac{\ds \qp{q}{\infty}}{\ds 
    \qp{q^{n}}{\infty}}\lim_{t\to 
    0}\left[t^{n}\qp{q/t}{n}\right]\eqskip\\
    &=&\dsum_{n=1}^{\infty}(-1)^{n}\frac{\ds q^{n(n+1)/2}}{\ds 1-q^{n}} .
    \end{array}
    \]
    \item[(v)] \[
    \dsum_{n=0}^{\infty}\left[\frac{\ds \qp{q}{n}}{\ds 
    \qp{q}{\infty}}-\frac{\ds \qp{q}{\infty}}{\ds 
    \qp{q}{n}}\right] = \dsum_{n=0}^{\infty}\left[\frac{\ds \qp{q}{n}}{\ds 
    \qp{q}{\infty}}-1\right]+\dsum_{n=0}^{\infty}\left[ 1 -
    \frac{\ds \qp{q}{\infty}}{\ds 
    \qp{q}{n}}\right]\eqskip
    \]
    and the result follows from~(\ref{fineid}) and (i) with $t=q$.
    \item[(vi)] Let
    \[
    g_{n}=\frac{\ds \qp{t}{n}}{\ds \qp{q}{n}}+
    \frac{\ds \qp{t}{\infty}}{\ds \qp{q}{\infty}}
    \]
    in Theorem~\ref{gthm1ajo}. Then,
    \[
    g(x) = \frac{\ds \qp{tx}{\infty}}{\ds \qp{x}{\infty}}+
    \frac{\ds \qp{t}{\infty}}{\ds 
    \qp{q}{\infty}}\frac{\ds 1}{\ds 1-x}
    \]
    and we have that (with $a=q$)
    \[
    \dsum_{n=0}^{\infty}\left\{ \left[\frac{\ds \qp{t}{n}}
    {\ds \qp{q}{n}}\right]^{2}-\left[\frac{\ds \qp{t}{\infty}}
    {\ds \qp{q}{\infty}}\right]^{2}\right\} = \frac{\ds \qp{t}{\infty}}
    {\ds \qp{q}{\infty}} \dsum_{n=1}^{\infty}\frac{\ds \qp{q/t}{n}}
    {\ds \qp{q}{n}}\left[ \frac{\ds \qp{tq}{\infty}\qp{q}{n-1}}{\ds 
    \qp{q}{\infty}\qp{tq}{n-1}}+\frac{\ds \qp{t}{\infty}}{\ds 
    \qp{q}{\infty}}\frac{\ds 1}{\ds 1-q^{n}}\right]t^{n},
    \]
    from which the result follows.
    \item[(vii)] Let $t\to 0$ in (vi).
    \item[(viii)] Let
    \[
    g(x) = \qp{qx}{\infty} = \dsum_{n=0}^{\infty}(-1)^{n}\frac{\ds 
    q^{n(n+1)/2}}{\ds \qp{q}{n}} x^{n}
    \]
    in Theorem~\ref{gthm1ajo}.
    \item[(ix)] Let $t\to 0$ in (viii).
    
    \end{enumerate}
\end{proof}

In a similar way, it is possible to obtain the following version
of Theorem~2 in~\cite{ajo}.
\begin{thm}
    \[
    \sum_{n=0}^{\infty} \left[\frac{\ds \qp{a}{n}\qp{b}{n}}{\ds 
    \qp{c}{n}\qp{q}{n}} - \frac{\ds \qp{a}{\infty}\qp{b}{\infty}}{\ds 
    \qp{c}{\infty}\qp{q}{\infty}}\right] = \frac{\ds \qp{a}{\infty}\qp{b}{\infty}}{\ds 
    \qp{c}{\infty}\qp{q}{\infty}}
    \sum_{n=1}^{\infty}\left[\frac{\ds \qp{q/b}{n}}{\ds 
    \qp{q}{n}} b^{n} +
    \frac{\ds \qp{c/a}{n}}{\ds 
    \qp{b}{n}} a^{n}\right]\frac{\ds 
    1}{\ds 1-q^{n}}.
    \]
\end{thm}
\begin{proof}
    Let
    \[
    f(x) = \frac{\ds \qp{aqx}{\infty} \qp{bqx}{\infty}}{\ds 
    \qp{cqx}{\infty} \qp{dqx}{\infty}} -1.
    \]
    Since
    \[
    \frac{\ds \qp{aqx}{\infty}}{\ds \qp{cqx}{\infty}} = 
    \dsum_{n=0}^{\infty}\frac{\ds \qp{a/c}{n}}{\ds \qp{q}{n}} 
    c^{n}q^{n}x^{n},
    \]
    we have that
    \[
    f(x) = \dsum_{n=0}^{\infty}\frac{\ds 
    \qp{\alpha}{n}}{\qp{q}{n}}(cqx)^{n}\dsum_{n=0}^{\infty}\frac{\ds 
    \qp{\beta}{n}}{\qp{q}{n}}(dqx)^{n},
    \]
    where $\alpha= a/c$ and $\beta= b/d$, and hence
    \[
    f(x) = \dsum_{n=1}^{\infty}\left[\dsum_{k=0}^{n}\frac{\ds
    \qp{\alpha}{k}\qp{\beta}{n-k}}{\ds 
    \qp{q}{k}\qp{q}{n-k}}c^{k}d^{n-k}\right]q^{n}x^{n}.
    \]
    Using this in Lemma~\ref{exchangelem} with $g(x)=1/(1-x)$ yields
    \begin{equation}\label{fourp}
    \dsum_{n=1}^{\infty}\left[\dsum_{k=0}^{n}\frac{\ds
    \qp{\alpha}{k}\qp{\beta}{n-k}}{\ds 
    \qp{q}{k}\qp{q}{n-k}}c^{k}d^{n-k}\right]\frac{\ds q^{n}}{\ds 1-q^{n}}
    x^{n} = \dsum_{n=0}^{\infty}\left[ \frac{\qp{aq^{n+1}x}{\infty} \qp{bq^{n+1}x}
    {\infty}}{\ds \qp{cq^{n+1}x}{\infty} 
    \qp{dq^{n+1}x}{\infty}}-1\right].
    \end{equation}
    
    The left--hand side in~(\ref{fourp}) equals
    \[
    \dsum_{n=1}^{\infty}\frac{\qp{\beta}{n}}{\ds \qp{q}{n}}\frac{\ds 
    (dqx)^{n}}{\ds 1-q^{n}} + \dsum_{k=1}^{\infty}\dsum_{n=k}^{\infty}
    \frac{\ds \qp{\alpha}{k}\qp{\beta}{n-k}}{\ds \qp{q}{k}\qp{q}{n-k}}
    \frac{\ds c^{k} d^{n-k} (qx)^{n}}{\ds 1-q^{n}},
    \]
    and the second term in this expression is in turn equal to
    \begin{equation}\label{exp1}
    \dsum_{k=1}^{\infty}
    \frac{\ds \qp{\alpha}{k}}{\ds \qp{q}{k}}(cqx)^{k}\dsum_{n=0}^{\infty}
    \frac{\ds \qp{\beta}{n}}{\ds \qp{q}{n}}\frac{\ds  (dqx)^{n}}{\ds 1-q^{n+k}}. 
    \end{equation}
    If we now let
    \[
    g(x) = \frac{\ds x}{\ds 1-q^{k}x} \mbox{ and }
    f(x) = \frac{\ds \qp{bx}{\infty}}{\qp{dx}{\infty}}
    \]
    in Lemma~\ref{exchangelem}, we obtain that~(\ref{exp1}) equals
    \[
    \frac{\ds \qp{bqx}{\infty}}{\ds 
    \qp{dqx}{\infty}}\dsum_{n=0}^{\infty}\frac{ \ds \qp{dqx}{n}}{\ds 
    \qp{bqx}{n}}q^{kn}.
    \]
    
    On the other hand, the right--hand side in~(\ref{fourp}) equals
    \[
    \frac{\ds \qp{aqx}{\infty}\qp{bqx}{\infty}}{\ds \qp{cqx}{\infty}
    \qp{dqx}{\infty}}\dsum_{n=0}^{\infty}\left[ \frac{\ds \qp{cqx}{n}
    \qp{dqx}{n}}{\ds \qp{aqx}{n}\qp{bqx}{n}} - \frac{\ds 
    \qp{cqx}{\infty}
    \qp{dqx}{\infty}}{\ds \qp{aqx}{\infty}\qp{bqx}{\infty}}\right].
    \]
    
    If we now let $x=1/b$ throughout, we obtain
    \[
    \begin{array}{l}
    \frac{\ds \qp{aq/b}{\infty}\qp{q}{\infty}}{\ds \qp{cq/b}{\infty}
    \qp{dq/b}{\infty}}\dsum_{n=0}^{\infty}\left[ \frac{\ds \qp{cq/b}{n}
    \qp{dq/b}{n}}{\ds \qp{aq/b}{n}\qp{q}{n}} - \frac{\ds 
    \qp{cq/b}{\infty}
    \qp{dq/b}{\infty}}{\ds 
    \qp{aq/b}{\infty}\qp{q}{\infty}}\right] \eqskip\\
    \hspace*{1cm}= \dsum_{n=1}^{\infty}\frac{\qp{\beta}{n}}{\ds \qp{q}{n}}\frac{\ds 
    (q/\beta)^{n}}{\ds 1-q^{n}} + \frac{\ds \qp{q}{\infty}}{\ds 
    \qp{q/\beta}{\infty}} \dsum_{k=1}^{\infty}\frac{\ds \qp{\alpha}{k}}
    {\ds \qp{q}{k}}\left(\frac{\ds cq}{\ds b}\right)^{k}
    \dsum_{n=0}^{\infty}\frac{ \ds \qp{q/\beta}{n}}{\ds 
    \qp{q}{n}}q^{kn}\eqskip\\
    \hspace*{1cm}= \dsum_{n=1}^{\infty}\frac{\qp{\beta}{n}}{\ds \qp{q}{n}}\frac{\ds 
    (q/\beta)^{n}}{\ds 1-q^{n}} + \frac{\ds \qp{q}{\infty}}{\ds 
    \qp{q/\beta}{\infty}} \dsum_{k=1}^{\infty}\frac{\ds \qp{\alpha}{k}}
    {\ds \qp{q}{k}}\left(\frac{\ds cq}{\ds 
    b}\right)^{k}\qp{q^{k+1}/\beta}{\infty}\frac{\ds 1}{\ds 
    \qp{q^{k}}{\infty}}\eqskip\\
    \hspace*{1cm}= \dsum_{n=1}^{\infty}\frac{\qp{b/d}{n}}{\ds \qp{q}{n}}\frac{\ds 
    (dq)^{n}}{\ds b^{n}(1-q^{n})} +\dsum_{n=1}^{\infty}\frac{\ds 
    \qp{a/c}{n}}{\ds \qp{dq/b}{n}}\left(\frac{\ds cq}{\ds 
    b}\right)^{n}\frac{\ds 1}{\ds 
    1-q^{n}}
    \end{array}
    \]
    Substituting now
    \[
    cq/b \to a, \;\; dq/b \to b \mbox{ and } aq/b \to c,
    \]
    gives the desired result.
\end{proof}

\section{\label{last}Comments and open questions}

The fecundity of Proposition~\ref{propajo} in applications as noted in 
the Introduction (cf.~\cite{ajo,coog,coon,loon}) suggests that there 
might also be such application for our main results in Section 2.

We are also drawn to the ``Liouville Mystery'' which is described in 
detail in~\cite{andr2} and in Section 7 of~\cite{andr3}. The mystery 
in question concerns the possibility of finding a linear dependency 
between series $\mathcal{L}_{k}(q)$ defined by
\[
\mathcal{L}_{k}(q) = \dsum_{n_{1},\ldots,n_{k+1}\geq1}q^{n_{1}n_{2}+
n_{2}n_{3}+\ldots+n_{k}n_{k+1}},
\]
and the series appearing in our Theorem~\ref{liouville}.

As we have observed in Section~\ref{prel}, our main result is purely 
and simply the natural generalization of Abel's Lemma. Given the 
widespread use of Abel's Lemma and its appearance in numerous classic 
texts on complex variable, one may well wonder why our main results 
were not proved years ago. It seems to us plausible that this is 
explained by condition (ii) in Propositions~\ref{propajo} and~\ref{pextension}.
The natural classical applications would be to convergent Gaussian 
hypergeometric series with argument $1$. These converge too slowly to 
allow condition (ii) to hold.

% ------------------------------------------------------------------------



\subsection*{Acknowledgment} The work of the second author was 
partially done while visiting Penn State University. He would like to 
thank the first author for his hospitality.
%Here go the acknowledgments, if any.

% ------------------------------------------------------------------------


\begin{thebibliography}{9999}
    
\bibitem[A1]{andr} G. E. Andrews, Ramanujan's ``lost'' notebook, V: 
Euler's partition identity, Adv. Math. {\bf 61} (1986), 156-164.

\bibitem[A2]{andr2} G. E. Andrews, Stacked lattice boxes, Ann. Comb. 
{\bf 3} (1999), 115--130.

\bibitem[A3]{andr3} G. E. Andrews, Some debts I owe, from The Andrews 
Festschrift, D. Foata and G.-N. Han, eds., Springer, Berlin, 2001, 
1--16.

\bibitem[AJO]{ajo} G. E. Andrews, J. Jim\'{e}nez-Urroz and K. Ono, 
{\em $q-$series identities and values of certain $L-$functions},  Duke Math. 
J. {\bf 108} (2001), 395--419.

\bibitem[ACS]{acs} G. E. Andrews, D. Crippa and K. Simon, $q-$series 
arising from the study of random graphs, SIAM J. Discrete Math. {\bf 
10} (1997), 41--56.

\bibitem[C]{coog} G. Coogan, More generating functions for $L-$function 
values, Contemp. Math. {\bf 291} (2001), 109-114.

\bibitem[CO]{coon} G. Coogan and K. Ono, A $q-$series identity and the 
arithmetic of Hurwitz zeta functions, Proc. Amer. Math. Soc. (to 
appear).

%\bibitem[D]{dilc} K. Dilcher, Some $q-$series identities related to 
%divisor functions, Discrete Math. {\bf 145} (1995), 83--93.

\bibitem[F]{fine} N. J. Fine, {\em Basic Hypergeometric Series and 
Applications}, Math. Surveys Monogr. {\bf 27}, Amer. Math. Soc., 
Providence, 1988.

\bibitem[LO]{loon} J. Lovejoy and K. Ono, Hypergeometric generating 
functions for values of Dirichlet and other $L-$functions, Proc. Nat. 
Acad. Sci. U.S.A. (to appear).

\bibitem[R]{rama} S. Ramanujan, {\em The Lost Notebook and Other 
Unpublished Papers}, Narosa, New Delhi 1988.

\end{thebibliography}

\end{document}

