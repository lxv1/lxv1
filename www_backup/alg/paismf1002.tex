%
\magnification=\magstep1
\input amstex
\input table
\documentstyle{amsppt}
\NoBlackBoxes
\pageheight{7.5in}
\pagewidth{5.5in}

\leftheadtext{George E. Andrews}
\TagsOnRight
\def\c{\cite}
\def\fr{\frac}
\def\D{\Cal{D}}
\def\R{\frak{R}}
\def\pf{\hfill $\square$}
\def\wa{\widetilde{\alpha}}
\def\wb{\widetilde{\beta}}

\topmatter
\title
	Partitions: \ At the Interface of $q$-Series and 
	Modular Forms
\endtitle
\author
	George E. Andrews\footnote{Partially 
	supported by National Science Foundation Grant
	DMS-9206993. \hfill\hfil }
\endauthor
\address
	The Pennsylvania State University  
	University Park, PA  16802
\endaddress
\abstract
	In this paper we explore five topics from the theory of 
	partitions: (1) \ the Rademacher conjecture, (2) the
	Herschel-Cayley-Sylvester formulas, (3) the asymptotic
	expansions of E. M. Wright, (4) the asymptotics of mock 
	theta function coefficients, (5) modular transformations
	of $q$-series
\endabstract
\endtopmatter


\document

\subhead
1. \ Introduction
\endsubhead

The twentieth century found two major currents in the theory of
partitions.  The first is characterized by the role played by modular
forms.  Seminal papers in this vein include the Hardy-Ramanujan paper
on $p(n)$ \c{17}, the proofs of the Ramanujan congruences by Watson
\c{38} and Atkin \c7, and the many recent contributions by Ken Ono
\c{25}, \c{26} to name only a few.

The second has been extensive study of the application of basic
hypergeometric series or $q$-series to partitions.  This has its
genesis in the work of MacMahon \c{22} and Schur \c{34}.  In the
1960's a variety of results were discovered; these are described in
a survey by H. L.  Alder \c1 (cf. \c3).

Surprisingly perhaps, there has not been a large amount of interplay
between these two themes.  In this paper, I hope to survey topics
that lie at the interface of modular forms and $q$-series.  Some
concern important and unjustly neglected conjectures.  Each should
suggest a number of research possibilities.

Section 2 will be devoted to Rademacher's conjecture \c{29; p. 302}
which concerns his partial fraction decomposition for the generating 
function of $p(n)$.  Section 3 considers alternative representations
of the restricted partition function $p(n,m)$.  Section 4 looks at
a sequence of papers by E. M. Wright \c{39}, \c{40}, \c{41}.  The
third of these is quite unlike any other in the history of partition
asymptotics and foreshadows many possibilities.  In Section 5, we 
look again at problems posed by Dragonette \c{12} for the asymptotics
of the coefficients of Ramanujan's mock theta functions.  The sixth
section considers the efforts by Ehrenpreis \c{13} and others to
develop a modular type transformation theory for $q$-series.  We
conclude with brief synthesizing observations.

\subhead
2. \ The Rademacher Conjecture
\endsubhead

Near the end of his book, Topics in Analytic Number Theory \c{29;
p. 300}, Rademacher derives the following partial fraction
decomposition for the generating function of the partition function
$p(n)$.

\head
Rademacher's Partial Fraction Decomposition
\endhead

We use the following notation for forward differences
$$
	\Delta_{\alpha} f(\alpha) = f(\alpha + 1) - f(\alpha)\,,
$$
and for nonnegative integers $j$
$$
	\Delta_{\alpha}^j f(\alpha) = \sum_{h=0}^j (-1)^h
	\binom{j}{h} f(\alpha + j - h).
$$
Then
$$
\align
	F(x) =  & - 2\pi \left(\fr{\pi}{12}\right)^{3/2}
	\sum_{k=1}^{\infty} k^{-5/2}  \sum_{\Sb h \mod k \\
	(h,k) = 1 \endSb} w_{h,k}  \tag2.1    \\
	& \sum_{j=0}^{\infty} \Delta_{\alpha}^j \left[ L_{3/2}
	\left( - \fr{\pi^2}{6k^2} (\alpha + 1) \right)\right]
	\left(x \,e^{-\fr{2\pi ih}{k}} - 1\right)^{-j-1},
\endalign
$$
where $\alpha = 1/24$, $w_{hk}$ is a root of unity \c{29; p. 269} and
$$
	F(x) = \cases
	\displaystyle{\sum_{n=0}^{\infty}} p(n) x^n\,, \qquad & |x| < 1  \\
	0\,,  \qquad & |x| > 1\,.
	\endcases
$$

Immediately after proving the above expansion for $F(x)$, 
Rademacher \c{29, p. 301} observes that:

If $|x| < 1$,
$$
\align
	\lim_{N\rightarrow\infty} \fr1{(1 - x)(1 - x^2) \cdots
	(1 - x^N)} & = \fr1{\displaystyle{\prod_{m=1}^{\infty}}
	(1 - x^m)}  \tag2.3    \\
	& = \sum_{n=0}^{\infty} p(n) x^n = F(x)\,,
\endalign
$$
and if $|x| > 1$
$$
	\lim_{N\rightarrow \infty}  \fr1{(1 - x)(1 - x^2) \cdots
	(1 - x^N)} = 0 = F(x).
\tag2.4
$$

Of course, one can perform the classical partial fractions decomposition
of this finite product.  Indeed, we may write
$$
	\fr1{\displaystyle{\prod_{m=1}^N}(1 - x^m)} = 
	\sum_{\Sb 0 \leqq h < k \leqq n \\ (h,k) = 1 \endSb}
	\sum_{j=1}^{\lfloor N/k\rfloor}
	\fr{C_{hkj}(N)}{(x - e^{\fr{2\pi ih}{k}})^j}\;.
\tag2.5
$$

As Rademacher notes, there is a natural identification 
suggested by these two expansions.
\vskip .1in
{\bf Rademacher's Conjecture} \c{29; p. 302}.

$$
	\lim_{N\rightarrow \infty}  C_{hkj}(N)
$$
exists and is given by
$$
	C_{hkj}(\infty) = -2\pi \left(\fr{\pi}{12}\right)^{3/2}
	\fr{w_{hk} e^{\fr{2\pi ihj}{k}}}{k^{5/2}}
	\Delta_{\alpha}^{j-1} L_{3/2} \left( - 
	\fr{\pi^2}{6k^2} (\alpha + 1)\right)\,,
$$
where $\alpha = 1/24$, and 
$$
\aligned
	L_{3/2}(-y^2) & = - \fr1{2\sqrt{\pi}y} \;\fr{d}{dy}\;
	\left(\fr{\sin 2y}{y}\right)   \\
	& = - \fr1{2\sqrt{\pi} y^2} \left( 2 \cos 2y -
	\fr{\sin 2y}{y} \right)\,.
\endaligned
$$
Rademacher goes on to provide a small table of values (reprinted
here with one correction):
\vskip .1in
\centerline{\begintable
\begintableformat
\center " \left " \left " \left
\endtableformat
\-
\br{\:} N " $C_{011}$ " $C_{012}$  " $C_{121}$  \er{}
\-
\br{\:} ${}$ " ${}$ " ${}$ " ${}$ \er{}
\br{\:} $1$ " $-1$ " $0$ " $0$ \er{}
\br{\:} ${}$ " ${}$ " ${}$ " ${}$ \er{}
\br{\:} $2$ " $-\fr14 = - 0.25$ " $\fr12 = 0.5$ " $\fr14 = 0.25$ \er{}
\br{\:} ${}$ " ${}$ " ${}$ " ${}$ \er{}
\br{\:} $3$ " $-\fr{17}{72} = - 0.23611 \dots$ " $\fr14 = 0.25$ " 
$\fr18 = 0.125$ \er{}
\br{\:} ${}$ " ${}$ " ${}$ " ${}$ \er{}
\br{\:} $4$ " $-\fr{17}{72} = -0.23611 \dots$ " 
$\fr{59}{288} = 0.204861\dots$\quad " $\fr18 = 0.125$ \er{}
\br{\:} ${}$ " ${}$ " ${}$ " ${}$ \er{}
\br{\:} $5$ " $-\fr{20831}{86400} = -0.24101 \dots$\quad " 
$\fr3{16} = 0.1875$ " $\fr{13}{128} = 0.1015625$ \er{}
\br{\:} ${}$ " ${}$ " ${}$ " ${}$ \er{}
\-
\endtable}
\vskip .1in
and he lists explicitly three of the conjectured limits:
$$
\aligned
	C_{011}(\infty) & = - \fr6{25} \left(1 + \fr{2\sqrt{3}}{5\pi}
	\right) = - 0.273339 \dots,   \\
	C_{012}(\infty) & = \fr{24}{25\cdot 49} \left( 6 + 
		\fr{109\sqrt{3}}{35\pi} \right) = 0.15119 \dots,  \\
	C_{121}(\infty) & = - \fr{\sqrt{6}}{25} \left( \cos 
	\fr{5\pi}{12} - \fr{12}{5\pi} \sin \fr{5\pi}{12} \right)
	= 0.046941.
\endaligned
$$

Of these coefficients he says \c{29; p. 301}:

``No explicit formula for $C_{hkl}(N)$ is known, not even for the
simplest case $h = 0$, $k = 1$, $l = 1$, and variable $N$.''

In the hope of stimulating serious consideration of this problem, I
will present an explicit formula for $C_{011}(N + 1)$ below.  The same
procedure can obviously be applied to provide a formula for
$C_{hkj}(N+1)$ in general.  However, owing to the fact that we cannot
prove Rademacher conjecture from our result, we shall not complicate
matters by presenting the full formula for $C_{hkj}(N+1)$.  It should
be pointed out that Rademacher's conjecture lies at the interface of
the theory of modular forms and the theory of $q$-series.  Thus
progress on this problem may require contributions from two areas that
have had less contact in the past than might have been expected or
hoped for.

\proclaim
{Theorem 1}  With $\rho_j = e^{2\pi i/j}$, and $H_N(x_1,\dots,x_n)$
the $N$-th homogeneous symmetric function of $x_1,x_2,\dots,x_n$,
then
$$
\aligned
	C_{011} & (N + 1)  \ \   \\
	& = \fr{-1}{(N+1)!} \sum_{h_1 = 1}^1 \sum_{h_2 = 1}^2
	\sum_{h_N = 1}^N   \\
	& \qquad \left(\prod_{i=1}^{N} \rho_{i+1}^{-h_i}
	\right)\;H_N\; \left(\fr{\rho_2^{h_1}}{1 - \rho_2^{h_1}},\dots,
	\fr{\rho_{N+1}^{h_N}}{1 - \rho_{N+1}^{h_N}}\right)\,.
\endaligned
$$
\endproclaim

\demo
{Proof}  Noting that (2.5) may be presented as
$$
\aligned
	\fr1{(x;x)_{N+1}} = & \fr{C_{01N+1}(N+1)}{(x - 1)^{N+1}}
	+ \fr{C_{01N}(N+1)}{(x - 1)^N} + \cdots + \fr{C_{011}(N+1)}
	{(x - 1)}  \\
	& + \sum_{\Sb 0 \leqq h < k \leqq N + 1 \\ (h,k) = 1  \\
	k > 1 \endSb} \sum_{j=1}^{\lfloor n/k\rfloor}
	\fr{C_{hkj}(N+1)}{(x - e^{2\pi ih/k})^j}\,,
\endaligned
$$
we find by multiplying by $(x - 1)^{N+1}$, differentiating $N$ times
and then setting $x = 1$ that
$$
	N! \;C_{011}(N+1) = \left. \fr{d^N}{dx^N} \;
	\fr{(x - 1)^{N+1}}{(x;x)_{N+1}} \right|_{x=1} \,,
$$
where
$$
	(A;q)_N = (1 - A)(1 - Aq) \dots(1 - Aq^{N-1}).
$$

So we clearly need the $r$-dimensional Leibnitz rule
$$
	\fr{d^N}{dx^N} \;f_1 \,f_2 \cdots f_r = \sum_{\Sb
	n_1,\dots,n_r \geqq 0  \\  n_1 + \cdots + n_r = N\endSb}
	\binom{N}{n_1,n_2,\dots,n_r}  f_1^{(n_1)} f_2^{(n_2)}
	\cdots f_r^{(n_r)}\,.
$$

Now recalling that
$$
	\fr1{1 - q^M} = \fr1{M} \sum_{J=0}^{M-1} 
		\fr1{(1 - \rho_M^j q)}\;,
$$
where $\rho_M = e^{2\pi i/M}$, we easily deduce
$$
	\fr{1 - q}{1 - q^M} = \fr1{M} \sum_{j=1}^{M-1}
	\fr{(1 - \rho_M^{-j})}{(1 - \rho_M^j q)}\;.
$$
Consequently
$$
	\fr{d^k}{dq^k}\; \fr{(1 - q)}{(1 - q^M)} = \fr{k!}{M}
	\sum_{j=1}^{M-1} \fr{(1 - \rho_M^{-j})\rho_M^{jk}}
	{(1 - \rho_M^j q)^{k+1}}\;.
$$
Next we recall the standard representation of the homogeneous 
symmetric function in terms of monomials:
$$
	H_N(X_1,X_2,\dots,X_N) = \sum_{\Sb n_1,n_2,\dots,n_N \geqq 0\\
	n_1 + n_2 + \cdots + n_N = N \endSb} X_1^{n_1} X_2^{n_2}
	\cdots X_N^{n_N}\,.
$$
As a result,
$$
\aligned
	& N!\,C_{011}(N + 1)   \\
	& = (-1)^{N+1} \sum_{\Sb n_1,\dots,n_N \geqq 0 \\
	n_1 + \cdots + n_N = N \endSb} \binom{N}{n_1,n_2,\dots,n_N}
	\left(\fr{d^{n_1}}{dx^{n_1}} \;\fr{(1 - x)}{(1 - x^2)}\right)
	\cdots \left.\left( \fr{d^{n_N}}{dx^{n_N}}\;\fr{(1 - x)}{(1 - x^{N+1})}
	\right)\right|_{x = 1}
	\\
	& = (-1)^{N+1} \sum_{\Sb n_1,\dots,n_N \geqq 0 \\
	n_1 + \cdots + n_N = N \endSb} \binom{N}{n_1,n_2,\dots,n_N}
	\left(\fr{n_1!}{2}\; \sum_{j_1 = 1}^{2-1} \fr{(1 - \rho_1^{-j_1})
	\rho_2^{j_1 n_1}}{(1 - \rho_2^{j_1} q)^{n_1 + 1}}\right)
	\\
	& \qquad \cdots \left(\fr{n_N !}{(N+1)} \sum_{j_N = 1}^{(N+1)-1}
	\left.
	\fr{(1 - \rho_{N+1}^{-j_N})\rho_{N+1}^{j_N n_N}}{(1 - \rho_{N+1}^{j_N}
	)^{n_N + 1}}\right) \right|_{x=1}
	\\
	& = \fr{(-1)^{N+1}}{(N+1)!} \sum_{j_1=1}^1\sum_{j_2=1}^2 \cdots
	\sum_{j_N=1}^N \fr{(1 - \rho_2^{-j_1})}{(1 - \rho_2^{j_1})}\;
	\fr{(1 - \rho_3^{-j_2})}{(1 - \rho_{3}^{j_2})} \cdots
	\fr{(1 - \rho_{N+1}^{-j_N})}{(1 - \rho_{N+1}^{j_N})}  
	\\
	& \qquad \sum_{\Sb n_1,\dots,n_N \geqq 0 \\ n_1 + \cdots + n_N = N 
	\endSb} \left(\fr{\rho_{N+1}^{j_N}}{1 - \rho_{N+1}^{j_N}}\right)^{n_N}
	\cdots \left(\fr{\rho_2^{j_1}}{1 - \rho_2^{j_1}}\right)^{n_1}
	\\
	& = \fr{-1}{(N+1)!} \sum_{j_1 = 1}^1 \sum_{j_2 = 1}^2 \cdots
	\sum_{j_N = 1}^N \left(\prod_{i=1}^N \rho_{i+1}^{-j_i}\right)
	\;H_N\; \left(\fr{\rho_2^{j_1}}{1 - \rho_2 j_1}\,,\cdots ,\,
	\fr{\rho_{N+1}^{j_N}}{1 - \rho_{N+1}^{j_N}}\right)\,,
\endaligned
$$
as desired.  \pf
\enddemo

It may be reasonably objected that Theorem 1 provides little hope of
proving Rademacher's Conjecture.  At the most it may suggest the
value of finding better formulas for $C_{hkj}(N)$.

L. Ehrenpreis in \c{13; p. 317} mentions that his student, Jane
Friedman, studied computer algorithms for this problem but he states
that: ``Unfortunately, the computer study proved inconclusive.''
While this is still true, we can add a few more values to those
already computed for $C_{011}(N)$:
$$
\alignedat3
	& N & \qquad & C_{011}  \\
	& 6 & \qquad & - \fr{85823}{345600} \; = \; -0.24833  \\
	& 7 & \qquad & - \fr{19554517}{76204800} \; = \; -0.25660 \\
	& 8 & \qquad & - \fr{80858443}{304819200} \; = \; -0.26527
\endalignedat
$$

\subhead
3. \ Formulae for p(n,m)
\endsubhead

We denote the number of partitions of $n$ into at most $m$ parts by
$p(n,m)$.  There is an extensive literature concerning formulae for
$p(n,m)$, there is a full account of the history up to 1958 given in
the Royal Society Table of Partitions \c{16} which is mostly devoted
to extensive tables of $p(n,m)$.  Many authors, among them, Paoli
\c{27}, DeMorgan \c{11}, Herschel \c{18}, Cayley \c{10}, Sylvester
\c{35}, Glaisher \c{14} and more recently Arkin \c6,
have all made major contributions to the study of exact formulae for 
$p(n,m)$.

When one proceeds to examine the actual formulae, one is struck by a
substantial difference that arises between $m \leqq 4$ and $m > 4$.
Namely, we see immediately that
$$
	p(n,1) = 1
\tag3.1
$$
$$
	p(n,2) = \left\lfloor \fr{n+2}{2}\right\rfloor\,.
\tag3.2
$$
DeMorgan \c{11} first proved that
$$
	p(n,3) = \left\{ \fr{(n+3)^2}{12} \right\}\,,
\tag3.3
$$
and Gl\"osel proved \c{15; p. 138}:
$$
	p(n,4) = \left\{ \left\lfloor \fr{n+4}{2}\right\rfloor^2 \left( 3
	\left\lfloor \fr{n+9}{2} \right\rfloor - \left\lfloor 
	\fr{n+10}{2}\right\rfloor \right) \fr1{36}\right\}\,.
\tag3.4
$$
where $\lfloor x \rfloor$ is the greatest integer $\leqq x$, and
$\{x\}$ is the nearest integer to $x$.

Now there are several immediate observations to make about these
formulae.  First of all, they are all obviously real numbers indeed,
positive integers.  Furthermore, they are given in one line and are
computed with familiar functions.  We do remark that (3.4) could be
improved if one could eliminate some of the instances of the greatest
integer function.

For $m > 5$ (Gl\"osel has a lengthy formula for $p(n,5)$ that is of
the same type as (3.1)--(3.4)), one runs into the stark fact that one
needs representations of periodic sequences.  As is well-known \c{28,
p. 76}, such sequences are representable by Finite Fourier series.
So, for example, an alternative to (3.4) is \c{23; p. 153}
$$
\aligned
	p(n,4) = & \fr1{144} (n + 5)^3 - \fr5{96} (n + 5) + \fr1{32}
	(-1)^n (n + 5) \\
	& + \fr1{27} (r_3^n + r_3^{-n} - r_3^{n+1} - r_3^{-n-1})
	\\
	& + \fr1{16} i^n (1 + (-1)^n)\,,
\endaligned
$$
where $r_3 = e^{2\pi i/3}$.

Now this formula is rather disturbing.  A little thought reveals that
it is real because it is equal to its complex conjugate.  However, its
integrality is far from obvious.

One way to avoid this difficulty is to introduce the circulator
notation for a periodic sequence of period $m$ \c{16; p. xvii}
$$
	(A_0,A_1,\dots,A_{m-1}) crm_n = A_i \text{ if } 
	n \equiv i \pmod{m}\,.
\tag3.6
$$
Computationally this approach avoids complex numbers, but it merely
introduces notation for periodic sequences.

In this new notation we find \c{16; p. xxvii, eq. (6.7)}
$$
\aligned
	p(n,4) = & \fr1{24}  \binom{n+6}{3} + \left(\fr1{9} - \fr1{16}
	(2,3) cr 2_n\right) (n + 5)  \\
	& + \fr1{9} (1,0,-1) cr 3_n + \fr1{8} (1,0,-1,0) cr 4_n.
\endaligned
\tag3.7
$$
In this section, we shall make the simple observation
that periodic sequences may also be represented by the greatest 
integer function.  From that knowledge it is easy to derive formulas
such as
$$
	p(n,4) = \left\{(n + 5)\left(n^2 + n+ 22 + 18 \left\lfloor
	\fr{n}{2}\right\rfloor\right)\bigg/144 \right\}\,,
\tag3.8
$$
and
$$
	p(n,5) = \left\{(n + 8)\left(n^3 + n+ 22n^2 + 44n + 248 +
	180 \left\lfloor
	\fr{n}{2}\right\rfloor\right)\bigg/2880 \right\}\,,
\tag3.9
$$

Note that these formulas are much more in the spirit of (3.1)--(3.3)
than is (3.5).  The formula (3.9) is strikingly simpler than any
of the several given in \c{16}.  Furthermore such formulas for
$p(n,m)$ continue for larger $m$.  For example, 
$$
\align
	p(n,6) = & \bigg\{(n + 11) \bigg((6n^4 + 249 n^3 + 2071n^2
	- 4931n + 40621)\big/518400   \tag3.10   \\
	& \left.\left. + \left\lfloor \fr{n}{2}\right\rfloor
	(n + 10)/192 + \left(\left\lfloor \fr{n+1}{3}\right\rfloor
	+ 2\left\lfloor \fr{n}{3}\right\rfloor \right)\big/54\right)
	\right\}
\endalign
$$
$$
\align
	p(n,7) = & \bigg\{(n + 14) \bigg((n^5 + 70 n^4 + 1785n^3
	- 15365 n^2 + 9702n + 277032\big/3628800   \tag3.11   \\
	& \left.\left. + \left\lfloor \fr{n}{2}\right\rfloor
	(n + 14)/384 + \left\lfloor \fr{n}{3}\right\rfloor
	\big/54\right)\right\}
\endalign
$$
$$
\align
	p(n,8) = & \left\{(n+18)\bigg((n^6 + 108n^5 + 4503n^4 + 79911n^3 
	\right.   \tag3.12  \\
	& + 522148n^2 - 202687n + 9441216)/203212800  \\	
	& + \left\lfloor \fr{n}{2}\right\rfloor (n^2 + 36 + 231)/9216 \\
	& + \left(\left\lfloor \fr{n +1}{3}\right\rfloor + 2
	   \left\lfloor \fr{n}{3}\right\rfloor\right)\big/162  \\
	& + \left.\left.\left\lfloor \fr{n}{4}\right\rfloor \big/64
	\right)\right\}
\endalign
$$
$$
\align
	p(n,9) = & \left\{(n+22)\bigg((n^7 + 158n^6 + 10034n^5 + 327352n^4 
	\right.   \tag3.13  \\
	& + 5419144n^3 - 32063602n^2 + 5172096n + 564401888)/14631321600  \\
	& + \left\lfloor \fr{n}{2}\right\rfloor (2n^2 + 91n + 728)/36864 \\
	& + \left(\left.\left.(n + 20)\left\lfloor \fr{n+1}{3}\right\rfloor 
	+ 2(n + 23)\left\lfloor \fr{n}{3}\right\rfloor \right)
	\bigg/2916 + \left(\left\lfloor \fr{n}{4}\right\rfloor +
	\left\lfloor \fr{n + 2}{4}\right\rfloor \right)\bigg/256
	\right)\right\}
\endalign
$$

In order to establish such results, we require the following:

\proclaim
{Theorem 2}  For a fixed positive integer $m$, the periodic sequence 
in $n$ given by
$$
\align
	(A_0,A_1, & \dots,A_{m-1}) cr\,m_n  \tag3.14  \\
	& = A_0 \left\lfloor \fr{n+m}{m}\right\rfloor + \sum_{j=1}^{m-1}
		(A_j - A_{j-1}) \left\lfloor \fr{n + m-j}{m}\right\rfloor 
		- A_{m-1} \left\lfloor \fr{n}{m}\right\rfloor  \\
	& = A_0 + \sum_{j=1}^m (A_j - A_{j-1}) \left\lfloor \fr{n+m-j}{m}
		\right\rfloor \,,
\endalign
$$
where we define $A_m = A_0$.
\endproclaim

\demo
{Proof}  Let us define $f(n)$ to be the right-hand side of (3.14).
$$
\aligned
	f(n + m) - f(n) & = \sum_{j=1}^m (A_j - A_{j-1})\left(
	\left\lfloor \fr{n + 2m-j}{m}\right\rfloor  -
	\left\lfloor \fr{n + m - j}{m}\right\rfloor \right)
	\\
	& = \sum_{j=1}^m (A_j - A_{j-1}) = A_m - A_0 = 0\,.
\endaligned
$$
Hence $f(n)$ has period $m$.  Furthermore for $0 \leqq i < m$,
$$
\aligned
	f(i) & = A_0 + \sum_{j=1}^m (A_j - A_{j-1}) \left\lfloor 
		\fr{i + m-j}{m}\right\rfloor   \\
	& = A_0 + \sum_{j=1}^i (A_j - A_{j-1})   \\
	& = A_i\,.
\endaligned
$$
Therefore $f(n) = (A_0,A_1,\dots,A_{m-1})crm_n$ as desired.  \pf
\enddemo

\proclaim
{Corollary}  
$$
\align
	& (A_0,A_1,\dots,A_{n-1}) cr\,m_n  \\
	& = A_0 + (A_1 - A_0)n + \sum_{j=2}^m (A_j - A_{j-1} - A_1
		+ A_0)\left\lfloor \fr{n+m-j}{m}\right\rfloor \,.
\endalign
$$
\endproclaim

\demo
{Proof}  This follows immediately from Theorem 1 once we recall that
$$
	\left\lfloor \fr{n+m-1}{m}\right\rfloor  = n -
	\left\lfloor \fr{n}{m}\right\rfloor  -
	\left\lfloor \fr{n+1}{m}\right\rfloor  - \cdots -
	\left\lfloor \fr{n+m-2}{m}\right\rfloor \,.
$$
\pf
\enddemo


Formulas (3.8)--(3.13) are now easily derived from the literature, in
particular, from the formulas \c{16; pp. xxvii--xxviii, eq. (6.7)}.
Since the process is routine, we illustrate with the cases $m = 4$
and $5$.

Note that in (3.7) we may drop the $\fr19 (1,0,-1) cr 3_n + \fr18
(1,0,-1,0) cr 4_n$ because this expression is in absolute value
$< \fr19 + \fr18 < \fr12$.  Hence immediately
$$
\align
	p(n,4) & = \left\{ \binom{n+6}{3} \fr1{24} + \left( \fr19
	- \fr1{16} \left(2 + n - 2 \left\lfloor \fr{n}{2}\right\rfloor  
	\right)\right) (n+5)\right\}  \tag3.16
	\\
	& = \left\{(n+5)\left(n^2 + n + 22 + 18 \left\lfloor \fr{n}{2}
	\right\rfloor  \right)\big/144 \right\}\,,
\endalign
$$
as desired.

Similarly \c{6; p. xxvii, eq. (6.7)}
$$
\align
	p(n,5) & = \fr1{120} \binom{n+9}{4} - \fr5{288}\binom{n+8}{2}
	+ \fr1{64} (n + 8) (1,-1) cr 2  \tag3.17  \\
	& + \fr19 (1,0,0) cr 3_n + \fr1{64}(0,1,-8,-7) cr 4_n  \\
	& + \fr15 (1,0,0,0,0) cr 5_n  \\
	& =\left\{ \fr1{120} \binom{n+9}{4} - \fr5{288}\binom{n+8}{2}
	+ \fr1{64} (n+8)\left(1 - 2n + 4 \left\lfloor \fr{n}{2}\right\rfloor  
	\right) \right\}  \\
	& = \left\{ (n + 8)\left(n^3 + 22n^2 + 44n + 248 + 180 
	\left\lfloor \fr{n}{2}\right\rfloor  \right)\bigg/2880\right\}
\endalign
$$
as desired.

Formulas (3.10)--(3.13) are handled in exactly this way.

\subhead
4. \ Asymptotics of Non-modular Generating Functions
\endsubhead

The topic of this section has had extensive development since 1941
when Ingram's seminal paper appeared \c{19}.  Among the authors
obtaining general theorems on this topic are Bateman and Erdos \c8,
Nicolas and Sarkozy \c{24} Richmond \c{32}, Roth and Szekeres \c{33},
\c{36}, McIntosh \c{21}.  This list is undoubtedly incomplete and 
is actually only tangentical to the real object of this section.

Prior to Ingham's work, E. M. Wright wrote three papers on the
asymptotics of particular generating functions.  The first of these
\c{39} was devoted to the asymptotics of the plane partition function.
Its method was essentially the saddlepoint method that forms the
foundation of many of the succeeding works mentioned above.  Wright's
second paper \c{40} in this series was based on a more elementary than
analytic account but was not a surprising development.

The point I wish to make in the section is that Wright's third paper
\c{41} on partitions into powers IS UNIQUE in the history of this 
subject.  Its starting point and fundamental philosophy are different
from anything that has come before or since.

Wright's focus in \c{41} is
$$
	f(x) = \prod_{\ell=1}^{\infty} (1 - x^{\ell^k})^{-1}
	= 1 + \sum_{n=1}^{\infty} p_k (n) x^n\,,
\tag4.1
$$
where $p_k(n)$ is the number of partitions of $n$ into $k$-th
powers.

His results are intricate to say the least.  However, after
stating his central results we shall examine the import of his
work.

Write
$$
	a = \fr1{k}, \quad b = \fr1{k+1},\quad j = j(k) = 0\quad
	(k \text{ even})
\tag4.2
$$
$$
	j = j(k) = \fr{(-1)^{(k+1)/2}}{(2\pi)^{k+1}}
	\Gamma (k + 1) \zeta(k+1) \qquad (k \text{ odd}).
\tag4.3
$$
Define a generalized Bessel function as follows:
$$
	\varphi(z) = \sum_{\ell = 0}^{\infty} \fr{z^{\ell}}
	{\Gamma (\ell + 1) \Gamma (\ell a - \fr12)}
\tag4.4
$$

Wright then produces the following asymptotic formula for 
$p_k(n)$ which is obtained by examining in detail the singularity
of $f(x)$ at $x = 1$.
$$
\align
	p_k(n_ = (n + j)^{-\fr32} (2\pi)^{-\fr{k}{2}} \varphi
	(\Gamma(1 + a) \zeta(1 & + a)(n+j)^a) \tag4.5  \\
	& + 0 (e^{(\Delta - \alpha)n^b}),  
\endalign
$$
where
$$
	\Delta = (k + 1)(a \Gamma(1 + a) \zeta(1 + a))^{1-b},
	\qquad \alpha = \alpha(k) > 0.
\tag4.6
$$

Now the important point here is that the error term in (4.5) is
of \underbar{exponentially} lower order of magnitude than the
main term.

Wright accomplishes this by proving a complete generalization of 
the modular transformation formulae for Dedekind's eta function.

Let $p,h,s$ and $\ell$ be integers with $1 \leqq p < q$, $(p,q) = 1$
except that, when $q = 1$, $p = 0$.
$$
	1 \leq h \leq q, \qquad 1 \leq s \leq k, \qquad
	l \geq 0.
$$
If $d_h \neq 0$, we write
$$
	\mu_{h,s} = \fr{d_h}{q} (s \text{ odd}),\;\;\mu_{h,s}
	= \fr{q - d_h}{q} \;(s \text{ even}).
\tag4.7
$$
If $d_h = 0$, we take $\mu_{h,s} = 1$.  Hence always $q,\mu_{h,s}
\geq 1$. 

In connection with any particular values of $p$ and $q$ we write
$$
	X = xe_q(-p) = e^{-y}, \quad Y = q^k y,
\tag4.8
$$
and we take $y$ real and positive when $X$ is real and $0 < X < 1$.
We write also
$$
	t_s = \left(\fr{2\pi}{Y}\right)^a \exp \left\{ a \,\pi\,i
	\left( s - \fr12\right)\right\}\,,	
\tag4.9
$$
where $Y^a$ is that $k$-th root of $Y$ which is real and positive
when $Y$ is real and positive,
$$
\split
	g(h,l,s) = \exp \{2\,\pi\,i(l + \mu_{h,s})^a t_s\} e_q
	(-h)   \\	
	= \exp \left\{ \fr{(2\pi)^{1+a}(l + \mu_{h,s})^a e^{\fr12
	\pi a i (2s + k -1)}}{qy^n} - \fr{2h\,\pi\,i}{q} \right\}\,,
\endsplit
\tag4.10
$$
and
$$
	P_{p,q} = \prod_{h=1}^q \prod_{s=1}^k \prod_{l=1}^{\infty}
	\left\{ 1 - g(h,l,s)\right\}^{-1}\,.
\tag4.11
$$
Then we have:

\proclaim
{Wright's Transformation Theorem \c{41; p. 149, Th. 4}}
If $\R(y) > 0$, then $P_{p,q}$ is convergent and 
$$
	f(x) = f(e^{-y} e_q(p)) = C_{p,q} y^{\fr12} e^{jy} \exp
	\left( \fr{\Lambda_{p,q}}{y^a}\right) \;P_{p,q}\,,
\tag4.12
$$
where $C_{p,q}$ and $\Lambda_{p,q}$ are explicitly given constants
depending on $p$ and $q$.
\endproclaim

In the case $k = 1$, (4.12) reduces to the modular function
transformation used by Hardy and Ramanujan \c{17}.

In addition, various instances of $k = 2$ appear in \c{17} and
have been extended by R. J. Baxter \c9 in his proof of the Doochel Kim
conjectures \c{20}.

There are a couple of things to say about Wright's theorems.  First,
how large is the class of partition generating functions for which
there are analogs of Wright's Transformation Theorem?  Second, how
good is the full analog of the Hardy-Ramanujan-Rademacher formula for
$p(n)$ that Wright gives for $p_k(n)$ (Theorem 3 of \c{41} but not
restated here). Wright suggests that each of his main terms has order
$$
	n^{c_1} e^{c_2 n^{\fr1{k+1}}}.
$$
He goes on to compare his error term to the famous $O(n^{-\fr14})$
that appears in the Hardy-Ramanujan formula for $p(n)$ \c{17}.
``If we attempt to make a similar improvement for $p_k(n)$, we
find that we can choose $q_0 = q_k(k,n)$ so that the error term
is $O(e^{n^d})$, where $d < b$.  This is not so good as the result
for $k = 1$, and, in view of the heavy analysis required for this
further step, I am content to prove $[O(n^{en^0})]$.

One would hope that the advances on closely related work on modular
forms would allow a complete analysis of this error term.  Does it
tend to $0$ as $n \to \infty$?  If not, is it bounded by polynomial
growth?  If not, what then?

\subhead
5. \ The Power Series Coefficients of Mock Theta Functions
\endsubhead

In his last letter to G. H. Hardy \c{31; pp. 127--131}, Ramanujan
first presents his mock theta functions.  G. H. Hardy \c{30; p. 354}
summarizes the idea of a mock theta function: It is a function,
``defined by a $q$-series convergent for $|q| < 1$, for which we 
can calculate asymptotic formulae, when $q$ tends to a ``rational
point $e^{2\pi ir/s}$, of the same degree of precision as those
furnished, for the ordinary $\D$-functions, by the theory of linear
transformation.''  Subsequently, starting with G. N. Watson's 
famous L. M. S. Presidential address \c{37}, there have been a
number of studies of mock theta functions.  Much of this work was
summarized in a survey article \c5.  Ramanujan focuses upon one mock 
theta function in particular:
$$
	f(q) = 1 + \sum_{n=1}^{\infty} \fr{q^{n^2}}{(1 + q)^2
	(1 + q^2)^2 \cdots (1 + q^n)^2}\;.
\tag5.1
$$
He asserts that:  ``The coefficient of $q^n$ in $f(q)$ is
$$
	\fr{(-1)^{n-1} e^{\pi(\fr{n}{6} - \fr1{144})^{\fr12}}}
	{2(n - \fr1{24})^{\fr12}}  + O \left(\fr{e^{\fr{\pi}{2}
	(\fr{n}{6} - \fr1{144})^{\fr12}}}{(n - \fr1{24})^{\fr12}}
	\right)\;.
\tag5.2
$$

In 1964, building on the work of G. N. Watson \c{37} and L. A.
Dragonette \c{12}, I proved \c{2} a theorem equivalent to the 
assertion that if
$$
	f(q) = \sum_{n=0}^{\infty} \alpha(n) q^n\,,
\tag5.3
$$
then
$$
	\alpha(n) = \sum_{k=1}^{\lfloor n^{\fr12}\rfloor}
	\fr{(-1)^{\lfloor\fr{k+1}{2}\rfloor} A_{2k}(n - k(1 + (-1)^k)/4)
	\sinh \left(\fr{\pi}{k} \left(\fr{n}{6} - \fr1{144}\right)^{\fr12}
	\right)}{k^{\fr12} \left(n - \fr1{24}\right)^{\fr12}}
	+ O(n^\epsilon)
\tag5.4
$$
\c{2; p. 455, Th. 5.1}.  The $A_k(n)$ is the exponential sum that 
appears in the formula for $p(n)$ \c{4; p. 70}.

However, earlier L. A. Dragonette \c{12; p. 494}, \c{2, p. 456} had
suggested that in fact the actual error might indeed be smaller 
than $\fr12$ in absolute value.  If so, then the main sum rounded
to the nearest integer would give $\alpha(n)$ exactly.  Dragonette
supported this possibility with calculations at $n = 100$ and 
$200$.

Indeed it is possible that the main sum \underbar{converges} when
extended to infinity and, if so, may actually equal $\alpha(n)$
(\c{2; p. 456}).  The following table provides further evidence:

\vskip .1in
\centerline{\begintable
\begintableformat
\center " \center " \center " \center 
\endtableformat
\br{\:} $n$ " $\alpha(n)$ " {the main sum in (5.4)} " {the main sum in (5.4)} \er{}
\br{\:} {} " {} " {} " {extended to $n$ terms} \er{} 
\br{\:} {} " {} " {} " {} \er{} 
\br{\:} {100} " {-18520} " {-18520.18} " {-18520.01} \er{} 
\br{\:} {} " {} " {} " {} \er{} 
\br{\:} {200} " {-2660008} " {-2660008.01} " {-2660008.05} \er{} 
\br{\:} {} " {} " {} " {} \er{} 
\br{\:} {300} " {-128045286} " {-128045285.83} " {-128045285.99} \er{} 
\br{\:} {} " {} " {} " {} \er{} 
\br{\:} {400} " {-3447212602} " {-3447212601.86} " {-34472122602.05} \er{} 
\br{\:} {} " {} " {} " {} \er{} 
\br{\:} {500} " {063676485905} " {-63676485905.01} " {-63676485905.06} \er{} 
\br{\:} {} " {} " {} " {} \er{} 
\br{\:} {600} " {-897840541970} " {-897840541969.91} " {-897840541969.98} \er{}\br{\:} {} " {} " {} " {} \er{} 
\br{\:} {700} " {-10302538222405} " {-10302538222405.14} " {-10302538222405.02} \er{} 
\br{\:} {} " {} " {} " {} \er{} 
\br{\:} {800} " {-100343014357869} " {-100343014357869.11} " {-100343014357869.01} \er{} 
\br{\:} {} " {} " {} " {} \er{} 
\br{\:} {900} " {-854282301584078.03} " {-854282301584078.03} " {-854282301584078.02} \er{} 
\br{\:} {} " {} " {} " {} \er{} 
\br{\:} {1000} " {-6495836381771105} " {-649583638177105.004} " {-649583638177105.105} \er{}  
\endtable}
\vskip .1in

While three of the errors of $\sqrt{n}$ terms are smaller than the 
corresponding errors for $n$ terms, it is nonetheless the case that
the average error in column 2 is $0.0884$ and in column 4 it is
$0.0355$ which offers some support to our conjecture that the 
series in (5.4) actually converges to $\alpha(n)$.

Indeed, in order to prove convergence of the series, it would 
suffice to prove that
$$
	\sum_{k=1}^{\infty} \fr{(-1)^{\lfloor\fr{k+1}{2}\rfloor}
	A_{2k}(n - k(1 + (-1)^k)/4)}{k^{\fr32}}
$$
converges.  This in turn asks for an analysis of Dirichlet series
such as 
$$
	A(n,s) = \sum_{k=1}^{\infty} \fr{(-1)^{\lfloor\fr{k+1}{2}\rfloor}
	A_{2k}(n - k(1 + (-1)^k)/4)}{k^{s}}
\tag5.6
$$
which can easily be shown to converge absolutely for $\text{Re }s
> \fr32$.

\subhead
6. \ Modular Transformations and $q$-Hypergeometric Series
\endsubhead

Let us state an example of the type of problem that concerns us
here.

{\bf Problem.}  Prove directly that if
$$
	P(q) = 1 + \sum_{n=1}^{\infty} \fr{q^{n^2}}{(1 - q)^2
	(1 - q^2)^2 \cdots (1 - q^n)^2}\;,
\tag6.1
$$
then 
$$
	P(e^{2\pi i\tau}) = e^{\fr{-\pi i}{4}} \tau^{\fr12}
	q^{\fr{\pi i}{12}(\tau + \fr1{\tau})} 
	P(e^{-\fr{2\pi i}{\tau}})\,.
\tag6.2
$$

The catch in this problem is the single word ``directly.''  Otherwise
the problem is merely a reformulation of the transformation of
Dedekind's $\eta$-function because \c{4; p. 70}
$$
	P(q) = \fr{q^{\fr1{24}}}{\eta(\tau)} = \prod_{n=1}^{\infty}
	\fr1{1 - q^n}
\tag6.3
$$
where $q = e^{2\pi i \tau}$.

None of the methods which have been used to prove (5.3) seems to
adapt to a direct treatment of the series in (6.1).

The only work I know of on this topic is by L. Ehrenpreis.  His
observations are presented in \c{13} see especially Section 3 thereof.
Of his work, he says: ``This method sheds light on the question of 
why the generating function for Rogers-Ramanujan is an automorphic
function and why automorphicity is difficult to prove.''


\subhead
7. \ Conclusion
\endsubhead

The four topics chosen for this paper are tightly related.  The
problems presented do not fall naturally into the home turf of either
modular forms or $q$-series.  Consequently they have received little
attention.  However, they are calling out for an attack by a joint
effort of these two areas.

\Refs

\ref
  \no 1
  \by H. L. Alder
  \paper Partition identities \underbar{\ \ \ \ \ \ \ \ } from
	Euler to the present
  \jour Amer. Math. Monthly
  \vol 76
  \yr 1969
  \pages 733--746
\endref

\ref
  \no 2
  \by G. E. Andrews
  \paper On the theorems of Watson and Dragonette for Ramanujan's 
	mock theta functions
  \jour Amer. J. Math
  \vol 88
  \yr 1966
  \pages 454--490
\endref

\ref
  \no 3
  \by G. E. Andrews
  \paper Partition identities
  \jour Advances in Math.	
  \vol 9
  \yr 1972
  \pages 10--51
\endref

\ref
  \no 4
  \by G. E. Andrews
  \paper The Theory of Partitions
  \paperinfo Addison-Wesley, Reading 1976; reprinted, Cambridge
	University Press, Cambridge, 1984, 1998
\endref

\ref
  \no 5
  \by G. E. Andrews
  \paper Mock theta functions
  \jour Proc. Symp. in Pure Math
  \vol 49
  \yr 1989
  \pages 283--298
\endref

\ref
  \no 6
  \by J. Arkin
  \paper Researches on partitions
  \jour Duke Math. J.
  \vol 38
  \yr 1970
  \pages 403--409
\endref

\ref
  \no 7
  \by A. O. L. Atkin
  \paper Proof of a conjecture of Ramanujan
  \jour Glasgow Math. J.
  \vol 8
  \yr 1967
  \pages 14--32
\endref

\ref
  \no 8
  \by P. Bateman and P. Erd\"os
  \paper Monotonicity of partition functions
  \jour Mathematika
  \vol 3
  \yr 1956
  \pages 1--14
\endref

\ref
  \no 9
  \by R. J. Baxter
  \paper A direct proof of Kim's identities
  \jour J. Phys. A: Math. Gen.
  \vol 31
  \yr 1998
  \pages 1105--1108
\endref

\ref
  \no 10
  \by A. Cayley
  \paper Researches on the partition of numbers
  \jour Phil. Trans. Royal Soc.
  \vol 146
  \yr 1856
  \pages 127--140
  \finalinfo (Reprinted:  Coll. Math. Papers, 2 (1889), 235--249)
\endref

\ref
  \no 11
  \by A. DeMorgan
  \paper On a new form of difference equation
  \jour Cambridge Math. J.
  \vol 4
  \yr 1843
  \pages 87--90
\endref

\ref
  \no 12
  \by L Dragonette
  \paper Some asymptotic formulae for the  mock theta series of
	Ramanujan
  \jour Trans. Amer. Math. Soc.
  \vol 72
  \yr 1952
  \pages 474--500
\endref

\ref
  \no 13
  \by L. Ehrenpreis
  \paper Function theory for Rogers-Ramanujan-like partition identities
  \jour Contemporary Math
  \vol 143
  \yr 1993
  \pages 259--320
\endref

\ref
  \no 14
  \by J. W.L. Glaisher
  \paper On the number of partitions of a number of partitions of a
	number into a given number of parts
  \jour Quart. J. Pure and Appl. Math.
  \vol 40
  \yr 1908
  \pages 57--143
\endref

\ref
  \no 15
  \by K. Gl\"osel
  \paper \"Uber die Zerlegung der ganzen zahlen
  \jour Monatschefte Math. Phys.
  \vol 7
  \yr 1896
  \pages 133--141
\endref

\ref
  \no 16
  \by H. Gupta, E. E. Gwyther and J. C. P. Miller
  \paper Tables of Partitions
  \paperinfo Royal Soc. Math. Tables, Vol. 4, Cambridge University
	Press, Cambridge, 1958
\endref

\ref
  \no 17
  \by G. H. Hardy and S. Ramanujan
  \paper Asymptotic formulae in combinatory analysis
  \jour Proc. London Math. Soc., (2)
  \vol 17
  \yr 1918
  \pages 75--115
\endref

\ref
  \no 18 
  \by J. F. W. Herschel
  \paper On circulating functions and on the integration of a class
	of equations of finite differences into which they enter as
	coefficients
  \jour Phil. Trans. Royal Soc. London
  \vol 108
  \yr 1818
  \pages 144--168
\endref

\ref
  \no 19
  \by A. E. Ingham
  \paper A Tauberian theorem for partitions
  \jour Annals of Math
  \vol 42
  \yr 1941
  \pages 1075--1090
\endref

\ref
  \no 20
  \by D. Kim
  \paper Asymmetric XXZ chain at the antiferromagnetic transition:
	spectra and partition functions
  \jour J. Phys. A: Math. Gen.
  \vol 30
  \yr 1996
  \pages 3817--3836
\endref

\ref
  \no 21
  \by R. McIntosh
  \paper Some asymptotic formulae for $q$-hypergeometric series
  \jour J. London Math. Soc. (2)
  \vol 51
  \yr 1995
  \pages 120--136
\endref

\ref
  \no 22
  \by P. A. MacMahon
  \paper Combinatory Analysis
  \paperinfo Vol. 2, Cambridge University Press, London, 1916;
	reprinted Chelsea, New York, 1960
\endref

\ref
  \no 23
  \by E. Netto
  \paper Lehrbuch der Combinatorik
  \paperinfo 2nd ed. Teubner, Berlin, 1927; reprinted:  Chelsea,
	New York, 1958
\endref

\ref
  \no 24
  \by J. L. Nicolas and A. S\'ark\"ozy
  \paper On the asymptotic behavior of general partition functions
  \jour Ramanujan Journal
  \vol 4
  \yr 2000
  \pages 29--39
\endref

\ref
  \no 25
  \by K. Ono
  \paper On the parity of the partition function in arithmetic 
	progressions
  \jour J. f\"ur die r. und a. Math.
  \vol 472
  \yr 1996
  \pages 1--16
\endref

\ref
  \no 26
  \by K. Ono
  \paper Distribution of the partition function modulo $m$
  \jour Annals of Math.
  \vol 151 
  \yr 2000
  \pages 293--307
\endref

\ref
  \no 27
  \by P. Paoli
  \paper Opuscula analytica
  \paperinfo Liburni, 1780, Opusc. II (Meditations Arith.), $\S$1
\endref

\ref
  \no 28
  \by H. Rademacher
  \paper Lectures on Elementary Number Theory
  \paperinfo Blaisdell, New York, 1964
\endref

\ref
  \no 29
  \by H. Rademacher
  \paper Topics in Analytic Number Theory
  \paperinfo Springer, New York, 1973
\endref

\ref
  \no 30
  \by S Ramanujan
  \paper Collected Papers
  \paperinfo Cambridge University Press, London, 1927; reprinted:
	A. M. S. Chelsea, 2000 with new preface by B. Berndt
\endref

\ref
  \no 31
  \by S. Ramanujan
  \paper The Lost Notebook and Other Unpublished Papers
  \paperinfo  Narosa, New Delhi, 1988
\endref

\ref
  \no 32
  \by L. B. Richmond
  \paper A general asymptotic result for partitions
  \jour Canadian J. Math.
  \vol 27
  \yr 1975
  \pages 1083--1091
\endref

\ref
  \no 33
  \by K. F. Roth and G. Szekeres
  \paper Some asymptotic formulae in the theory of partitions
  \jour Quant. J. Math., Oxford Series (2)
  \vol 5 
  \yr 1954
  \pages 241--259
\endref

\ref
  \no 34
  \by I. Schur
  \paper Ein Beitrag zur additiven Zahlentheorie und zur Theorie der 
	Kettenbr\"uche
  \paperinfo S.-B. Preuss. Akad. Wiss., Phys.-Math. Kl., 1926
	488--495
\endref

\ref
  \no 35
  \by J. J. Sylvester
  \paper On subinvariants, i.e. semi-invariants to binary quantics
	of an unlimited order.  With an excursus on rational 
	fractions and partitions
  \jour Amer. J. Math.
  \vol 5 
  \yr 1882
  \pages 79--136
\endref

\ref
  \no 36
  \by G. Szekeres
  \paper An asymptotic formula in the theory of partitions, II
  \jour Quart. J. Math. Oxford Series 2
  \vol 2
  \yr 1951
  \pages 85--108
\endref

\ref
  \no 37
  \by G. N. Watson
  \paper The final problem:  an account of the mock theta functions
  \jour J. London Math. Soc.
  \vol 11
  \yr 1936
  \pages 55--80
\endref

\ref
  \no 38
  \by G. N. Watson
  \paper Ramanujans Vermutung \"uber Zerf\"alllungsanzahlen
  \jour J. reine und angew. Math.
  \vol 179
  \yr 1938
  \pages 97--128
\endref

\ref
  \no 39
  \by E. M. Wright
  \paper Asymptotic partition formulae, I. plane partitions
  \jour Quart. J. Math., Oxford Series
  \vol 2 
  \yr 1931
  \pages 177--189
\endref

\ref
  \no 40
  \by E. M. Wright
  \paper Asymptotic partition formulae, II.  Weighted partitions
  \jour Proc. London Math. Soc. (2)
  \vol 36
  \yr 1932
  \pages 117--141
\endref

\ref
  \no 41
  \by E. M. Wright
  \paper Asymptotic partitions formulae, III. Partitions into $k$-th powers
  \jour Acta Math.
  \vol 63
  \yr 1934
  \pages 143--191
\endref

\endRefs

\enddocument











