\documentclass[reqno]{amsart}

\usepackage{amssymb}
\usepackage{alltt}

%\numberwithin{equation}{section}

\allowdisplaybreaks[1]

%%% !!!!!!!!!!!!!!!!!!!!!!!!!!!!!!!!!!!!!!!!! %%%
%%% We change the \tt font from cmtt to cmtex %%%
%%% !!!!!!!!!!!!!!!!!!!!!!!!!!!!!!!!!!!!!!!!! %%%
\DeclareFontFamily{OT1}{cmtex}{\hyphenchar\font=-1}
\DeclareFontShape{OT1}{cmtex}{m}{n}
     {%
      <5><6><7><8>cmtex8<9>cmtex9%
      <10><10.95><12><14.4><17.28><20.74><24.88>cmtex10%
      }{}
\renewcommand{\ttdefault}{cmtex}

\newcommand{\lam}{\lambda}

\newcommand{\lbr}{\symbol{123}}
\newcommand{\rbr}{\symbol{125}}
\newcommand{\ttge}{\,\symbol{29}\,}
\newcommand{\ttlam}{\symbol{8}}
\newcommand{\mysub}[1]{\(\sb{\mathtt{#1}}\)}
\newcommand{\mysuper}[1]{\textsuperscript{#1}}

\newlength{\axellength}
\newcommand{\In}[1]{\settowidth{\axellength}{\texttt{123456#1}}%
\makebox[\axellength][r]{\footnotesize\textsf{In[#1]:=}}}
\newcommand{\Out}[1]{\settowidth{\axellength}{\texttt{123456#1}}%
\makebox[\axellength][r]{\footnotesize\textsf{Out[#1]=}}}

\DeclareMathOperator*{\Omegaoper}{\Omega}
\newcommand{\Omegaop}{\Omegaoper_{\scriptscriptstyle \geqq}}

\theoremstyle{plain}
\newtheorem{theorem}{Theorem}
\newtheorem{corollary}{Corollary}
\newtheorem{proposition}{Proposition}
\newtheorem{lemma}{Lemma}
\theoremstyle{definition}
\newtheorem{definition}{Definition}
\newtheorem{problem}{Problem}
\theoremstyle{remark}
\newtheorem*{remark}{Remark}

%%%%%%%%%%%%%%%%%%%%%%%%%%%%%%%%%%%%%%%%%%%%%%%%%%%%%%%%%%%%%%%%%%%%%%%%%%
\begin{document}

\title[MacMahon's Partition Analysis IX: $k$-Gon Partitions]
      {MacMahon's Partition Analysis IX:\\
       $\boldsymbol{k}$-Gon Partitions}

\author{George E.~Andrews}
\address{Department of Mathematics\\
         The Pennsylvania State University\\
         University Park, PA~16802, USA}
\email{andrews@math.psu.edu}
\thanks{The first author was partially supported by NSF Grant DMS-9206993}

\author{Peter Paule}
\address{Research Institute for Symbolic Computation\\
         Johannes Kepler University Linz\\
         A--4040~Linz, Austria}
\email{Peter.Paule@risc.uni-linz.ac.at}

\author{Axel Riese}
\address{Research Institute for Symbolic Computation\\
         Johannes Kepler University Linz\\
         A--4040~Linz, Austria}
\email{Axel.Riese@risc.uni-linz.ac.at}
\thanks{The third author was supported by SFB Grant F1305 of the Austrian FWF}

\date{February 22, 2001}
\dedicatory{Dedicated to George Szekeres on the occasion of his 90th birthday}

\begin{abstract}
MacMahon devoted a significant portion of Volume II of
his famous book ``Combinatory Analysis" to the introduction
of Partition Analysis as a computational method for solving combinatorial
problems in connection with systems of linear diophantine inequalities
and equations. In a series of papers we have shown that MacMahon's
method turns into an extremely powerful tool when implemented in computer
algebra. In this note we explain how the use of the package \texttt{Omega}
developed by the authors has led to a generalization of a classical
counting problem related to triangles with sides of integer length.
\end{abstract}

\maketitle


\section{Introduction}
%%%%%%%%%%%%%%%%%%%%%%
\label{sec:intro}

In his famous book ``Combinatory Analysis"~%
\cite[Vol.~II, Sect.~VIII, pp.~91--170]{MacMahon:CA}
MacMahon introduced Partition
Analysis as a computational method for solving combinatorial problems in
connection with systems of linear diophantine inequalities and equations.

We will use MacMahon's method and the \texttt{Omega} package
for a study of a classical combinatorial problem related to
triangles with sides of integer size. We start out by stating
the well-known base case which has been discussed
at various places; see e.g.~\cite{Liu,Jordan et al 1,GEA:77,
Jordan et al 2,Honsberger},
and \cite[Ch.~4, Ex.~16]{Stanley:BOOK}.

\begin{problem} \label{prob:triangle}
Let $t_3(n)$ be the number of non-congruent triangles whose sides
have integer length and whose perimeter is $n$. For instance,
$t_3(9)=3$, corresponding to $3+3+3$, $2+3+4$, $1+4+4$. Find
$\sum_{n\ge 3} t_3(n) q^n$.
\end{problem}

Obviously the corresponding generating function is
\begin{equation}
\label{restricted sum}
T_3(q):=\sum_{n\ge 3} t_3(n) \, q^n =
\sum\nolimits^* q^{a_1+a_2+a_3}
\end{equation}
where $\sum^*$ is the restricted summation over
all positive integer triples
$(a_1,a_2,a_3)$ satisfying $a_1 \leq a_2 \leq a_3$
and  $a_1+a_2 > a_3$.

\medskip
In order to see how Partition Analysis can be used
to compute a closed form representation for
$\sum_{n\ge 3} t_3(n) q^n$,
we need to recall
the key ingredient of MacMahon's method, the Omega
operator $\Omegaop$.

\begin{definition} \label{dfn:Omega def}
The operator $\Omegaop$ is given by
\[
\Omegaop \sum_{s_1=-\infty}^{\infty} \cdots \sum_{s_r=-\infty}^{\infty}
A_{s_1,\dots,s_r} \lambda_1^{s_1} \cdots \lambda_r^{s_r} :=
\sum_{s_1=0}^{\infty} \cdots \sum_{s_r=0}^{\infty} A_{s_1,\dots,s_r},
\]
where the domain of the $A_{s_1,\dots,s_r}$ is the field of
rational functions over $\mathbb{C}$ in several complex variables
and the $\lam_i$ are restricted to a neighborhood of the circle
$|\lam_i| = 1$. In addition, the $A_{s_1,\dots,s_r}$ are required
to be such that any of the series involved is absolute convergent
within the domain of the definition of $A_{s_1,\dots,s_r}$.
\end{definition}

We emphasize that it is essential to treat everything analytically rather
than formally because the method relies on unique Laurent series
representations of rational functions.

Another fundamental aspect of Partition Analysis is the
use of elimination rules which describe the action of the Omega
operator on certain base cases. MacMahon begins the discussion
of his method by presenting
a catalog~\cite[Vol.~II, pp.~102--106]{MacMahon:CA} of fundamental
evaluations. Subsequently he extends this table by new rules whenever
he is enforced to do so. Once found, most of these fundamental rules
are easy to prove. This is illustrated by the following examples which
are taken from MacMahon~\cite[Vol.~II, Art.~354, p.~106]{MacMahon:CA}.

\begin{proposition} \label{prop:1}
For integer $s\ge 0$ and variables $A, B$ being free of $\lam$,
\begin{align} \label{ruleI}
\Omegaop \frac{\lambda^{-s}}{(1-\lambda A) \big(1-\frac{B}{\lambda}\big)} & =
  \frac{A^{s}}{ (1-A) (1-A B)}; \\
\label{ruleII}
\Omegaop \frac{\lambda^s}{ (1-\lambda A) \big(1-\frac{B}{\lambda} \big)} & =
  \frac{ 1-A B - B^{s+1}+A B^{s+1}}
  { (1-A)(1-B)(1-A B) }.
\end{align}
\end{proposition}

\begin{proof} Rule~\eqref{ruleII}
is a special case of the more general rule~\eqref{red};
see Lemma~\ref{lem:2}. Rule~\eqref{ruleI} is proved as follows.
By geometric series expansion the left-hand side equals
\[
\Omegaop \sum_{i,j \ge 0} \lam^{i-j-s} A^i B^j =
\Omegaop \sum_{j,k \ge 0} \lam^k A^{k+j+s} B^j,
\]
where the summation parameter $i$ has then been replaced by $k+j+s$.
But now $\Omegaop$ sets $\lam$ to $1$, which completes the proof.
\end{proof}

Now we are ready for deriving the closed form
expression for $T_3(q)$
with Partition Analysis.

First, in order to get rid of the
diophantine constraints, one rewrites the restricted sum
expression in \eqref{restricted sum} into what MacMahon has
called the ``crude form'' of the generating function,
\begin{align*}
T_3(q) & =  \Omegaop \sum_{a_1 \ge 1,~a_2,a_3 \ge 0}
\lam_1^{a_2-a_1} \lam_2^{a_3-a_2} \lam_3^{a_1+a_2-a_3-1}
q^{a_1+a_2+a_3} \\
& = \Omegaop \frac{q \lambda_1^{-1}}
{\big(1- q \frac{\lam_3}{\lam_1} \big)
\big(1-q \frac{\lam_1\lam_3}{\lam_2} \big)
\big(1-q \frac{\lam_2}{\lam_3} \big)},
\end{align*}
where the last line is by geometric series summation.

Next by applying again rule~\eqref{ruleI} we eliminate successively
$\lam_2$, $\lam_1$, and $\lam_3$,
\begin{align}
\label{cf 3}
T_3(q) & = \Omegaop \frac{q \lam_1^{-1}}
{\big(1- q \frac{\lam_3}{\lam_1} \big)
\big(1-\frac{q}{\lam_3} \big)
(1-q^2 \lam_1 )}\nonumber  \\
& = \Omegaop \frac{q^3}{(1- q^2)
(1- q^3 \lam_3) \big(1-\frac{q}{\lam_3} \big)}\nonumber \\
& = \frac{q^{3}}{(1- q^2) (1- q^3) (1- q^4)}.
\end{align}
This completes the generating function computation and
Problem~\ref{prob:triangle} is solved.

\medskip
With our package \texttt{Omega}\footnote{available at
\textit{http://www.risc.uni-linz.ac.at/research/combinat/risc/software/Omega}}
the whole computation
can be done automatically and in one stroke. Note that
setting-up the crude generating function is done also
by the package:
\begin{alltt}
\In{1} <<Omega2.m
\end{alltt}
\begin{alltt}
\Out{1} Axel Riese's Omega implementation version 2.33 loaded
\end{alltt}
\begin{alltt}
\In{2} OSum[q\mysuper{a\mysub{1}+a\mysub{2}+a\mysub{3}}, {\lbr}a\mysub{2}{\ttge}
a\mysub{1}, a\mysub{3}{\ttge}a\mysub{2}, a\mysub{1}+a\mysub{2}\,>\,a\mysub{3}, a
\mysub{1}{\ttge}1\rbr, \ttlam]
\end{alltt}
\begin{alltt}
        Assuming a\mysub{2}{\ttge}0
        Assuming a\mysub{3}{\ttge}0
\end{alltt}
\Out{2}
\[
  \underset{\lam_1,\lam_2,\lam_3}{\Omegaop}
  \frac{q}{\lambda_1 \big(1-\frac{q\,\lam_2}{\lam_3} \big) \big(1-\frac{q\,\lam_
3}{\lam_1} \big)
  \big(1-\frac{q\,\lam_1\,\lam_3}{\lam_2} \big)}
\]
\begin{alltt}
\In{3} OR[%]
\end{alltt}
\begin{alltt}
        Eliminating \ttlam\mysub{3}...
        Eliminating \ttlam\mysub{2}...
        Eliminating \ttlam\mysub{1}...
\end{alltt}
\Out{4}
\[
  \frac{q^{3}}{(1- q^2) (1- q^3) (1- q^4)}
\]

\medskip
As already pointed out in~\cite{Andrews:MMPA2}, with Partition
Analysis one is able to derive much more information. Namely,
we can consider the full generating function
\[
S_3(x_1,x_2,x_3):=
\sum\nolimits^*  x_1^{a_1} x_2^{a_2} x_3^{a_3},
\]
where $\sum^*$ denotes again the restricted summation over
all positive integer triples
$(a_1,a_2,a_3)$ satisfying $a_1 \leq a_2 \leq a_3$
and  $a_1+a_2 > a_3$. On this expression we can
carry out essentially the same Partition Analysis
steps as above for obtaining a closed form expression for it.
For the crude form one gets
\begin{align*}
S_3(x_1,x_2,x_3) & =  \Omegaop \sum_{a_1 \ge 1,~a_2,a_3 \ge 0}
\lam_1^{a_2-a_1} \lam_2^{a_3-a_2} \lam_3^{a_1+a_2-a_3-1}
x_1^{a_1} x_2^{a_2} x_3^{a_3} \\
& = \Omegaop \frac{x_1 \lambda_1^{-1}}
{\big(1- x_1 \frac{\lam_3}{\lam_1} \big)
\big(1-x_2 \frac{\lam_1\lam_3}{\lam_2} \big)
\big(1-x_3 \frac{\lam_2}{\lam_3} \big)}.
\end{align*}

Next by applying again rule~\eqref{ruleI},
we eliminate successively $\lam_2$, $\lam_1$, and $\lam_3$
as above and obtain
\begin{align}
\label{parameter form}
S_3(x_1,x_2,x_3) & = \Omegaop \frac{x_1 \lam_1^{-1}}
{\big(1- x_1 \frac{\lam_3}{\lam_1} \big)
\big(1-\frac{x_3}{\lam_3} \big)
(1-x_2 x_3 \lam_1)} \nonumber \\
& = \Omegaop \frac{x_1 x_2 x_3}{(1- x_2 x_3)
(1- x_1 x_2 x_3 \lam_3)
\big(1-\frac{x_3}{\lam_3} \big)} \nonumber \\
& =  \frac{x_1 x_2 x_3}{(1- x_2 x_3)
(1- x_1 x_2 x_3)
(1- x_1 x_2 x_3^2)}.
\end{align}

This not only generalizes the generating function $T_3(q)$,
i.e.\ $T_3(q)=S_3(q,q,q)$, but gives rise also
to a complete, parameterized solution of the underlying
diophantine set of equations
\[
1 \leq a_1,\, a_1 \leq a_2 ,\, a_2 \leq a_3,
\text{ and } a_1+a_2 > a_3.
\]
This can be seen by geometric series expansion of
\eqref{parameter form}, namely
\[
S_3(x_1,x_2,x_3)=\sum_{n_1,n_2,n_3\ge 0}
x_1^{n_2+n_3+1} x_2^{n_1+n_2+n_3+1} x_3^{n_1+n_2+2 n_3+1}.
\]
In other words, by choosing
\[
a_1 = n_2+n_3+1, a_2 = n_1+n_2+n_3+1,
\text{ and } a_3=n_1+n_2+2 n_3+1,
\]
and running through all non-negative integers $n_1,n_2,n_3$,
one constructs in a one-to-one fashion \textit{all} non-degenerate
triangles with sides of integer size.

\medskip
In~\cite{APR:Omega3} we considered the following generalization
of the triangle problem to $k$-gons where $k \ge 3$.

\begin{definition} As the set of \textit{non-degenerate $k$-gon partitions
into positive parts} we define
\[
\tau_k:=\{(a_1,\dots,a_k)\in \mathbb{Z}^k \mid
1\leq a_1 \leq a_2 \leq \dots \leq a_k \text{ and }
a_1+\dots +a_{k-1}> a_k \}.
\]
As the set of \textit{non-degenerate $k$-gon partitions of $n$ into positive
parts} we define
\[
\tau_k(n):=\{(a_1,\dots,a_k)\in \tau_k \mid
a_1+\dots + a_{k}=n \}.
\]
The corresponding cardinality is denoted by
\[
t_k(n):=| \tau_k(n) |.
\]
\end{definition}

The term ``non-degenerate'' refers to the restriction to strict
inequality, i.e.\ to $a_1+\dots +a_{k-1}> a_k$. In the form
of~\eqref{cf 3} we computed a rational expression
for $T_3(q)=\sum_{n\ge 3} t_3(n) q^n$.
With the \texttt{Omega} package in hand, we are able to compute
also the next cases in a purely mechanical manner. For instance,
\begin{align} \label{T4}
\sum_{n\ge 4} t_4(n) \, q^n & =
\frac{q^4 (1+q+q^5)}{(1-q^2)(1-q^3)(1-q^4)(1-q^6)},\\
\label{T5}
\sum_{n\ge 5} t_5(n) \, q^n & =
\frac{q^5(1-q^{11})}{(1-q)(1-q^2)(1-q^4)(1-q^5)(1-q^6)(1-q^8)},\\
\intertext{and}
\label{T6}
\sum_{n\ge 6} t_6(n) \, q^n &=
\frac{q^6(1-q^4+q^5+q^7-q^8-q^{13})}
{(1-q)(1-q^2)(1-q^3)(1-q^4)(1-q^6)(1-q^8)(1-q^{10})}.
\end{align}

\medskip
{}From these results we were able to derive a number
of partition theoretical consequences.
However, despite the fact that the particular instances of
$\sum_{n \ge k} t_k(n) q^n$ can be computed so
easily, we were not able to find a common underlying
pattern. So we stated as an open problem:

\begin{problem} \label{open}
In view of the generating function representations
\eqref{cf 3}, \eqref{T4}, \eqref{T5}, and \eqref{T6}: Is it
possible to find a common pattern for all possible
choices of~$k$?
\end{problem}

In Section~\ref{sec:genfu} we provide an affirmative answer to this
problem. More precisely, we give closed form expressions
for $T_k(q)$ as well as for the
corresponding general version $S_k(x_1,\dots,x_k)$
defined as follows:

\begin{definition} For integer $k\ge 3$,
\[
T_k(q):= \sum_{n\ge k} t_k(n) \, q^n,
\]
and
\[
S_k(x_1,\dots,x_k):=
\sum_{(a_1,\dots,a_k)\in \tau_k} x_1^{a_1}\cdots x_k^{a_k}.
\]
\end{definition}

Finally, Section~\ref{sec:conclusion} provides some concluding remarks.


\section{Generating Functions for $k$-Gon Partitions} \label{sec:genfu}
%%%%%%%%%%%%%%%%%%%%%%%%%%%%%%%%%%%%%%%%%%%%%%%%%%%%%

In this section we prove the following main result for $k$-gon
partitions.

\begin{theorem} \label{thm:1}
   Let $k \ge 3$ and $X_i = x_i \cdots x_k$ for $1 \le i \le k$.
   Then
   \begin{equation} \begin{split} \label{main}
      & S_k(x_1,\dots,x_k) = \frac{X_1}{(1-X_1)(1-X_2)\cdots(1-X_k)} \\
      & \quad - \frac{X_1 X_k^{k-2}}{1-X_k} \;
      \frac{1}{(1-X_{k-1})(1-X_{k-2}X_k)(1-X_{k-3}X_k^2)
      \cdots (1-X_1 X_k^{k-2})}.
   \end{split}\end{equation}
\end{theorem}

Since $T_k(q)=S_k(q,\dots,q)$, Theorem~\ref{thm:1} implies the
desired generating function representation.

\begin{corollary} \label{cor:1}
   For $k \ge 3$,
   \begin{equation} \label{Tkq}
      T_k(q) = \frac{q^k}{(1-q)(1-q^2)\cdots(1-q^k)}
      - \frac{q^{2k-2}}{1-q} \;
      \frac{1}{(1-q^2)(1-q^4) \cdots (1-q^{2k-2})}.
   \end{equation}
\end{corollary}

\begin{remark}
   It is easily verified that \eqref{Tkq} is bringing the
   representations \eqref{cf 3}, \eqref{T4}, \eqref{T5}, and \eqref{T6}
   of the special cases $k=3,4,5,6$ under one umbrella.
\end{remark}

We shall prove Theorem~\ref{thm:1} with Partition Analysis.
To this end we first need the crude form of $S_k(x_1,\dots,x_k)$.

\begin{proposition} \label{prop:2}
   For $k \ge 3$,
   \begin{equation}\begin{split} \label{SkO}
      & S_k(x_1,\dots,x_k) \\
      & \quad =
      \Omegaop \frac{x_1 \lam_1^{-1}}
      {\big(1-x_1 \frac{\lam_k}{\lam_1}\big)\big(1-x_2 \frac{\lam_1\lam_k}{\lam_
2}\big)
      \big(1-x_3 \frac{\lam_2\lam_k}{\lam_3}\big) \cdots
      \big(1-x_{k-1} \frac{\lam_{k-2}\lam_k}{\lam_{k-1}}\big)
      \big(1-x_k \frac{\lam_{k-1}}{\lam_k}\big)}.
   \end{split}\end{equation}
\end{proposition}

\begin{proof}
   For fixed integer $k \ge 3$,
   \[
     S_k(x_1,\dots,x_k) = \Omegaop \sum_{\substack{a_1 \ge 1 \\ a_2,\dots,a_k \g
e 0}}
     x_1^{a_1} \cdots x_k^{a_k} \lam_1^{a_2-a_1} \cdots
     \lam_{k-1}^{a_k-a_{k-1}} \lam_k^{a_1+\dots+a_{k-1}-a_k-1}
   \]
   by the definition of $\Omegaop$. The rest follows by geometric series
   summation.
\end{proof}

The next step is the successive elimination of $\lam_1, \lam_2, \dots, \lam_{k-1
}$
from the crude form~\eqref{SkO}.
For this it is convenient to introduce a lemma.

\begin{lemma} \label{lem:1}
   Let $k \ge 3$ and let $y_1, \dots, y_k$ be free of
   $\lam_1, \dots, \lam_{k-1}$. Then
   \begin{align*}
      & \Omegaop \frac{y_1 \lam_1^{-1}}
      {\big(1-\frac{y_1}{\lam_1}\big)\big(1-y_2 \frac{\lam_1}{\lam_2}\big)
      \cdots \big(1-y_{k-1} \frac{\lam_{k-2}}{\lam_{k-1}}\big)
      (1-y_k \lam_{k-1})} \\
      & \quad = \frac{y_1 \cdots y_k}{(1-y_k)(1-y_{k-1}y_k)\cdots
      (1-y_1 \cdots y_k)}.
   \end{align*}
\end{lemma}

\begin{proof}
   We proceed by induction on $k$. For $k=3$,
   \begin{align*}
      & \Omegaop \frac{y_1 \lam_1^{-1}}
      {\big(1-\frac{y_1}{\lam_1}\big)\big(1-y_2 \frac{\lam_1}{\lam_2}\big)
      (1-y_3 \lam_2)} \\
      & \quad = \Omegaop \frac{y_1 \lam_1^{-1}}
      {\big(1-\frac{y_1}{\lam_1}\big) (1-y_3) (1-y_2 y_3 \lam_1)}
      \tag{by \eqref{ruleI} with $s=0$}\\
      & \quad = \frac{y_1 y_2 y_3} {(1-y_2 y_3) (1-y_3) (1-y_1 y_2 y_3)}
      \tag{by \eqref{ruleI} with $s=1$}.
   \end{align*}
   For the induction step we apply again rule~\eqref{ruleI} with $s=0$,
   \begin{align*}
      & \Omegaop \frac{y_1 \lam_1^{-1}}
      {\big(1-\frac{y_1}{\lam_1}\big)\big(1-y_2 \frac{\lam_1}{\lam_2}\big)
      \cdots \big(1-y_{k-1} \frac{\lam_{k-2}}{\lam_{k-1}}\big)
      \big(1-y_k \frac{\lam_{k-1}}{\lam_k}\big)
      (1-y_{k+1} \lam_k)} \\
      & \quad = \frac{1}{1-y_{k+1}} \; \Omegaop \frac{y_1 \lam_1^{-1}}
      {\big(1-\frac{y_1}{\lam_1}\big)\big(1-y_2 \frac{\lam_1}{\lam_2}\big)
      \cdots \big(1-y_{k-1} \frac{\lam_{k-2}}{\lam_{k-1}}\big)
      (1-y_k y_{k+1} \lam_{k-1})} \\
      & \quad = \frac{1}{1-y_{k+1}} \; \frac{y_1 \cdots y_{k+1}}
      {(1-y_k y_{k+1}) (1-y_{k-1} y_k y_{k+1}) \cdots (1-y_1 \cdots y_{k+1})};
   \end{align*}
   for the last line we used the induction hypothesis.
\end{proof}

Now we are in the position to state the crude form of $S_k(x_1,\dots,x_k)$.

\begin{proposition} \label{prop:3}
   Let $k \ge 3$ and $X_i=x_i \cdots x_k$ for $1 \le i \le k$. Then
   \begin{equation}\begin{split}\label{crude}
      & S_k(x_1,\dots,x_k) \\
      & \quad = \frac{X_1}{1-X_{k-1}} \; \Omegaop \frac{\lam_k^{k-3}}
      {\big(1-\frac{X_k}{\lam_k}\big)(1-X_{k-2} \lam_k)
      (1-X_{k-3} \lam_k^2)\cdots(1-X_1 \lam_k^{k-2})}.
   \end{split}\end{equation}
\end{proposition}

\begin{proof}
   By Proposition~\ref{prop:2},
   \[
     S_k(x_1,\dots,x_k) =
     \Omegaop \frac{y_1 \lam_1^{-1} \lam_k}
     {\big(1-\frac{y_1}{\lam_1}\big) \big(1-y_2 \frac{\lam_1}{\lam_2}\big)
     \cdots \big(1-y_{k-1} \frac{\lam_{k-2}}{\lam_{k-1}}\big)
     (1-y_k \lam_{k-1})},
   \]
   where $y_1=x_1 \lam_k, \dots, y_{k-1} = x_{k-1} \lam_k$ and
   $y_k=x_k/\lam_k$.
   By Lemma~\ref{lem:1} this is equal to
   \[
     \Omegaop \frac{x_1 \cdots x_k \lam_k^{k-3}}
     {\big(1-\frac{x_k}{\lam_k}\big) (1-x_{k-1} x_k)
     (1-x_{k-2} x_{k-1} x_k \lam_k) \cdots (1-x_1 \cdots x_k \lam_k^{k-2})}
   \]
   which is the right-hand side of~\eqref{crude}.
\end{proof}

In order to complete the proof of Theorem~\ref{thm:1} we need another
elementary lemma;
namely, the special case $m=1$, $k=1$, and $j_i=i$ of our reduction algorithm
described in~\cite{APR:Omega6}.
However, for the sake of better readability we state and prove it
explicitly.

\begin{lemma} \label{lem:2}
   Let $k \ge 1$, $a \ge 0$, and let $y, y_1, \dots, y_k$ be
   free of $\lam$. Then
   \begin{equation}\begin{split} \label{red}
      & \Omegaop \frac{\lam^a}{\big(1-\frac{y}{\lam}\big)
      (1-y_1 \lam)(1-y_2 \lam^2) \cdots (1-y_k \lam^k)} \\
      & \quad = \frac{1}{(1-y_1)\cdots(1-y_k)(1-y)} -
      \frac{y^{a+1}}{(1-y_1 y)(1-y_2 y^2) \cdots (1-y_k y^k)(1-y)}.
   \end{split}\end{equation}
\end{lemma}

\begin{remark}
   Formula~\eqref{ruleII} of Proposition~\ref{prop:1} is the special case $k=1$.
\end{remark}

\begin{proof}
   The left hand-side of~\eqref{red} equals
   \begin{align*}
      & \Omegaop \sum_{s_1,\dots,s_k \ge 0} \sum_{r \ge 0} y_1^{s_1}
      \cdots y_k^{s_k} y^r \lam^{1 \cdot s_1 + 2 \cdot s_2 + \dots +
      k \cdot s_k+a-r}\\
      & \quad = \sum_{s_1,\dots,s_k \ge 0} y_1^{s_1} \cdots y_k^{s_k}
      \sum_{r=0}^{1 \cdot s_1 + \dots + k \cdot s_k+a} y^r
   \end{align*}
   and the lemma follows by applying $\sum_{r=0}^m y^r =
   (1-y^{m+1})/(1-y)$.
\end{proof}

Finally we come to the proof of Theorem~\ref{thm:1}.

\begin{proof}[Proof of Theorem~\textup{\ref{thm:1}}]
   By Proposition~\ref{prop:3},
   \begin{align*}
      & S_k(x_1,\dots,x_k) \\
      & \quad = \frac{X_1}{1-X_{k-1}} \; \Omegaop \frac{\lam_k^{k-3}}
      {\big(1-\frac{X_k}{\lam_k}\big)(1-X_{k-2} \lam_k)
      (1-X_{k-3} \lam_k^2)\cdots(1-X_1 \lam_k^{k-2})}\\
      & \quad = \frac{X_1}{1-X_{k-1}} \; \bigg( \frac{1}{(1-X_1)(1-X_2)
      \cdots (1-X_{k-2})(1-X_k)}\\
      & \hspace*{18ex} - \frac{X_k^{k-2}}{(1-X_{k-2}X_k)
      (1-X_{k-3}X_k^2) \cdots (1-X_1 X_k^{k-2})(1-X_k)} \bigg),
   \end{align*}
   where the last equality is by Lemma~\ref{lem:2} with $a=k-3$
   and $y=X_k, y_1=X_{k-2}, y_2=X_{k-3},\dots,y_{k-2}=X_1$.
   This completes the proof of Theorem~\ref{thm:1}.
\end{proof}


\section{Conclusion} \label{sec:conclusion}
%%%%%%%%%%%%%%%%%%%%

As shown in a series of articles~\cite{APR:Omega3,APR:Omega6,APR:Omega7,APR:Omeg
a8,APR:Omega5},
Partition Analysis is ideally suited for being supplemented by
computer algebra methods. In these papers the \textsf{Mathematica} package
\texttt{Omega} which had been developed by the authors, was used as
an essential tool.

The \texttt{Omega} package played a crucial role also in
discovering Theorem~\ref{thm:1} above. However, it is important to
note that the computations \eqref{cf 3}, \eqref{T4}, \eqref{T5},
and \eqref{T6} for $T_k(q)$ with $k=3,4,5,6$ have not led us
to Theorem~\ref{thm:1}. Rather than this, the main point in the study
of $k$-gon partitions
was the careful \texttt{Omega} investigation of the full
generating function $S_k(x_1,\dots,x_k)$; only in this generality
the underlying pattern was finally revealed.

Another remark concerns the constructive use of Theorem~\ref{thm:1}.
As a matter of fact, formula~\eqref{main} can be used to
\textit{construct} $k$-gon partitions in the same way as
explained in the introduction
with the special case~\eqref{parameter form}.

In~\cite{Andrews:MMPA2} refinements of the base case $k=3$
of Theorem~\ref{thm:1} and Corollary~\ref{cor:1} have been considered. We expect
that experiments
with the \texttt{Omega} package will lead to more general
results in this direction.

\renewcommand{\ttdefault}{cmtt}

\begin{thebibliography}{99}

\bibitem{GEA:77} G.E.~Andrews, \textit{A note on partitions
   and triangles with integer sides}, Amer.\ Math.\ Monthly
   {\bf 86} (1979), 477--478.

\bibitem{Andrews:MMPA2}
   \bysame,
   \textit{MacMahon's partition analysis II:
   Fundamental theorems}, Ann.~Comb. {\bf 4} (2000).

\bibitem{APR:Omega3}
   G.E.~Andrews, P.~Paule and A.~Riese, \textit{MacMahon's
   partition analysis~III: The \texttt{Omega} package},
   SFB Report \textbf{99-24}, J.~Kepler University, Linz, 1999.
   (to appear)

\bibitem{APR:Omega6}
   \bysame, \textit{MacMahon's
   partition analysis~VI: A new reduction algorithm},
   SFB Report \textbf{01-4}, J.~Kepler University, Linz, 2001.
   (to appear)

\bibitem{APR:Omega7}
   \bysame, \textit{MacMahon's
   partition analysis~VII: Constrained compositions},
   SFB Report \textbf{01-5}, J.~Kepler University, Linz, 2001.
   (to appear)

\bibitem{APR:Omega8}
   \bysame, \textit{MacMahon's
   partition analysis~VIII: Plane Partition Diamonds},
   SFB Report \textbf{01-6}, J.~Kepler University, Linz, 2001.
   (to appear)

\bibitem{APR:Omega5}
   G.E.~Andrews, P.~Paule, A.~Riese and V.~Strehl, \textit{MacMahon's
   partition analysis~V: Bijections, recursions, and magic squares},
   SFB Report \textbf{00-18}, J.~Kepler University, Linz, 2000.
   (to appear)

\bibitem{Honsberger} R.~Honsberger, \textit{Mathematical Gems III},
   Math.~Assoc.\ of America, Washington, 1985.

\bibitem{Jordan et al 1} J.H.~Jordan, R.~Walsh and R.J.~Wisner,
   \textit{Triangles with integer sides}, Notices Amer.\ Math.\ Soc.\
   {\bf 24} (1977), A-450.

\bibitem{Jordan et al 2} J.H.~Jordan, R.~Walsh, and R.J.~Wisner,
   \textit{Triangles with integer sides}, Amer.\ Math.\ Monthly
   {\bf 86} (1979), 686--689.

\bibitem{Liu} C.I.~Liu, \textit{Introduction to Combinatorial
   Mathematics}, McGraw-Hill, New York, 1968.

\bibitem{MacMahon:CA} P.A.~MacMahon, \textit{Combinatory Analysis}, 2~vols.,
   Cambridge University Press, Cambridge, 1915--1916. (Reprinted: Chelsea,
   New York, 1960)

\bibitem{Stanley:BOOK} R.P.~Stanley, \textit{Enumerative Combinatorics --- Volum
e~\textup{1}},
   Wadsworth, Monterey, California, 1986.

\end{thebibliography}

\end{document}
