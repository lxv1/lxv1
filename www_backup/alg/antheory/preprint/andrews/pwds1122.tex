%
\magnification=\magstep1
\input amstex
\documentstyle{amsppt}
\NoBlackBoxes

\vsize=7.5in

\def\c{\cite}
\def\fr{\frac}
\def\sumty{\displaystyle{\sum_{n=0}^{\infty}}}
\def\pf{\hfill $\square$}
\def\N{\Bbb N}


\topmatter
\title
	Partitions with Designated Summands
\endtitle
\author
	G. E. Andrews, R. P. Lewis, J. Lovejoy
\endauthor
\abstract
This paper is devoted to the study of a new class of partitions,
partitions with designated summands.  In such partitions, among those
parts of the same magnitude one is tagged or designated.  The
resulting partition functions are related to several classical
ordinary partition functions.  Also they have many interesting
divisibility properties and satisfy quite appealing modular relations.
\endabstract
\endtopmatter

\document
\baselineskip 20pt
\subhead
1. \ Introduction
\endsubhead

The object of our study is partitions with designated summands.  They
are constructed by taking ordinary partitions and tagging exactly one
of each part size.  Thus there are 15 partitions of 5 with designated
summands:
$$
\gathered
	5',4'+1', 3'+2',3'+1'+1,3'+1+1', 2'+2+1',  \\
	2+2'+1', 2'+1'+ 1+1, 2'+1+1'+1,2'+1+1+1', 1'+1+1+1+1,  \\
	1+1'+1+1+1, 1+1+1'+1+1,1+1+1+1'+1,1+1+1+1+1'\,.
\endgathered
$$
The total number of partitions of $n$ with designated summands is
denoted by $PD(n)$.  Hence $PD(5) = 15$.

We shall also study $PDO(n)$ the total number of partitions of
$n$ with designated summands in which all parts are odd.  Thus
$PDO(5) = 8$.

While these objects have not appeared in the literature before,
they are tacitly considered by P. A. MacMahon \c{M2} in his work
on generalized divisor sums.  Indeed, MacMahon's $A_{n,k}$ is the
number of partitions of $n$ with designated summands wherein exactly
$k$ different magnitudes occur among all the parts.  MacMahon
\c{M2; Section 17} is able to connect $A_{n,k}$ with numerous
divisor sum identities due to  Glaisher \c{G}, Ramanujan \c{R}
and others.  However, MacMahon neglected the very interesting
case wherein the number of different magnitudes of parts is
surpressed.

We shall consider the generating functions:
$$
	pd(q) = \sumty PD(n) q^n\,,
\tag1.1
$$
and
$$
	pdo(q) = \sumty PDO(n) q^n\,.
\tag1.2
$$

In Section 2, we shall show (among other things) that
$$
	pd(q) = \fr{\eta(6\tau)}{\eta(\tau)\eta(2\tau)\eta(3\tau)}\;,
\tag1.3
$$
where
$$
	\eta(\tau) = q^{fr1{24}} \prod_{n=1}^{\infty} (1 - q^n)\,,
\tag1.4
$$
and
$$
	q = e^{2\pi i\tau}\,.
\tag1.5
$$
Also
$$
	pdo(q) = \fr{\eta(4\tau)\eta^2(6\tau)}{\eta(\tau)
	\eta(3\tau) \eta(12\tau)}\;.
\tag1.6
$$

We examine $q$-series identities in Section 3, and in Section 4 we
relate $PD(n)$ and $PDO(n)$ to several well-known partition functions.
As a result, we can easily deduce that $PDO(n)$ is even unless either
$n$ is $0$ or $n$ is a perfect square not divisible by 3.

Sections 5 and 6 are devoted to congruences for $PD(n)$ and modular
relations for $pd(q)$ while Sections 7 and 8 treat the same issues
for $PDO(n)$ and $pdo(q)$ respectively.

\subhead
2. \ The Generating Functions
\endsubhead

\proclaim
{Theorem 1} Suppose $S$ is a set of positive integers.  We denote by
$PD_S(n)$ the number of partitions of $n$ with designated summands all
of whose parts lie in $S$, and we denote by $pd_S(q)$ the generating
function for $PD_S(n)$.  Then
$$
	pd_S(q) = \prod_{n\in S} \fr{(1 - q^{6n})}{(1 - q^n)
	(1 - q^{2n})(1 - q^{3n})}\;.
\tag2.1
$$
\endproclaim

\demo
{Proof}  We see immediately that 
$$
\aligned
	pd_S(q) & = \prod_{n\in S} (1 + q^n + 2q^{2n} + 3q^{3n}
	+ 4q^{4n} + \cdots )   \\
	& = \prod_{n\in S} \left( 1 + \fr{q^n}{(1 - q^n)^2}\right)
	\\
	& = \prod_{n\in S}^{\infty} \fr{(1 - q^n + q^{2n})}{(1 - q^n)^2}
	\\
	& = \prod_{n\in S}^{\infty} \fr{(1 + q^{3n})}{(1 - q^n)^2(1 + q^n)}
	\\
	& = \prod_{n\in S}^{\infty} \fr{(1 - q^{6n})}{(1 - q^n)(1 - q^{2n})
	(1 - q^{3n})}\;.  
\endaligned
$$
\qed
\enddemo

\proclaim
{Corollary 1.1}  Equation (1.3) is valid.
\endproclaim

\demo
{Proof}  This is the case $S = \N$, the set of all positive 
integers.  \pf
\enddemo

\proclaim
{Corollary 1.2}  If $S_k$ is the set of all positive integers not 
divisible by $k$, then
$$
	pd_{S_k}(q) = pd(q)/pd(q^k)\;.
$$
\endproclaim

\demo
{Proof}  This assertion is the specialization of Theorem 1 to
$S_k$.  \pf
\enddemo

\proclaim
{Corollary 1.3}  Equation (1.6) is valid.
\endproclaim

\demo
{Proof}  This follows directly from Corollaries 1 and Corollary 2
with $k = 2$.  \pf
\enddemo

Indeed we might note that if $S$ consists of all positive integers
congruent to one of $a_1,a_2,\dots,a_j$ modulo $k$, and if $S$ is
symmetric (i.e. whenever $\alpha \in S$ then all positive integers
$\equiv - \alpha\pmod{k}$ are in $S$), then $pd_S(q)$ is a modular
form.  The subsequent sections are restricted to the instances 
of $pd_S(q)$ that seem to have the most interest, namely $pd(q)$
and $pdo(q)$.

\subhead
3. \ $q$-Series for $pd(q)$ and $pdo(q)$
\endsubhead

There are several reasons for considering $q$-series expansions of
these generating functions.  Often such expansions provide efficient
algorithms for the calculation of their coefficients.  This turns
out to be the case for both $pd(q)$ (see (3.4)) and $pdo(q)$ (see
(3.14)).  In addition, we occasionally uncover new mysteries.  In
our present study, it is quite mysterious that the $q$-series (3.3)
and (3.13) which seem to have complex coefficients, in fact do not.

\proclaim
{Theorem 2}  For $|q| < 1$ and $\zeta = e^{\pi i/3}$,
$$
\align
	pd(q)  & = \fr{(\zeta q;q)_{\infty}(\zeta^{-1} q;q)_{\infty}}
	{(q;q)_{\infty}^2}   \tag3.1   \\
	& = 1 + \sum_{n=1}^{\infty} \fr{(-q^3;q^3)_{n-1} q^n}
		{(q;q)_n(q^2;q^2)_{n-1} (1 - q^n)}  \tag3.2  \\
	& = 1 + \zeta \sum_{n=1}^{\infty} \left(\fr{(\zeta q;q)_{n-1}}
		{(q;q)_n}\right)^3 (1 - \zeta q^{2n})(-1)^n 
		q^{\binom{n+1}{2}} \zeta^{-n}  \tag3.3  \\
	& = \fr{\sumty q^{\binom{n+1}{2}} - 3 
		\sumty q^{\binom{3n+2}{2}}}
		{\sumty (-1)^n (2n+1) q^{\binom{n+1}{2}}}\;,
		\tag3.4
\endalign
$$
where
$$
	(a;q)_n = (1 - a)(1 - aq) \cdots (1 - aq^{n-1})\,.
\tag3.5
$$
\endproclaim

\demo
{Proof}  By (1.3),
$$
\aligned
	pd(q) & = \fr{(q^6;q^6)_{\infty}}{(q;q)_{\infty}
		(q^2;q^2)_{\infty} (q^3;q^3)_{\infty}}  \\
	& = \fr{(-q^3;q^3)_{\infty}}{(q;q)_{\infty}^2 (-q;q)_{\infty}}
	\\
	& = \fr{(\zeta q;q)_{\infty}(\zeta^{-1} q;q)_{\infty}}
		{(q;q)_{\infty}^2}\;,
\endaligned
$$
which proves (3.1).

Identity (3.2) may be rewritten as follows:
$$
	\sum_{n-0}^{\infty} \fr{(\zeta;q)_n(\zeta^{-1};q)_n}
		{(q;q)_{\infty}(q;q)_n} = \fr{(\zeta q;q)_{\infty}
		(\zeta^{-1} q;q)_{\infty}}{(q;q)_{\infty}(q;q)_{\infty}}\;,
\tag3.6
$$
and in this form we see that it is a specialization of the $q$-analog
of Gauss's theorem \c{A2; p. 20, Cor. 2.4, $a=\zeta$, $b = \zeta^{-1}$,
$c = q$}.

Identity (3.3) may be rewritten as follows:
$$
	\lim_{t\rightarrow 0} \sumty
	\fr{(\zeta;q)_n^3}{(q;q)_n^3} \fr{(1 - \zeta q^{2n})}
	{(1 - \zeta)} \fr{(t^{-1};q)_n}{(\zeta qt;q)_n} \left(
	\fr{t\zeta q}{\zeta^2}\right)^n = 
	\fr{(\zeta^{-1}q;q)_{\infty}(\zeta q;q)_{\infty}}
		{(q;q)_{\infty}(q;q)_{\infty}}\;,
\tag3.7
$$
and in this form we see that it is a specialization of the 
limiting form of Jackson's theorem \c{G-R; p. 238, eq. (II,20)}.

To obtain (3.4), we first note that
$$
	\fr{(-1)^n (\zeta^{-n}- \zeta^{n+1})}{(1 - \zeta)} =
	\cases
		1 \qquad & \text{ if } n \equiv 0 \pmod{3} \\
		-2 \qquad & \text{ if } n \equiv 1 \pmod{3} \\
		1 \qquad & \text{ if } n \equiv 2 \pmod{3}.
	\endcases
\tag3.8
$$
Hence by Jacobi's triple product \c{G-R; p. 239, eq. (II.28)}
$$
\aligned
	\fr{(\zeta q;q)_{\infty}(\zeta^{-1}q;q)_{\infty}}
	{(q;q)_{\infty}^2} & = \fr{(q;q)_{\infty}(\zeta;q)_{\infty}
	(\zeta^{-1}q;q)_{\infty}}{(1 - \zeta) \; (q;q)_{\infty}^3}
	\\
	& = \fr{\sumty (-1)^nq^{\binom{n+1}{2}}
	\zeta^{-n}/(1 - \zeta)}{(q;q)_{\infty}^3}
	\\
	& = \fr{\sumty q^{\binom{n+1}{2}}((-1)^n
	(\zeta^{-n} - \zeta^{n+1})/(1 - \zeta))}{\sumty
	(-1)^n (2n+1)q^{\binom{n+1}{2}}}
	\\
	& \hskip 1in  \text{(by \c{H-R; p. 285, Th. 357})}
	\\
	& = \fr{\sumty q^{\binom{n+1}{2}} - 3 
		\sumty q^{\binom{3n+2}{2}}}  
		{\sumty (-1)^n (2n+1) q^{\binom{n+1}{2}}}\;. 
\endaligned
\tag3.9
$$
\pf
\enddemo

From (3.4) we immediately deduce a computationally fast recurrence
for $PD(n)$.  

\proclaim
{Corollary 2.1}  For $n \geqq 0$
$$
	\sum_{j\geqq 0} PD\left(n - \binom{j+1}{2}\right) 
	(-1)^j (2j+1) = \cases
		0 \ & \text{\rm if } n \neq \text{\rm triangular
			number}  \\
		1 \ & \text{\rm if } n = \binom{m+1}{2} 
			\text{\rm and } m \not\equiv 1 \pmod{3}
		\\
		-2 \ & \text{\rm otherwise}
	\endcases
\tag3.10
$$
\endproclaim

\proclaim
{Corollary 2.2}  $PD(3n+2) \equiv 0 \pmod{3}$.
\endproclaim

\demo
{Proof}  By (3.4), we see that
$$
\aligned
	pd(q) & \equiv \fr{\sumty q^{\binom{n+1}{2}}}{(q;q)_{\infty}^3}
	\;\;\pmod{3}   \\
	& \equiv \fr{\sumty q^{\binom{n+1}{2}}}{(q^3;q^3)_{\infty}}
	\;,\;\pmod{3}   
\endaligned
$$
but this last expression has no powers of $q \equiv 2 \pmod{3}$
in its expansion.  \pf
\enddemo

\proclaim
{Theorem 3}  For $|q| < 1$ and $\zeta = e^{\pi i/3}$,
$$
\align
	pdo(q) & = \fr{(\zeta q;q^2)_{\infty}(\zeta^{-1}q;q^2)_{\infty}}
		{(q;q^2)_{\infty}^{2}}  \;   \tag3.11   \\
	& = 1 + \sum_{n=1}^{\infty} \fr{(-q^6;q^6)_{n-1} q^n}
		{(q;q)_{2n} (-q^2;q^2)_{n-1}}   \tag3.12   \\
	& = 1 - \zeta \sum_{n=1}^{\infty} \fr{(\zeta q;q)_{2n-1}
		(\zeta q^2;q^2)_{n-1} (-1)^n q^{n^2}}{(q;q)_{2n}
		(q;q^2)_n}   \tag3.13    \\
	& = \fr{\displaystyle{\sum_{n=-\infty}^{\infty}} q^{(3n)^2}
		- \sum_{n=-\infty}^{\infty} q^{(3n+1)^2}}
		{1 + 2 \sum_{n=1}^{\infty} (-1)^n q^{n^2}}\;.  \tag3.14
\endalign
$$
\endproclaim

\demo
{Proof}  By Corollary 1.2 with $k = 2$
$$
\align
	pdo(q) & = pd(q)/pd(q^2) \tag3.15   \\
	& = \fr{(\zeta q;q)_{\infty}(\zeta^{-1} q;q)_{\infty}}
		{(q;q)_{\infty}^2}\;\cdot\; \fr{(q^2;q^2)_{\infty}}
	{(\zeta q^2;q^2)_{\infty}(\zeta^{-1}q^2;q^2)_{\infty}}
	\\
	& = \fr{(\zeta q;q^2)_{\infty}(\zeta^{-1}q;q^2)_{\infty}}
	{(q;q^2)_{\infty}^2}\;.
\endalign
$$

Identity (3.12) may be rewritten as follows:
$$
	\sumty \fr{(\zeta;q^2)_n(\zeta^{-1};q^2)_n}{(q^2;q^2)_n
	(q;q^2)_n} \;q^n = \fr{(\zeta^{-1} q;q^2)_{\infty}
	(\zeta q;q^2)_{\infty}}{(q;q^2)_{\infty} (q;q^2)_{\infty}}\;,
\tag3.16
$$
and in this form we see that it is a specialization of the $q$-analog
of Gauss's theorem \c{A2; p. 20, Cor. 2.4, $q$ replaced by $q^2$,
$a = \zeta$, $b = \zeta^{-1}$, $c = q$}.

Identity (3.13) may be rewritten as follows:
$$
\align
	\lim_{t\rightarrow 0} \; & \sumty\; \fr{(\zeta/q;q^2)_n}{(q^2;q^2)_n}
	\;\fr{(1 - \zeta q^{2n-1})}{(1 - \zeta/q)} \; \fr{(\zeta;q^2)_n^2}
	{(q;q^2)_n^2} \; \fr{(t^{-1};q^2)_n}{(\zeta qt;q^2)_n}\;
	t^n q^n \zeta^{-n}    \tag3.17   \\
	& = \fr{(\zeta q;q^2)_{\infty}(\zeta^{-1} q;q^2)_{\infty}}
	{(q;q^2)_{\infty}(q;q^2)_{\infty}}\;, 
\endalign
$$
and in this form we see that it is a specialization of the limiting
form of Jackson's theorem \c{G-R; p. 238, eq. (II.20)}.

To obtain (3.14) we note that
$$
	(-\zeta)^n + (-\zeta)^{-n} = \cases
		2 \qquad & \text{ if } n \equiv 0 \pmod{3}  \\
		-1 \qquad & \text{ otherwise}\,.
	\endcases
\tag3.18
$$
Hence by Jacobi's triple product
$$
\align
	& \fr{(\zeta q;q^2)_{\infty}(\zeta^{-1}q;q^2)_{\infty}}
		{(q;q^2)_{\infty}^2} = \fr{(q^2;q^2)_{\infty}
		(\zeta q;q^2)_{\infty}(\zeta^{-1}q;q^2)_{\infty}}
		{(q^2;q^2)_{\infty} (q;q^2)_{\infty}^2}
		\tag3.19 	
	\\
	& = \fr{\displaystyle{\sum_{n=-\infty}^{\infty}} (-1)^n
		q^{n^2} \zeta^n}{\displaystyle{\sum_{n=-\infty}^{\infty}} 
		(-1)^n q^{n^2}}  
	\\
	& = \fr{1 + \displaystyle{\sum_{n=1}^{\infty}} q^{n^2}
		((-\zeta)^n + (-\zeta)^{-n})}{1 + 2 
		\displaystyle{\sum_{n=1}^{\infty}} (-1)^n q^{n^2}}
	\\
	& = \fr{\displaystyle{\sum_{n=-\infty}^{\infty}} q^{(3n)^2}
		- \displaystyle{\sum_{n=-\infty}^{\infty}} q^{(3n+1)^2}}
		{1 + 2 \displaystyle{\sum_{n=1}^{\infty}}(-1)^n q^{n^2}}\;.
\endalign
$$
\pf
\enddemo

From (3.14) we immediately deduce a computationally fast recurrence
for $PDO(n)$.  

\proclaim
{Corollary 3.1}  For $n \geqq 0$
$$
	PDO(n) + 2 \sum_{j\geqq 1} PDO(n-j^2) (-1)^j = \cases
	0 \quad \text{ if } n \text{ is not a square}    \\
	1 \quad \text{ if } n = 0   \\
	2 \quad \text{ if } n = (3m)^2   \\
	-1 \quad\text{ otherwise}\,.
	\endcases
\tag3.20
$$
\endproclaim

\proclaim
{Corollary 3.2}  $PDO(n)$ is odd precisely when either $n$ is $0$
or a square not divisible by $3$.
\endproclaim

\demo
{Proof}  By (3.14)
$$
	pdo(q) \equiv 1 - \sum_{n=-\infty}^{\infty} q^{(2n+1)^2}
	\qquad \pmod{2}\,.
$$
\pf
\enddemo

\subhead
4. \ Relations to Other Partition Functions
\endsubhead

In this section we shall consider three classical partition functions
with an extensive literature.

$M(q)$ will denote the generating function for partitions without $1$'s
and without any two parts differing by $1$.  It is a theorem of
MacMahon \c{M1; p. 54} (see also \c{A1}) that
$$
	M(q) = \prod_{\Sb n = 1 \\ n \not\equiv \pm \pmod{6} \endSb}^{\infty}
	(1 - q^n)^{-1} = \fr{(-q^3;q^3)_{\infty}}{(q^2;q^2)_{\infty}}
\tag4.1
$$

$S(q)$ will denote the generating function for partitions in 
which parts differ by at least 3 and multiples of 3 differ by at
least 4.  Schur \c{S} has shown that
$$
	S(q) = \prod_{\Sb n = 1 \\ n \not\equiv \pm \pmod{6} \endSb}^{\infty}
	(1 - q^n)^{-1} = \fr{(-q;q)_{\infty}}{(q^3;q^3)_{\infty}}\;.
\tag4.2
$$

Our final generating function, $\Phi_2(q)$, is related to more
complicated objects, generalized Frobenius partitions, and we refer
the reader to \c{A3} for an  introduction of this topic.  $\Phi_2(q)$
is the generating function for generalized Frobenius partitions 
wherein any number can appear on each row as a part at most twice.
In \c{A3; eq. (5.11)}, it is shown that
$$
\align
	\Phi_2(q) & = \prod_{n=1}^{\infty} \fr1{(1-q^n)(1 - q^{12n-10})
	(1 - q^{12n-2})(1 - q^{12n-3})(1 - q^{12n-2})}  \tag4.3
	\\
	& = \fr{\displaystyle{\sum_{m=-\infty}^{\infty}} q^{(3m)^2} 
	- \displaystyle{\sum_{m=-\infty}^{\infty}} q^{(3m+1)^2}}
	{(q;q)_{\infty}^2}
	\\
	& \hskip .5in \text{(by \c{A4; eq. (10.4)})}\,.
\endalign
$$

Given the above generating functions, we can easily combine these
formulas with those of Theorem 2 and 3 to deduce the following:

\proclaim
{Theorem 4}  For $|q| < 1$,
$$
\align
	pd(q) \,(q;q)_{\infty} & = M(q)   \tag4.4    \\
	pd(q) \,S(q) & = \fr1{(q;q)_{\infty}^{2}}   \tag4.5  \\
	pd(q) & = (q^2;q^2)_{\infty} \Phi_2(q)  \tag4.6 
\endalign
$$
\endproclaim

\demo
{Proof}  Identity (4.4) follows from (4.1) and (3.1); identity (4.5)
follows from (4.2) and (3.1).  Finally identity (4.6) follows from
(4.3) and (3.14).   \pf
\enddemo

It should be observed that these relationships are quite simple and
suggest the possibility of our obtaining insight for partitions with
designated summands from these classical partition functions.

\Refs
\widestnumber\key{G-R}
\ref
  \key A1
  \manyby G. E. Andrews
  \paper A generalization of a partition theorem of MacMahon
  \jour J. Comb. Th.
  \vol 3
  \yr 1967
  \pages 100--101
\endref

\ref
  \key A2
  \bysame 
  \paper The Theory of Partitions
  \paperinfo Encycl. Math. and Its Applications, Vol. 2,
	Addison-Wesley, Reading, 1976 [Reissued:  Cambridge
	University Press, Cambridge, 1985, paperback, 1998]
\endref

\ref
  \key A3
  \bysame 
  \paper Generalized Frobenius partitions
  \jour Mem. Amer. Math. Soc.
  \vol 49
  \yr 1984
  \finalinfo No. 301, iv + 44 pp
\endref

\ref
  \key G-R
  \by G. Gasper and M. Rahman
  \paper Basic Hypergeometric Series
  \paperinfo Encyl. Math and Its Applications, Vol. 35, Cambridge
	University Press, Cambridge, 1990
\endref

\ref
  \key G
  \by J. W. L. Glaisher
  \paper Expressions for the first five powers of the series in which
	the coefficients are the sums of the divisors of the exponents
  \paperinfo Messenger of Math., No. 171, July 1885
\endref

\ref
  \key H-R
  \by G. H. Hardy and E. M. Wright
  \paper An Introduction to the Theory of Numbers
  \paperinfo 4th ed., Oxford Univ., Press, London and New York,
	1960
\endref

\ref
  \key M1
  \manyby P. A. MacMahon
  \paper Combinatory Analysis
  \paperinfo Vol. 2, Cambridge University Press, London, 1916
	[Reprinted:  Chelsea, New York, 1960]
\endref

\ref
  \key M2
  \bysame 
  \paper Divisors of numbers and their continuations in the theory
	of partitions
  \jour Proc. London Math. Soc., Ser. 2
  \vol 19
  \yr 1919
  \pages 75--113
\endref

\ref
  \key R
  \by S. Ramanujan
  \paper On certain arithmetical functions
  \paperinfo Proc. Cambridge Phil. Soc., 22 (1916), 159--184.
	[Reprinted:  Collected Papers of S. Ramanujan, Cambridge
	Univ. Press, Cambridge, 1927, pp. 136--162 \{Reissued:
	Chelsea, New York, 1962\}]
\endref

\ref
  \key S
  \by I. Schur
  \paper Zur additiven Zahlentheorie
  \paperinfo S.-B. Preuss. Akad. Wiss. Phys.-Math. Kl., 1926,
	pp. 488--495 [Reprinted:  I. Schur, Gessammelte
	Abhandlungen, Vol. 3, pp. 43--50, Springer, Berlin,
	1973\
\endref
\endRefs
\enddocument