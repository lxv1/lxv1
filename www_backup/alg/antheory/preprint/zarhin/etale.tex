

%latex2e file
%
%\'Etale cohomology and reduction of abelian varieties
%
% A. Silverberg and Yu. G. Zarhin
%
%%%%%%%%%%%%%%%%%%%%%%%%%%%%%%%%%%%%%%%

\documentclass{amsart}

\def\Q{{\mathbf Q}}
\def\Z{{\mathbf Z}}
\def\C{{\mathbf C}}
\def\R{{\mathbf R}}
\def\Gal{\mathrm{Gal}}
\def\End{\mathrm{End}}
\def\Aut{\mathrm{Aut}}
\def\Hom{\mathrm{Hom}}
\def\I{{\mathcal I}}
\def\J{{\mathcal J}}
\def\s{{\mathcal S}}
\def\fchar{\mathrm{char}}
\def\rank{\mathrm{rank}}
\def\GL{\mathrm{GL}}
\def\SL{\mathrm{SL}}
\def\Sp{\mathrm{Sp}}
\def\M{\mathrm{M}}
\def\dim{\mathrm{dim}}
\def\O{{\mathcal O}}
\def\P{{\mathbf P}}
\def\n{{n}}
\def\q{{\mathfrak q}}
\def\a{{\mathfrak a}}
\def\X{{\mathcal X}}
\def\Y{{\mathcal Y}}
\def\T{{T}}
\def\B{{B}}
\def\invlim{{\displaystyle{\lim_{\leftarrow}}}}
\def\bmu{\boldsymbol \mu}
\newtheorem{thm}{Theorem}[section]
\newtheorem{lem}[thm]{Lemma}
\newtheorem{cor}[thm]{Corollary}
\newtheorem{prop}[thm]{Proposition}
\theoremstyle{definition}
\newtheorem{defn}[thm]{Definition}
\newtheorem{ex}[thm]{Example}
\newtheorem{rem}[thm]{Remark}
\newtheorem{rems}[thm]{Remarks}
\hyphenation{semi-stable}

\title[\'Etale cohomology]
{\'Etale cohomology and reduction of abelian varieties}

\author[A.\ Silverberg]{A.\ Silverberg}
\address{Department of Mathematics, Ohio State University, 
231 W.\ 18 Avenue,
Columbus, Ohio 43210--1174, USA}
\email{silver\char`\@math.ohio-state.edu}

\author[Yu. G. Zarhin]{Yu. G. Zarhin}
\address{Department of Mathematics, Pennsylvania State University, 
University Park, PA 16802, USA}
\email{zarhin\char`\@math.psu.edu}

\begin{document}

\begin{abstract}
In this paper we study the \'etale cohomology groups
associated to abelian varieties. We obtain necessary
and sufficient conditions for an abelian variety to have
semistable reduction (or to have purely additive reduction
which becomes semistable over a quadratic extension) 
in terms of the action of the
absolute inertia group on the \'etale 
cohomology groups with finite coefficients.
\end{abstract}

\maketitle

\section{Introduction}
Suppose $X$ is a smooth projective variety over a field $F$,
$v$ is a discrete valuation on $F$, and $\ell$ is
a prime number not equal to the residue characteristic of $v$.
Let $F^s$ denote a separable closure of $F$, let ${\bar v}$
be an extension of $v$ to $F^s$, let $\I$ denote the
inertia subgroup at ${\bar v}$ of $\Gal(F^s/F)$, and let
${\bar X} = X\times_F F^s$. For every
positive integer $k$, the group $\I$ acts naturally
on the \'etale cohomology group 
$H^k_{\text{\'et}}({\bar X},\Q_\ell)$.
Grothendieck proved the Monodromy Theorem (see the Appendix to
\cite{SerreTate}, and 1.2 and 1.3 of
\cite{SGADeligne}), which says that $\I$ acts 
on $H^k_{\text{\'et}}({\bar X}, \Q_\ell)$ 
via quasi-unipotent operators, i.e.,
for every $\sigma\in\I$ we have 
$$(\sigma^m-1)^{r} H^k_{\text{\'et}}({\bar X},\Q_\ell)
=0$$
for some positive integers $m$ and $r$.
It is known (see 3.7 of \cite{SGADeligne}, and 
3.5 and 3.6 of \cite{SGA7}) 
that if $k=1$, then one may take $r=2$. 
It easily follows (see 
Theorem \ref{highercoh} below) that if 
$X$ is an abelian variety, then one may take $r=k+1$.
It is shown in \cite{SGA7I} (see 3.4 and 3.8 of \cite{SGADeligne},
and p.~VI of \cite{SGAintro})
that one may take $r=k+1$ whenever one knows the Purity
Conjecture (3.1 of \cite{SGADeligne}) and resolution of singularities.

 From now on, suppose $X$ is an abelian variety.
The N\'eron model $\X$ of $X$ at $v$ is a smooth 
separated model of $X$ over the valuation ring $R$
such that for every smooth scheme $\Y$ over $R$ and morphism 
$f : \Y \otimes_R F \to X$ over $F$ there is a unique morphism
$f_R : \Y \to \X$ over $R$ which extends $f$. The generic fiber of $\X$
can be canonically identified with $X$, and $\X$ is a commutative group
scheme over $R$ whose group structure extends that of $X$.
Let $X_v^0$ denote the identity component of 
the special fiber of $\X$ at $v$. 
Over an algebraic closure of the residue field, there is
an exact sequence of algebraic groups 
$$0 \to U \times \T \to X_{v}^{0} \to \B \to 0,$$
where $\B$ is an abelian variety, $\T$ is the maximal
algebraic torus in $X_{v}^{0}$, 
and $U$ is a unipotent group.
By definition, $X$ has {\it semistable reduction} at $v$ 
if and only if $U=0$.
As $\I$-modules,
$H^1_{\text{\'et}}({\bar X}, \Q_\ell)$
and the $\ell$-adic Tate module $V_{\ell}(X)$ are isomorphic.
Grothendieck's Galois Criterion for Semistable Reduction 
says that $X$ has semistable reduction at $v$ if and only if
every $\sigma\in\I$ acts on  $V_{\ell}(X)$ 
as a unipotent operator of echelon $\le 2$, i.e.,
if and only if
$$(\sigma-1)^2H^1_{\text{\'et}}({\bar X}, \Q_\ell)=0$$
for every $\sigma\in\I$.

Suppose $5 \le n \in \Z$. The authors 
(see \cite{degree} and \cite{banff})  proved that
$X$ has semistable reduction at $v$ if and only if
$$(\sigma-1)^{2} H^1_{\text{\'et}}({\bar X}, \Z/n\Z)
=0$$
for every $\sigma \in \I$.
In other words, necessary and sufficient conditions for 
semistability can be read off not only
from the $\ell$-adic representation, as shown by
Grothendieck, but also from the mod $n$ representation
(for $n \ge 5$).
The aim of this paper is to generalize this result to the 
case of higher $k$ (and all $n>1$). 
Assume that $0<k<2\dim(X)$, that $k+1 \le r\in\Z$, and that 
$n$ is a positive integer which does not belong
to a certain finite set  $N(r)$ of prime powers, 
defined explicitly in terms of $r$ in \S\ref{notation}.
(For example, $N(2)=\{1,2,3,4\}$.)  We show that if $k$ is odd, 
then $X$ has semistable reduction at $v$ if and only if
$$(\sigma-1)^{r} H^{k}_{\text{\'et}}({\bar X}, \Z/n\Z)
=0$$
for every $\sigma \in \I$.
If $k$ is even, we show  
(under a certain additional assumption on $v$) that
$$(\sigma-1)^{r} H^{k}_{\text{\'et}}({\bar X}, \Z/n\Z)
=0$$
for every $\sigma \in \I$
if and only if either $X$ has semistable reduction 
at $v$ or
$X$ has purely additive reduction at $v$ but acquires 
semistable reduction
over a (ramified) quadratic extension of $F$.

We hope that our results and/or methods will be useful in the study of
semistability for the more general class of motives
\cite{Motives}.

Silverberg would like to thank the IHES and the Bunting
Institute for their hospitality, and the NSA and the Science Scholars
Fellowship Program at the Bunting Institute for financial support.
Zarhin would like to thank the NSF for financial support. 

\section{Notation and definitions}
\label{notation}

If $F$ is a field, let $F^s$ denote a separable closure. 
Throughout this paper, 
$X$ is a $d$-dimensional abelian variety defined over $F$, and 
$v$ is a discrete valuation on $F$ of residue characteristic $p \ge 0$.
Let ${\bar v}$ denote an extension of $v$ to $F^s$, and 
let $\I$ denote the inertia subgroup at ${\bar v}$ of $\Gal(F^s/F)$.
If $\ell$ is a prime not equal to the characteristic of $F$,
let
$$\rho_{\ell,X} : \Gal(F^s/F) \to \Aut(T_\ell(X)) \cong \M_{2d}(\Z_\ell)$$
denote the $\ell$-adic representation on the Tate module $T_\ell(X)
= \invlim X_{\ell^n}$. 
Let $V_{\ell}(X)=T_\ell(X)\otimes_{\Z_{\ell}}\Q_{\ell}$
and let ${\bar X} = X \times_F F^s$.

\begin{defn}
We say $v$ satisfies (*) if at least one of the following two
conditions is satisfied:
\begin{enumerate}
\item[(a)]  $p \ne 2$, 
\item[(b)] the valuation ring 
is henselian and the residue field is separably closed.
\end{enumerate}
\end{defn}

\begin{defn}
If $r$ is a positive integer, define a finite set 
of prime powers $N(r)$ by 
$$N(r) = \{\text{prime powers $\ell^m : 0 \le m(\ell - 1) \le r $}\}.$$
If $n$ is a positive integer, let $R(r,n) = 1$ if $n \notin N(r)$, 
and let
$$R(r,n) = \ell^{s(r,n)} \quad 
\mbox{where} \quad 
s(k,n) = \max\{s \in \Z^+ : m(\ell - 1)\ell^{s-1} \leq r\}$$
if $1 \neq n = \ell^m \in N(r)$ with $\ell$ a prime.
\end{defn}
For example, $$N(1) = \{1, 2\}, \quad N(2) = \{1, 2, 3, 4\},$$
$$N(3) = \{1, 2, 3, 4, 8\},  \quad  
N(4) = \{1, 2, 3, 4, 5, 8, 9, 16\};$$
$$R(1,2) = 2, \quad \mbox{$R(1,n) = 1 $ if $n \ge 3$},$$
$$R(2,2) = 4, \quad R(2,3) = 3, \quad 
R(2,4) = 2, \quad \mbox{and} \quad \mbox{$R(2,n) = 1$ if $n \ge 5$}.$$

\section{Lemmas}

\begin{thm}[Galois Criterion for Semistable Reduction]
\label{galcrit}
Suppose $\ell$ is a prime and $\ell \ne p$. Then the
following are equivalent:
\begin{enumerate}
\item[(i)] $X$ has semistable reduction at $v$,
\item[(ii)] $\I$ acts unipotently on $T_\ell(X)$; i.e.,
all the eigenvalues of $\rho_{\ell,X}(\sigma)$ are $1$, 
for every $\sigma \in \I$,
\item[(iii)] for every $\sigma \in \I$, 
$(\rho_{\ell,X}(\sigma) - 1)^2 = 0$.
\end{enumerate}
\end{thm}

\begin{proof}
See 3.5 and 3.8 of \cite{SGA7}, 
and Theorem 6 on p.~184 of \cite{BLR}.
\end{proof}

\begin{thm}[Theorem 6.7 of \cite{serrelem}]
\label{localglobal}
Suppose $\O$ is an integral domain of characteristic zero,
$n$, $r$, and $g$ are positive integers 
such that no rational prime which divides $n$ is a unit in $\O$,
$A \in M_g(\O)$ satisfies $(A - 1)^r \in nM_g(\O)$, and $\lambda$ is
an eigenvalue of $A$ which is a root of unity.
Then $\lambda^{R(r,n)} = 1$; in particular, $\lambda = 1$ if 
$n \notin N(r)$.
\end{thm}

\begin{lem}[Lemma 4.2 of \cite{degree}]
\label{mevals}
Suppose $m$ is a positive integer, $\ell$ is a prime,
$p$ does not divide $m\ell$, and
$K$ is a degree $m$ extension of $F$
which is totally ramified above $v$. 
Suppose that for every $\sigma \in \I$,
all the eigenvalues of $\rho_{\ell,X}(\sigma)$ 
are $m$-th roots of unity. Then $X$ has
semistable reduction at the extension of $v$ to $K$.
\end{lem}

\begin{rem}
\label{ramifiedcyclic}
Suppose $m$ is a positive
integer not divisible by $p$. If $F(\zeta_m) = F$,
then $F$ has a cyclic extension of degree $m$ which is totally ramified at
$v$. In particular, if $p \ne 2$ then 
$F$ has a quadratic extension which is (totally and tamely) ramified at $v$.
If the valuation ring is henselian and the residue field is separably closed, 
then $F= F(\zeta_m)$ and therefore $F$ has a cyclic totally ramified extension 
of degree $m$. (See Remark 5.3 of \cite{semistab}.)
Note also that $F$ has no non-trivial unramified extensions if and only if 
the valuation ring is henselian and the residue field is separably closed.
\end{rem}

\begin{lem}[Corollary 1.10 of \cite{LenstraOort}]
\label{padd}
Let $\J \subset \I$ denote the first ramification group,
and let $\tau$ be a lift to $\I$ of a topological generator
of the procyclic group $\I/\J$.
Then $X$ has purely additive reduction at $v$ if and only if 
$1$ is not an eigenvalue for the action of $\tau$ on $V_{\ell}(X)^\J$.
\end{lem}

\section{Some linear algebra}
\label{linalgsect}
Suppose $V$ is a finite-dimensional vector space over a field 
of characteristic zero. 
Suppose $k$ is an integer and $0 < k < \dim(V)$.
Let 
$$f_{k} : \SL(V) \to \GL(\wedge^k V^\ast)$$
denote the natural representation.
Suppose $G$ is a finite group and $\rho:G \to \SL(V)$ is a
representation.
Let $\rho_{k} = f_{k}\circ\rho$.
For $g \in G$, let 
$\rho(g)=s_{g}u_{g} = u_{g}s_{g}$ 
be the Jordan decomposition,
where $u_{g}$, $s_{g} \in \SL(V)$, 
$u_{g}$ is unipotent, and
$s_{g}$ is semisimple.
Then $\rho_{k}(g) = f_k(\rho(g)) =
f_k(s_{g})f_k(u_{g}) = f_k(u_{g})f_k(s_{g})$,
$f_k(u_{g})$ is unipotent, and $f_k(s_{g})$ is semisimple. 

\begin{lem}
\label{lin}
\begin{enumerate}
\item[{(i)}] The kernel of $f_{k}$
is $\{1\}$ if $k$ is odd and is $\{1, -1\}$ if $k$ is even. 
\item[{(ii)}] If $g \in G$ and $(\rho(g)-1)^{2}=0$, then
$(\rho_{k}(g) - 1)^{k+1} = 0$.
\item[{(iii)}] If $k$ is even,
$g \in G$, and $(\rho(g)^{2}-1)^{2}=0$, then
$(\rho_{k}(g) - 1)^{k+1} = 0$.
\end{enumerate} 
\end{lem}

\begin{proof}
Part (i) is easy. 
Note that 
$$(\rho_{k}(g)-1)(v_{1}\wedge v_{2}\ldots \wedge v_{k})
= (\rho(g)v_{1}\wedge\ldots\wedge \rho(g)v_{k}) - 
(v_{1}\wedge v_{2}\ldots \wedge v_{k}).$$
Part (ii) follows by substituting $\rho(g)=1+\eta$.
If $(\rho(g)^{2}-1)^{2}=0$,
then the only eigenvalues
of $\rho(g)$ are $\pm 1$. If $k$ is even, then
$\rho_{k}(g)$ is unipotent. Therefore,
$f_{k}(s_{g})=1$. Thus $s_{g} = 1$ or $-1$, and
there exists $\eta \in \End(V)$ such that
$\eta^2=0$ and $\rho(g) = {\pm}1+\eta$.
It again follows easily that
$(\rho_{k}(g) - 1)^{k+1} = 0$.
\end{proof}

\section{Main results}

\begin{thm}
\label{highercoh}
Suppose $X$ is an abelian variety over a field $F$, and $v$ is a discrete 
valuation on $F$ of residue characteristic $p \ge 0$.
Suppose $k$ and $r$ are positive integers, and
$k < 2\dim(X)$.
\begin{enumerate}
\item[{(i)}] Suppose $k<r\in\Z$.
If either $X$ has semistable reduction at $v$, 
or $k$ is even and 
$X$ has purely additive reduction at $v$ which becomes
semistable over a quadratic extension of $F$,
then 
$$(\sigma - 1)^{r}H^k_{\text{\'et}}({\bar X}, \Z_{\ell}) = 0$$ 
for every $\sigma \in \I$ and every prime $\ell \ne p$,
and
$$(\sigma - 1)^{r}H^k_{\text{\'et}}({\bar X}, \Z/n\Z) = 0$$ 
for every $\sigma \in \I$ and every positive integer $n$ not
divisible by $p$.
\item[{(ii)}] Suppose $n$ is a positive integer not divisible by
$p$, and 
$$(\sigma - 1)^{r}H^k_{\text{\'et}}({\bar X}, \Z/n\Z) = 0$$ 
for every $\sigma \in \I$. 
Suppose $L$ is a degree $R(r,n)$ extension of $F$ which is
totally ramified above $v$, and let $w$ be the extension of
$v$ to $L$.
If $k$ is odd, then $X$ has semistable reduction at $w$.
If $k$ is even and $v$ satisfies (*), 
then either $X$ has semistable reduction at $w$, or
$X$ has purely additive reduction at $w$ which becomes
semistable over a quadratic extension of $L$.
\end{enumerate}
\end{thm}

\begin{proof}
Let $V = V_\ell(X)$ and let 
$H^k_\ell = H^k_{\text{\'et}}({\bar X}, \Z_\ell)$.
Note that $H^k_\ell$ is a free $\Z_\ell$-module, and
$H^k_\ell \otimes_{\Z_\ell}\Q_\ell = \wedge^kV^\ast$.
Let
$$\rho_{\ell,k} : \I \to \Aut(H^k_\ell) \cong \GL_b(\Z_\ell)$$
be the representation giving the action of $\I$ on $H^k_\ell$,
where $b=\rank_{\Z_{\ell}}(H^k_\ell)$.
In the notation of \S\ref{linalgsect},
$\rho_{\ell,k}=f_{k}\circ\rho_{\ell,X}$.
By the Galois equivariance of the Weil pairing, and the fact that
the inertia group acts as the identity on the $\ell$-power roots
of unity, we see that the image of $\rho_{\ell,X}$ lies in the
symplectic group $\Sp(V) \subseteq \SL(V)$. 
Suppose $\sigma \in \I$. 

Suppose $X$ has semistable reduction at $v$.
By Theorem \ref{galcrit}, $(\rho_{\ell,X}(\sigma) - 1)^{2} = 0$.
By Lemma \ref{lin}ii, $(\rho_{\ell,k}(\sigma) - 1)^{k+1} = 0$.

Suppose $k$ is even and
$X$ has purely additive reduction at $v$ which becomes
semistable over a quadratic extension of $F$. 
By Theorem \ref{galcrit}, 
$(\rho_{\ell,X}(\sigma)^2-1)^2=0$. 
By Lemma \ref{lin}iii, $(\rho_{\ell,k}(\sigma) - 1)^{k+1} = 0$.

Therefore,  $(\rho_{\ell,k}(\sigma) - 1)^{r} = 0$.

The second part of (i) follows from the first part for all
the prime divisors $\ell$ of $n$.

Conversely, suppose that 
$(\sigma - 1)^{r}H^k_{\text{\'et}}({\bar X}, \Z/n\Z) = 0$. 
Let $\ell$ be a prime divisor of $n$. Then
$$(\sigma - 1)^{r}H^k_\ell \subseteq nH^k_\ell,$$
so
$$(\rho_{\ell,k}(\sigma) - 1)^{r} \in n\M_b(\Z_\ell).$$
Since $X$ has semistable reduction over a finite extension of $F$,
by Lemma \ref{lin}ii there is a positive integer $m$ such that
$(\rho_{\ell,k}(\sigma^m) - 1)^{k+1} = 0$.
Let $\alpha$ be an eigenvalue of 
$\rho_{\ell,k}(\sigma)$. 
Then $(\alpha^m - 1)^{k+1} = 0$, so $\alpha^m = 1$.
Let $R=R(r,n)$.
By Theorem \ref{localglobal},
we have $\alpha^{R} = 1$.
Then $\rho_{\ell,k}(\sigma^{R})$ is unipotent.
In the notation of \S\ref{linalgsect} 
with $\rho=\rho_{\ell,X}$, 
we have
$f_k(s_{\sigma^{R}}) = 1$.
By Lemma \ref{lin}i,
if $k$ is odd then $s_{\sigma^{R}} = 1$, and
if $k$ is even then $s_{\sigma^{R}} \in \{\pm 1\}$.
It follows that if $k$ is odd, then $\sigma^{R}$ acts unipotently on
$V$. By Lemma \ref{mevals}, $X$ has semistable reduction at $w$.

Suppose that $k$ is even, and
$X$ does not have semistable reduction at $w$.
Suppose $L$ is a degree $R(r,n)$ extension of $F$ which is
totally ramified above $v$, and let $w$ be the extension of
$v$ to $L$.
We may assume that $L \subset F^{s}$, and that ${\bar v}$ is an
extension of $w$ to $F^{s}$. 
Let $\I_{w}$ denote the inertia
subgroup at ${\bar v}$ of $\Gal(F^{s}/L)$.
Let $\J$ (resp., $\J_{w}$) denote the first ramification subgroup
in $\I$ (resp., $\I_{w}$), and let
$\tau$ denote a lift to $\I$ of a topological generator of the 
procyclic group $\I/\J$.
Then $\tau^{R}$ is a lift to $\I_{w}$ of
a topological generator of $\I_{w}/\J_{w}$.

Suppose the valuation ring is henselian and the residue field is 
separably closed. Then $\I_{w} \cong \Gal(F^{s}/L)$, 
and the map $\sigma \mapsto s_{\sigma}$
defines a quadratic character 
$$\chi : \Gal(F^s/F) \to \{\pm 1\} \subseteq \Aut(X)$$
such that $X$ has semistable reduction over the quadratic extension
of $L$ cut out by $\chi$.
If $s_{\tau^R}=1$, then $s_\sigma=-1$ for some $\sigma\in\J_w$
(since $X$ does not have semistable reduction at $w$).
But then $V^{\J_w}=0$, so $X$ has purely additive 
reduction at $w$ by Lemma \ref{padd}.

Suppose instead that $p \ne 2$.
Since all the eigenvalues of $\rho_{\ell,X}(\sigma^{R})$
are square roots of unity (for all $\sigma \in \I$), 
$X$ has semistable reduction 
over a quadratic extension of $L$, by Lemma \ref{mevals} and
Remark \ref{ramifiedcyclic}.
We showed above that $s_{\sigma}^{R} = s_{\sigma^{R}} \in \{\pm 1\}$.
Since $2R$ is not divisible by $p$, and 
$\J_{w}$ is either trivial or a pro-$p$ group, 
therefore $\J_{w}$ acts unipotently on $V$.
Since $X$ does not have semistable reduction at $w$,
it follows that $s_{\tau^{R}}=-1$.
Therefore, $1$ is not an eigenvalue for the action of $\tau^{R}$
on $V$. By Lemma \ref{padd}, 
$X$ has purely additive reduction at $w$.
\end{proof}

If we restrict to the case where $n \notin N(r)$, we obtain
the following result. This result gives necessary and sufficient
conditions for semistable reduction, and also necessary and
sufficient conditions for $X$ to have either semistable reduction
or purely additive reduction which becomes semistable over a
quadratic extension.

\begin{cor}
\label{highercohcor}
Suppose $X$ is an abelian variety over a field $F$, 
suppose $v$ is a discrete valuation on $F$ of residue 
characteristic $p \ge 0$, suppose $k$, $n$, and $r$
are positive integers, and suppose $\ell$ is a prime number.
Suppose $k < 2\dim(X)$, 
suppose $k<r$, suppose $n$ and $\ell$ are not divisible by $p$, 
and suppose $n \notin N(r)$.
\begin{enumerate}
\item[{(i)}] Suppose $k$ is odd. Then the following are
equivalent:
\begin{enumerate}
\item[{(a)}]  $X$ has semistable reduction at $v$,
\item[{(b)}] for every $\sigma \in \I$,
$$(\sigma - 1)^{r}H^k_{\text{\'et}}({\bar X}, 
\Z_{\ell}) = 0,$$ 
\item[{(c)}]  for every $\sigma \in \I$,
$$(\sigma - 1)^{r}H^k_{\text{\'et}}({\bar X}, \Z/n\Z) = 0.$$ 
\end{enumerate}
\item[{(ii)}] Suppose $k$ is even and $v$ satisfies (*).
Then the following are equivalent:
\begin{enumerate}
\item[{(a)}]  either $X$ has semistable reduction at $v$, or
$X$ has purely additive reduction at $v$ which becomes
semistable over a quadratic extension of $F$,
\item[{(b)}] for every $\sigma \in \I$,
$$(\sigma - 1)^{r}H^k_{\text{\'et}}({\bar X}, \Z_{\ell}) = 
0,$$
\item[{(c)}] for every $\sigma \in \I$,
$$(\sigma - 1)^{r}H^k_{\text{\'et}}({\bar X}, \Z/n\Z) = 0.$$ 
\end{enumerate}
\end{enumerate}
\end{cor}

\begin{proof}
Apply Theorem \ref{highercoh}. 
Since $n \notin N(r)$, we have $R(r,n)=1$ and $L=F$.
\end{proof}

Next we specialize Theorem \ref{highercoh}ii to the case $r=1$.
Note that when $k=1$ we recover Raynaud's criterion for
semistable reduction (Proposition 4.7 of \cite{SGA7}).

\begin{cor}
\label{raynaudgen}
Suppose $X$ is an abelian variety over a field $F$, 
$v$ is a discrete valuation on $F$, 
$k$ and $n$ are positive integers, 
$k < 2\dim(X)$, $n \ge 3$, %and
$n$ is not divisible by the residue characteristic, and
$\I$ acts as the identity on 
%$$(\sigma - 1)
$H^k_{\text{\'et}}({\bar X}, \Z/n\Z)$.
%for every $\sigma \in \I$. 
If $k$ is odd, then $X$ has semistable reduction at $v$.
If $k$ is even and $v$ satisfies (*), 
then either $X$ has semistable reduction at $v$, or
$X$ has purely additive reduction at $v$ which becomes
semistable over a quadratic extension of $F$.
\end{cor}

\begin{thebibliography}{99}

\bibitem{BLR} S.\ Bosch, W.\ L\"utkebohmert, M.\ Raynaud, N\'eron models,
Springer, Berlin-Heidelberg-New York, 1990.
\bibitem{SGAintro} P.\ Deligne, 
Introduction to \cite{SGA7I}, pp.\ V--VII.
\bibitem{SGADeligne} P.\ Deligne, 
{\em R\'esum\'e des premiers expos\'es de A.\ Grothendieck},
in \cite{SGA7I}, pp.\ 1--24.
\bibitem{SGA7} A.\ Grothendieck, {\em Mod\`eles de N\'eron et monodromie},
in \cite{SGA7I}, pp.\ 313--523.
\bibitem{SGA7I} A.\ Grothendieck (ed.), 
Groupes de monodromie en g\'eometrie alg\'ebrique, SGA7 I,
Lecture Notes in Math.\ {\bf 288}, Springer,
Berlin-Heidelberg-New York, 1972.
\bibitem{LenstraOort} H.\ W.\  Lenstra, Jr., F.\ Oort, {\em Abelian varieties
having purely additive reduction}, J.\ Pure and Applied Algebra {\bf 36}
(1985), 281--298.
\bibitem{Motives} J-P.\ Serre, {\em Propri\'et\'es conjecturales 
des groupes de Galois motiviques et des repr\'esentations $\ell$-adiques}, 
in Motives 
(U.\ Jannsen, S.\ Kleiman, J-P.\ Serre, eds.), Proc.\ Symp.\ Pure 
Math.\ {\bf 55} (1994), Part 1, pp.~377--400.
\bibitem{SerreTate} J-P.\ Serre, J.\ Tate, {\em Good reduction of abelian 
varieties}, Ann.\ of Math.\ {\bf 88} (1968), 492--517. 
\bibitem{semistab} A.\ Silverberg, Yu.\ G.\ Zarhin, 
{\em Semistable reduction 
and torsion subgroups of abelian varieties}, 
Ann.\ Inst.\ Fourier {\bf 45}, no.~2 (1995), 403--420.
\bibitem{serrelem} A.\ Silverberg, Yu.\ G.\ Zarhin, 
{\em Variations on a theme of Minkowski and Serre},
J.\ Pure and Applied Algebra {\bf 111}  (1996),  285--302.
\bibitem{degree} A.\ Silverberg, Yu.\ G.\ Zarhin, 
{\em Semistable reduction of abelian varieties over extensions of 
small degree},
J.\ Pure and Applied Algebra, to appear.
\bibitem{banff} A.\ Silverberg, Yu.\ G.\ Zarhin, 
{\em Reduction of abelian varieties}, 
submitted to the Proceedings of the NATO/CRM Conference  
on the Arithmetic and Geometry of Algebraic Cycles.


\end{thebibliography}

\end{document}

