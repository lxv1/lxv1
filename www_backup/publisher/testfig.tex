\magnification=1100

\hsize=4.5 true in
\vsize=7.5 true in

\def\blackbox{\vrule height 1.2ex width 1.0ex depth -.2ex}

 \input psfig 
% at least 20 problems per chapter
% $10 per book
%Illistrator for figures
% contract in November, this Dec complete

\font\ch=cmbx10 at 18truept
\font\chtitle=cmbx10 at 21truept
\font\fourteenbf=cmbx10 scaled\magstep2
\font\twelvebf=cmbx10 scaled\magstep1
%\magnification=\magstep1
\catcode`\@=11
\def\m@th{\mathsurround=0pt }

%this is the code to write matrices 

\def\matriz#1{\,\vcenter{\normabaselines\m@th
\ialign{\hfil$##$\hfil&&\quad\hfil$##$\hfil\crcr
\mathstrut\crcr\noalign{\kern-\baselineskip}
#1\crcr\mathstrut\crcr\noalign{\kern-\baselineskip}}}\,}

%this is the code to write tableaux with square brackets 
 
\newdimen\p@renwd\setbox0=\hbox{\tenex B}\p@renwd=\wd0 
\def\rowtab#1{\begingroup \m@th\setbox0=\vbox{\def\cr{\crcr\noalign{\kern2pt\global\let\cr=\endline}}
\ialign{$##$\hfil\kern2pt\kern\p@renwd&\thinspace\hfil$##$\hfil
&&\quad\hfil$##$\hfil\crcr
\omit\strut\hfil\crcr\noalign{\kern-\baselineskip}
#1\crcr\omit\strut\cr}}
\setbox2=\vbox{\unvcopy0 \global\setbox1=\lastbox}
\setbox2=\hbox{\unhbox1 \unskip \global\setbox1=\lastbox}
\setbox2=\hbox{$\kern\wd1\kern-\p@renwd \left [ \kern-\wd1
\global\setbox1=\vbox{\box1\kern2pt}
\vcenter{\kern-\ht1 \unvbox0 \kern-\baselineskip} \,\right]$}
\;\vbox{\kern\ht1\box2}\endgroup}

\def\bull{\vrule height .9ex width .9ex depth -.05ex }

\pageno=166
% till page 95, but a lot of empty space in pictures
% \def\title{\hfil{\fourteenbf Chapter VI:  Transportation Problems}\hfil}
 
\headline={\ifnum\pageno=166 \else\ifodd\pageno\rightheadline\else \leftheadline\fi\fi}
 
\def\rightheadline{\tenrm\hfil{\it\S  16. Phase 1}\quad {\bf\folio}}
\def\leftheadline{\tenrm{\bf\folio}\quad{\it Chapter 6  Transportation Problems} \hfil}
\voffset=2\baselineskip

 \bigskip
 \bigskip
 \bigskip
\noindent
\noindent  {\ch  Chapter} {\chtitle 6}
\bigskip
 \bigskip
\noindent 
{\chtitle Transportation Problems}
 \medskip
\hrule
 \bigskip
\bigskip

\noindent
{\twelvebf   \S  16. Phase 1}

\smallskip
\noindent 

	Thus, $w_{ij}=c_{ij}+u_i-v_j=0$ for all chosen positions $(i,j)$.  It turns out that these equations determine potentials uniquely up to an additive constant.  This follows from the fact that the chosen positions determine a tree in the graphical representation (Figure 16.5).
\vskip -2.5in
$$\psfig{figure=Fig.16.5b.eps,height=4in,width=4in}$$
\centerline{{\bf Figure 16.5.}  Tree}
\centerline{ for the feasible solution of Example 15.7}  
\filbreak

\end

tex testfig

dvips -o testfig.ps testfig.dvi

ps2pdf testfig.ps
